\subsection{Теорема существования и единственности решения задачи Коши для нормальной системы дифференциальных уравнений}

\begin{definition}
	Система вида
	\begin{equation}
		\label{equ:norm-sys}
		\begin{cases*}
			\dot{x}^1 = f^1(t, \bar{x}) \\
			\dot{x}^2 = f^2(t, \bar{x}) \\
			... \\
			\dot{x}^n = f^n(t, \bar{x})
		\end{cases*}
	\end{equation}
	называется нормальной системой дифференциальных уравнений n-ого порядка.
	
\end{definition}

\begin{definition}
	Система
	\begin{equation}
		\label{equ:init-cond}
		\begin{cases*}
			x^1(t_0) = x_0^1 \\
			x^2(t_0) = x_0^2 \\
			... \\
			x^n(t_0) = x_0^n
		\end{cases*}
	\end{equation}
	называется начальным условием
\end{definition}	

\begin{proposition}
	Решить задачу Коши означает решить нормальную систему дифференциальных уравнений при заданном начальном условии
\end{proposition}

\begin{theorem}[Теорема Коши о существовании и единственности решения]
	Пусть $\forall i, j \hm= \overline{1, n}$ функции $f^i, \dfrac{\partial{f^i}}{\partial{x^j}}$ непрерывны в области $\Omega \subset \mathbb{R}^{n+1}$, тогда, $\forall (t_0, \overline{x_0}) \in \Omega \text{ } \exists h>0 : \forall t \hm\in [t_0-h, t_0+h]$ решение задачи Коши существует и единственно.
\end{theorem}

\begin{lemma}
	Если $\bar{f}(t, \bar{x})$ - непрерывны на $\Omega$, то система уравнений
	\begin{equation}
		\label{equ:int-sys}
		\overline{x}(t) = \overline{x_0} + \int\limits_{t_0}^t\bar{f}(\tau, \overline{x}(\tau))d\tau
	\end{equation}
	эквивалентна задаче Коши.
\end{lemma}

\begin{proof}
	Пусть $\varphi(t)$ - решение (\ref{equ:norm-sys}) при условии (\ref{equ:init-cond}), тогда
	\[
		\dot{\varphi}^i = f^i(t, \varphi^1(t), \dots, \varphi^n(t))
	\]
	Проинтегрируем полученное равенство по отрезку $[t_0, t]$
	\begin{align*}
		&\int\limits_{t_0}^t \dot{\varphi}^i(\tau)d\tau = \int\limits_{t_0}^t f^i(\tau, \varphi^1(\tau), \dots, \varphi^n(\tau)) d\tau \\
		&\varphi^i(t) - \varphi^i(t_0) = \int\limits_{t_0}^t f^i(\tau, \bar\varphi(\tau))d\tau \\
		&\varphi^i(t) = x_0^i + \int\limits_{t_0}^t f^i(\tau, \bar\varphi(\tau))d\tau
	\end{align*}
	Теперь пусть $\bar{\varphi}(t)$ - решение (\ref{equ:int-sys}). Тогда
	\[
		\varphi^i(t) \equiv x_0^i + \int\limits_{t_0}^t f^i(\tau, \bar{\varphi}(\tau))d\tau
	\]
	Отсюда видно, что функция $\varphi^i(t)$ - дифференцируемы. Тогда
	\begin{equation}
	\begin{cases*}
		\dot{\varphi}^i(t) = f^i(t, \bar{\varphi}(t)) \\
		\varphi^i(t_0) = x_0^i
	\end{cases*}
	\end{equation}
\end{proof}

\begin{corollary}
	Из 2 части леммы следует, что решение задачи Коши непрерывно дифференцируемо.
\end{corollary}

Введем оператор $A(\bar{x}) = \bar{x}_0 + \int\limits_{t_0}^t \bar{f}(\tau, \bar{x}(\tau)) d\tau$. Тогда систему интегральных уравнений (\ref{equ:int-sys}) можно записать в виде 
\begin{equation}
	\label{equ:altern}
	\bar{x}(t) = A(\bar{x})
\end{equation}

\begin{lemma}
	\[
	\left\Arrowvert \int\limits_{t_0}^t\bar{x}(\tau)d\tau \right\Arrowvert \le \left\arrowvert \int\limits_{t_0}^t \Arrowvert \bar{x}(\tau)\Arrowvert d\tau \right\arrowvert
	\]
\end{lemma}

\begin{proof}
	\begin{equation}
		\left\arrowvert \int\limits_{t_0}^t x^i(\tau)d\tau \right\arrowvert \le\left\arrowvert \int\limits_{t_0}^t \left\arrowvert x^i(\tau) \right\arrowvert d\tau\right\arrowvert \le \left\arrowvert \int\limits_{t_0}^t \Arrowvert \bar{x}(\tau)\Arrowvert d\tau \right\arrowvert
	\end{equation}
	Таким образом $max\{|\int\limits_{t_0}^t x^i(\tau)d\tau|\} = || \int\limits_{t_0}^t \bar{x}(\tau)d\tau || \le \arrowvert \int\limits_{t_0}^t \Arrowvert \bar{x}(\tau)\Arrowvert d\tau \arrowvert$
\end{proof}

\begin{lemma} (Адамара)
	Пусть $\bar{f}(\bar{x}), \dfrac{\partial f^i}{\partial x_j}$ непрерывны в $\Omega \subset \mathbb{R} $ - замкнутой, ограниченной, выпуклой области. Тогда $\forall i = \overline{1,n}, \bar{y} \in \Omega \hookrightarrow \| \bar{f}(\bar{y}) - \bar{f}(\bar{x}) \| \le n^{3/2}K_1\|\overline{y-x}\|$, где $K_1 = max_{i,j=\overline{1,n}}\{max_{x\in\Omega}\left\{\left|\dfrac{\partial f^i}{\partial x_j}\right|\right\} \}$
\end{lemma}

\begin{proof}
	
	$|\bar{f}| = \sqrt{\sum\limits_{i=1}^n (f^i)^2}$, $\|\bar{f}\|_C = max_{x\in\Omega}\{|\bar{f}(\bar{x})|\}$
	
	$\Omega$ - компакт, поэтому непрерывность частных производных позволяет говорить о существовании $K_1$. Возьмем производные точки $\bar{x}$ и $\bar{y}$ и соединим их отрезком $\bar{x} + t(\bar{y} - \bar{x})$, где $t\in[0, 1]$. Рассмотрим значение компоненты $f^i$ на отрезке:
	\[
		f^i(\bar{x} + t(\bar{y} - \bar{x})) = f^i(t)
	\]
	$f^i(t)$ - дифференцируема, тогда
	\begin{align*}
		&|f^i(\bar{y}) - f^i(\bar{x})| = |f^i(1) - f^i(0)| = \left| \frac{df}{dt}(t^*)\cdot(1-0)\right| = \\
		&= \left|\sum\limits_{j=1}^n\frac{\partial f^i}{\partial x^j}(t^*)\cdot(y^j - x^j)\right| \le \sum\limits_{j=1}^n\left|\frac{\partial f^i}{\partial x^j}(t^*)\right|\cdot\left|(y^j - x^j)\right| \le K_1\|\bar{y}-\bar{x}\|\cdot n
	\end{align*}
	Теперь рассмотрим вектор-функцию
	\begin{align*}
		&|\bar{f}(\bar{y}) - \bar{f}(\bar{x})| = \sqrt{\sum\limits_{k=1}^n (f^k(\bar{y}) - f^k(\bar{x}))^2} \le K_1n^{3/2}\|\bar{y}-\bar{x}\| \\
		&\Rightarrow \|\bar{f}(\bar{y}) - \bar{f}(\bar{x})\| \le K_1n^{3/2}\|\bar{y}-\bar{x}\|
	\end{align*}
\end{proof}

\begin{proof}
	(Основная теорема)
	
	Докажем, что $A(\bar{x})$ из системы (\ref{equ:altern}) является сжатием.
	
	Рассмотрим $\Pi = \{\|\bar{x}(t) - \bar{x}_0(t)\| \le b, |t-t_0| \le a   \} \subset \Omega$. Определим $K = \|\bar{f}\|_C = max_{\Pi}|\bar{f}|$. $K_1$ тоже определено в силу условий.
	
	Рассмотрим $\Pi_h = \{\|\bar{x}(t) - \bar{x}_0(t)\| \le b, |t-t_0| \le h \le a   \}$
	
	Банахово пространство $B$ - множество функций $\bar{x}(t)$ непрерывных на отрезке $|t-t_0|\hm\le h$. $M \subset B$ - множество функций $\|\bar{x}(t) - \bar{x}_0\| \le b$. $M$ ограничено, так как $\forall \bar{x}(t) \in M \hookrightarrow \|\bar{x}(t)\| = \|\bar{x}(t) - \bar{x}_0 + \bar{x}_0\| \le b + \|\bar{x}_0\| = C$
	
	Докажем, что $M$ замкнуто. Пусть $\bar{x}_n(t), n = 1, 2, \dots$ - последовательность точек в $M$, такая что $\lim\limits_{n\to \infty} \bar{x}_n(t) = \bar{x}(t)$. $\|\bar{x}(t)\| = \|\bar{x}(t) - \bar{x}_n + \bar{x}_n\| \le \|\bar{x}(t) - \bar{x}_n\| + \|\bar{x}_n\| \le \varepsilon + b$ $\Rightarrow \bar{x}(t) \in M$
	
	Подберем $h$ так, чтобы $A:M\to M$. То есть $\|A(\bar{x}) - \bar{x}_0\| \le b$.
	\[
		\|A(\bar{x}) - \bar{x}_0\| = \|\int\limits_{t_0}^t \bar{f}(\tau, \bar{x}(\tau))d\tau\| \le |\int\limits_{t_0}^t \|\bar{f}\|d\tau| \le Kh
	\]
	Получаем условие $h\le b/K$
	
	Чтобы доказать, что $A$ - сжатие, рассмотрим норму
	\begin{align*}
		&\|A(\bar{y}) - A(\bar{x})\| = \|\int\limits_{t_0}^t(\bar{f}(\tau, \bar{y}(\tau)) - \bar{f}(\tau, \bar{x}(\tau)))d\tau\|\le \\ &\le |\int\limits_{t_0}^t \|\bar{f}(\tau, \bar{y}) - \bar{f}(\tau, \bar{x})\|d\tau| \le K_1n^{3/2}\|\overline{y-x}\|\cdot|\int\limits_{t_0}^td\tau| \le K_1hn^{3/2}\|\overline{y-x}\|
	\end{align*}
	Откуда второе условие: $h < \dfrac{1}{n^{3/2}K_1}$
	
	Тогда оператор $A$ будет сжатием. Соответственно решение задачи Коши существует и единственно.
	
\end{proof}

\subsection{Теорема существования и единственности решения задачи Коши для уравнения $n$-го порядка в нормальном виде}

\begin{definition}
	Уравнение вида
	\begin{equation}
		\label{equ:norm}
		y^{(n)} = f(x, y, \dots, y^{(n-1)})
	\end{equation}
	называется уравнением $n$-го порядка в нормальной форме.
\end{definition}

\begin{definition}
	Система
	\begin{equation}
		\label{equ:init-norm}
		\begin{cases*}
			y(x_0) = y_0\\
			y'(x_0) = y_0'\\
			\dots \\
			y^{(n-1)}(x_0) = y^{(n-1)}_0
		\end{cases*}
	\end{equation}
	называется начальным условием уравнения n-го порядка в нормальной форме.
\end{definition}

\begin{proposition}
	Решить задачу Коши означает найти такое решение (\ref{equ:norm}), которое удовлетворяет условию (\ref{equ:init-norm})
\end{proposition}

\begin{theorem}[Теорема Коши о существовании и единственности решения]
	Если $f, \dfrac{\partial f}{\partial y'}, \dots, \dfrac{\partial f}{\partial y^{(n-1)}}$ непрерывны в $\Omega \subset \mathbb{R}^{n+1}$, тогда $\forall (x_0, \bar{y}_0) \in \Omega \exists h > 0: \forall x\in [x_0 +h, x_0-h]$ решение задачи Коши существует и единственно.
\end{theorem}

\begin{proof}
	Введем следующие функции: $y(x) = v_1(x), y'(x) = v_2(x), \dots, y^{(n-1)}(x) \hm= v_n(x)$. Таким образом получаем систему уравнений в нормальной форме
	\begin{equation}
		\begin{cases*}
			\dfrac{dv_1}{dx} = v_2\\
			\dots\\
			\dfrac{dv_n}{dx} = f(x, \bar{v})
		\end{cases*}
	\end{equation}
	А для нее решение существует и единственно.
\end{proof}
