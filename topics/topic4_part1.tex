\section{Билет 4. Линейные дифференциальные уравнения и линейные системы дифференциальных уравнений с переменными коэффициентами}

\subsection{Теоремы существования и единственности решения задачи Коши для нормальной линейной системы уравнений и
для линейного уравнения $n$-го порядка в нормальном виде}

Рассматривается линейная система вида

\begin{equation}
	\frac{d\overrightarrow{x}}{dt} = A\overrightarrow{x} + \overrightarrow{q}(t),
	\label{Issue5_2}
\end{equation}

где $A = ||a_j^i(t)||$, $i, j = \overrightarrow{1, n}$ -- матрица, $\overrightarrow{q}(t)$ -- заданная вектор-функция. Наряду с векторной записью также будем использовать координатную запись $\dot{x}^i = \sum\limits_{j = 1}^{n} a_j^i x^j + q^i(t),\ i = \overrightarrow{1, n}$.

$\textbf{Необходимым условием линейности}$ является факт того, что все $a_j^i$ и $q^i$ зависят только от $t$ и не зависят от $\overrightarrow{x}$.

Для $(\ref{Issue5_2})$ ставится задача Коши:

\[ \overrightarrow{x}(t_0) = \overrightarrow{x_0}.\]

\begin{theorem}

$\textbf{Основная теорема для линейных систем.}$ Пусть $a_j^i(t),\ i, j = \overrightarrow{1, n}$ и $\overrightarrow{q}(t)$ в $(\ref{Issue5_2})$ непрерывны на отрезке $[a;b]$. Тогда решение задачи Коши существует и единственно на всем отрезке $[a;b].$
\label{4_1_Cauchy}
\end{theorem}

$\textbf{Предварительные замечания:}$

Пусть вектор-функция $\overrightarrow{f}(x) \in B$ и $A$ -- линейный оператор, действующий из $B$ в $B$, т.е. $A(\overrightarrow{f} + \overrightarrow{g}) = A\overrightarrow{f} + A\overrightarrow{g}$.
Норму оператора $A$ в уравнении $\eqref{Issue5_2}$ определим как
\[ ||A|| =  \sup\limits_{\overrightarrow{\varphi} \in B,\ \overrightarrow{\varphi} \neq \overrightarrow{0}} \frac{||A\overrightarrow{\varphi}||}{||\overrightarrow{\varphi}||}. \]

Тогда получаем неравенство: $||A \overrightarrow{\varphi}|| \leqslant ||A|| \cdot ||\overrightarrow{\varphi}||$ (чем воспользуемся в дальнейшем).
Нормой для вектор-функции выберем $||\overrightarrow{x}(t)|| = \max\limits_{1 \leqslant i \leqslant n} (\max\limits_{t \in [a;b]} |x^i(t)|)$.

\begin{proof}

Определим $\overrightarrow{g}(t) = \overrightarrow{x_0} + \int\limits_{t_0}^{t} \overrightarrow{q}(s)ds$ и построим итерационную процедуру.

Так как $q^i(t) \in C_{[a;b]}\ \forall i = \overrightarrow{1, n} \Rightarrow \exists ||\overrightarrow{q}||_c = M_1.$ Тогда $||\overrightarrow{g}||_c = \Big| \Big|\overrightarrow{x_0} + \int\limits_{t_0}^{t}\overrightarrow{q}(s)ds \Big| \Big| \leqslant ||\overrightarrow{x_0}|| + \Big| \Big| \int\limits_{t_0}^{t}\overrightarrow{q}(s)ds \Big| \Big| \leqslant ||\overrightarrow{x_0}|| + M_1(b-a) = C$. Обозначим $||A|| = C_1$.

Рассмотрим интегральное уравнение $\overrightarrow{x} = \overrightarrow{g} + \int\limits_{t_0}^{t}A(s)\overrightarrow{x}(s)ds$.

Аналогично основной лемме доказывается, что последнее интегральное уравнение эквивалентно задаче $(\ref{Issue5_2})$.

Итерационная процедура: $\overrightarrow{x_0} = \overrightarrow{g}$, $\overrightarrow{x_k} = \overrightarrow{g} + \int\limits_{t_0}^{t} A(s)\overrightarrow{x_{k-1}}(s)ds $, $k = 0, 1, \dots$

Оценим норму:

$$ ||\overrightarrow{x_1} - \overrightarrow{x_0}|| = \Big| \Big| \int\limits_{t_0}^{t} A(s)\overrightarrow{g}(s)ds \Big| \Big| \leqslant \Big| \int\limits_{t_0}^{t} || A(s)\overrightarrow{g}(s) || ds \Big| \leqslant $$
$$\leqslant \Big| \int\limits_{t_0}^{t} || A(s) || \cdot || \overrightarrow{g}(s) || ds \Big| \leqslant C_1 C |t - t_0|;$$ 

Таким образом $||\overrightarrow{x_1} -\overrightarrow{x_0}|| \leqslant CC_1|t-t_0|.$

Теперь докажем по индукции неравенство: $||\overrightarrow{x_k} - \overrightarrow{x_{k-1}}|| \leqslant \frac{CC_1^k}{k!}|t-t_0|^k.$

Базой индукции выступает полученное выше неравенство. Предположим, что верно для $n = k$: $||\overrightarrow{x_k} - \overrightarrow{x_{k-1}}|| \leqslant \frac{CC_1^k}{k!}|t-t_0|^k$. Докажем для $n = k + 1$:

\[ || \overrightarrow{x_{k+1}} - \overrightarrow{x_{k}} || = \Big| \Big| \int\limits_{t_0}^{t} A(s)(\overrightarrow{x_k}(s) - \overrightarrow{x_{k-1}}(s))ds \Big| \Big| \leqslant \Big| \int\limits_{t_0}^{t} ||A(s)(\overrightarrow{x_k}(s) - \overrightarrow{x_{k-1}}(s))|| ds \Big| \leqslant \]

\[ \leqslant \Big| \int\limits_{t_0}^{t} ||A(s)||\cdot ||(\overrightarrow{x_k}(s) - \overrightarrow{x_{k-1}}(s))|| ds \Big| \leqslant C_1 \Big| \int\limits_{t_0}^{t} \frac{CC_1^k|s-t_0|^k}{k!}ds \Big| = \frac{CC_1^{k+1}|t-t_0|^{k+1}}{(k+1)!} \]

Так как $|t-t_0| \leqslant (b-a)$, то предыдущее неравенство можно усилить $||\overrightarrow{x_k} - \overrightarrow{x_{k-1}}|| \leqslant \frac{CC_1^k}{k!}(b-a)^k.$

Функциональная последовательность $\overrightarrow{x_k}$ сходится равномерно, так как сходится равномерно ряд $\overrightarrow{x_0} + (\overrightarrow{x_1} - \overrightarrow{x_0}) +\ \dots\ +(\overrightarrow{x_k} - \overrightarrow{x_{k-1}}) +\ \dots$, который мажорируется сходящимся рядом $||\overrightarrow{x_0}|| + ||(\overrightarrow{x_1} - \overrightarrow{x_0})|| +\ \dots\ + ||(\overrightarrow{x_k} - \overrightarrow{x_{k-1}})|| +\ \dots \leqslant ||\overrightarrow{x_0}|| + C_1\sum\limits_{k = 0}^{\infty}\frac{C^k|b-a|^k}{k!} = ||\overrightarrow{x_0}|| + C_1 e^{C(b-a)} < \infty \Rightarrow$ Существует (в силу банаховости пространства) непрерывно дифференцируемая функция $\overrightarrow{\varphi(t)}$ на $[a;b]$: $\exists \lim\limits_{n \rightarrow \infty} \overrightarrow{x_n} = \overrightarrow{\varphi(t)}.$

Рассмотрим $\Big| \Big| \int\limits_{t_0}^{t} A\overrightarrow{x_n}ds - \int\limits_{t_0}^{t} A\overrightarrow{\varphi} ds \Big| \Big| = \Big| \Big| \int\limits_{t_0}^{t} A(\overrightarrow{x_n} - \overrightarrow{\varphi})ds \Big| \Big| \leqslant ||A||\cdot \Big| \int\limits_{t_0}^{t} ||\overrightarrow{x_n} - \overrightarrow{\varphi}|| ds \Big| \leqslant ||A|| \cdot ||\overrightarrow{x_n} - \overrightarrow{\varphi}|| (t - t_0)$, где $||\overrightarrow{x_n} - \overrightarrow{\varphi} || \rightarrow 0$ при $n \rightarrow \infty$.

Таким образом, итерационная процедура сходится в силу существования пределов слева и справа.

Полученное решение эквивалентно решению задачи $(\ref{Issue5_2})$. В отличии от основной теоремы для нормальных систем ДУ: $\dot{\overrightarrow{x}} = \overrightarrow{f}(t, \overrightarrow{x})$, где существование было получено только на отрезке Пеано, для СЛДУ существование решения доказано для всего отрезка $[a;b]$ -- промежутка, где $a_j^i(t)$ и $\overrightarrow{q}(t)$ непрерывны. В нашем случае $\overrightarrow{f}$ соответствует $\overrightarrow{f} = A\overrightarrow{x} + \overrightarrow{q}$. Она непрерывна, так как полученное решение $\overrightarrow{x}(t)$ непрерывно. Условие непрерывности $\frac{\partial f}{\partial x_i}$ также выполнены, так как в нашем случае $\frac{\partial f}{\partial x_i} = a_{ij}(t)$ -- непрерывна на $[a;b]$. Отсюда следует единственность, так как два решения задачи $(\ref{Issue5_2})$, согласно основной теореме для нормальной системы, совпадает на промежутке, где они оба определены. В нашем случае это $[a;b]$.

Таким образом, теорема не носит локальных характер.

\end{proof}
