\section{Билет 1. Основные понятия, простейшие типы дифференциальных уравнений}
\subsection{Основные понятия}

\begin{definition} 
    Уравнение вида \[F(x, y(x), y'(x), y''(x), \dots, y^{(n)}(x)) = 0\] называется обыкновенным дифференциальным уравнением,
    где $x$ -- аргумент, $y(x)$ -- неизвестная функция, $F$ -- известная непрерывная функция в области $D$.
\end{definition}

\begin{definition}
    Если это уравнение удается разрешить относительно старшей производной, такое дифференциальное
    уравнение называется разрешённым относительно старшей производной и записывается в виде
    \[y^{(n)}(x) = f(x, y(x), y'(x), y''(x), \dots, y^{(n - 1)}(x))\]
\end{definition}

Порядок уравнения определяется порядком старшей производной от $y$.

\begin{definition}
    Функция $y = \varphi(x)$ называется решением ДУ, если она $n$ раз дифференцируема и
    \[\forall x \in G \rightarrow F(x, \varphi(x), \varphi'(x), \dots, \varphi^{(n)}(x)) \equiv 0, \; (x, \varphi(x), \dots, \varphi^{(n)}(x)) \in D,\]
    где $G$ -- область определения функции $\varphi(x)$ с её производными.
\end{definition}

\begin{definition}
    Система $n$ уравнений
    \begin{equation}
        \begin{cases}
            \dot x^1 = f_1(t, x^1(t), \dots, x^n(t)) \\
            \dots \\
            \dot x^n = f_n(t, x^1(t), \dots, x^n(t)) \\
        \end{cases}
    \end{equation}
    где $x^1(t), \dots, x^n(t)$ -- искомые функции, а $f_i$ -- некоторые непрерывные функции, называется нормальной системой ДУ $n$-го порядка.
\end{definition}

\begin{proposition}
    Рассмотрим ДУ $y^{(n)}(x) = f(x, y(x), y'(x), y''(x), \dots, y^{(n-1)}(x))$ $n$-ого порядка. Это уравнение эквивалентно следующей нормальной системе ДУ:
    \begin{equation}
        \begin{cases}
            \dot v_1 = v_2 \\
            \dot v_2 = v_3 \\
            \dots \\
            \dot v_{n-1} = v_n \\
            \dot v_n = f_n(x, v_1, v_2, \dots, v_n) \\
        \end{cases}
        \Leftrightarrow y^{(n)}(x) = f(x, y'(x), y''(x), \dots, y^{(n-1)}(x))
    \end{equation}
\end{proposition}

\begin{proof}
    Введём обозначения: $y = v_1(x)$, $y' = v_2(x)$,
    $y'' = v_3(x)$, $\dots$, $y^{(n - 1)} = v_n(x)$. Тогда имеем $\dot v_1 = v_2, \; \dot v_2 = v_3, \; \dots, \dot v_n = f(x, v_1, v_2, \dots, v_n)$, то есть получилась нормальная система дифференциальных уравнений $n$-ого порядка с неизвестными $v_i$.

    Обратными заменами системы уравнений можно получить исходное дифференциальное уравнение $y^{(n)}(x) = f(x, y(x), y'(x), y''(x), \dots, y^{(n-1)}(x))$.
\end{proof}

\begin{definition}
    Рассмотрим уравнение $1$-ого порядка $y' = f(x, y(x))$. Тогда задача решить это уравнение с условием $y(x_0) = y_0$ называется задачей Коши.
\end{definition}

\begin{definition}
    Пусть $\varphi(x)$ -- решение дифференциального уравнения $y' = f(x, y(x))$. График решения $\varphi(x)$ называется интегральной кривой. В силу определения функции $f(x, y)$ на множестве $\Omega$, вся интегральная кривая будет лежать в $\Omega$.
\end{definition}

\begin{definition}
    Проведём через каждую точку интегральной кривой $(x_0, y_0) \in \Omega$ малый отрезок с углом наклона по отношению к оси $x$ равным $\alpha$, причём $\tg \alpha = f(x_0, y_0)$. Получим так называемое поле направлений. 
\end{definition}

Из построения интегральной кривой следует, что интегральная кривая в каждой своей точке касается поля направлений. Верно и обратное: кривая, касающаяся в каждой своей точке поля направлений, является интегральной кривой.

\subsection{Простейшие типы уравнений первого порядка}
\subsubsection{Уравнения в полных дифференциалах}

Рассмотрим следующее дифференциальное уравнение: $P(x, y)dx + Q(x, y)dy = 0$, причём функции $P(x, y)$ и $Q(x, y)$ непрерывны в некоторой области $D$ и $\forall (x_0, y_0) \in D \rightarrow |P(x_0, y_0)| + |Q(x_0, y_0)| > 0$. Тогда кривая 
\begin{equation}
    \gamma = 
    \begin{cases}
        x = \varphi(t) \\ 
        y = \psi(t)
    \end{cases}, \; t_1 \leqslant t \leqslant t_2
\end{equation}
называется интегральной кривой рассматриваемого уравнения, если $\forall t: t \in [t_1; t_2]$ функции $\varphi(t)$ и $\psi(t)$ непрерывно дифференцируемы, $(\varphi(t), \psi(t)) \in D$, $(\varphi_t')^2 + (\psi_t')^2 > 0$ и выполнено

\begin{equation}
    P(\varphi(t), \psi(t)) \varphi_t' + Q(\varphi(t), \psi(t)) \psi_t' = 0.
\end{equation}

\begin{definition}
    Дифференциальное уравнение $P(x, y)dx + Q(x, y)dy = 0$ называется уравнением в полных дифференциалах, если $\exists F(x, y): P(x, y)dx + Q(x, y)dy = dF(x, y)$. 
\end{definition}

Тогда $dF(x, y) = 0 \Rightarrow F(x, y) = const$, то есть $F(x, y)$ определяет неявную функцию $y(x)$.

\begin{theorem}
    Пусть функции $P(x, y)$ и $Q(x, y)$ непрерывно дифференцируемы в области $D$. Для того, чтобы уравнение $P(x, y)dx + Q(x, y)dy = 0$ являлось уравнением в полных дифференциалах, необходимо выполнение условия $\frac{\partial P}{\partial y} = \frac{\partial Q}{\partial x}$, $\forall (x, y) \in D$. Если же область $D$ ещё и односвязна, то условие $\frac{\partial P}{\partial y} = \frac{\partial Q}{\partial x}$ является достаточным.
\end{theorem}

\begin{proof}
    Пусть $P(x, y)dx + Q(x, y)dy = 0$ -- уравнение в полных дифференциалах, тогда $\exists F(x, y): P(x, y)dx + Q(x, y)dy = dF(x, y) \Rightarrow P = \frac{\partial F}{\partial x}$, $Q = \frac{\partial F}{\partial y}$. По условию $P$ и $Q$ -- непрерывно дифференцируемы, тогда $\frac{\partial P}{\partial y}$ и $\frac{\partial Q}{\partial x}$ -- непрерывные функции, значит 
    \begin{equation}
        \frac{\partial P}{\partial y} = \frac{\partial^2 F}{\partial y \partial x} = \frac{\partial^2 F}{\partial x \partial y} = \frac{\partial Q}{\partial x}, \; \forall (x, y) \in D.
    \end{equation}

    Пусть теперь $D$ -- односвязная область. Рассмотрим значение интеграла
    \[ F = \int\limits^{(x; y)}_{(x_0, y_0)} P(x, y) dx + Q(x, y) dy, \]
    который берётся по кусочно гладкой кривой $\gamma$, лежащей в $D$ и соединяющей точки $(x_0, y_0)$ и $(x; y)$. Пусть $\frac{\partial P}{\partial y} = \frac{\partial Q}{\partial x}$. Тогда по теореме о независимости интеграла от пути интегрирования выходит, что значение интеграла не зависит от пути интегрирования $\gamma$, а является функцией от $(x, y)$, значит $F = F(x, y)$ -- функция и $P(x, y)dx + Q(x, y)dy = dF(x, y)$.
\end{proof}

\begin{definition}
    Непрерывно дифференцируемая функция $\mu(x, y) \neq 0$ в области $G$ называется интегрирующим множителем для уравнения $P(x, y) dx + Q(x, y) dy = 0$, если уравнение $\mu(x, y) (P(x, y) dx + Q(x, y) dy) = 0$ -- уравнение в полных дифференциалах, а исходное уравнение $P(x, y) dx + Q(x, y) dy = 0$ не является уравнением в полных дифференциалах.
\end{definition}

Если $\mu(x, y)$ -- интегрирующий множитель, то для достаточного условия имеем (с учётом требований теоремы выше)
\[ \frac{\partial (\mu P)}{\partial y} = \frac{\partial (\mu Q)}{\partial x} \Leftrightarrow P \frac{\partial \mu}{\partial y} + \mu \frac{\partial P}{\partial y} = Q \frac{\partial \mu}{\partial x} + \mu \frac{\partial Q}{\partial x}. \]

Полученное уравнение не легче исходного, так как теперь задача свелась к нахождению $\mu$. Обычно интегрирующий множитель ищут в виде $\mu(x), \; \mu(y), \; \mu(x^2 + y^2), \; \mu (x^{\alpha}, y^ {\beta})$.

\subsubsection{Уравнения с разделяющимися переменными}

Рассмотрим ДУ вида $P(y)dx + Q(x)dy = 0$, где $P(y) \in C_{[y_1; y_2]}^1$, $Q(x) \in C_{[x_1; x_2]}^1$. Если $\exists y_0: P(y_0) = 0$ или  $\exists x_0: Q(x_0) = 0$, тогда
\begin{equation}
    \begin{cases}
        x = t \\ 
        y = y_0
    \end{cases} \;
    \text{или} \;
    \begin{cases}
        x = x_0 \\ 
        y = t
    \end{cases}
\end{equation}
являются интегральными кривыми рассматриваемого ДУ соответственно. Если же выполняется $P(y) \neq 0$ и $Q(x) \neq 0$, то применим к уравнению интегрирующий множитель
\[ \mu(x, y) = \frac{1}{P(y)Q(x)}, \]
получив уравнение в полных дифференциалах
\begin{equation}
    \frac{dx}{Q(x)} + \frac{dy}{P(y)} = 0.
\end{equation}

Значение $\mu(x, y)$ действительно является интегрирующим множителем, так как выполняется
\begin{equation}
    \frac{\partial}{\partial y} \left( \frac{1}{Q(x)} \right) =  \frac{\partial}{\partial x} \left( \frac{1}{P(y)} \right) = 0.
\end{equation}

Тогда
\begin{equation}
    dF(x, y) = \frac{dx}{Q(x)} + \frac{dy}{P(y)} \Rightarrow \frac{\partial F}{\partial x} = \frac{1}{Q(x)} \Rightarrow F(x, y) = \int\limits_{x_0}^{x} \frac{dt}{Q(t)} + C(y),
\end{equation}

\begin{equation}
    \frac{\partial F}{\partial y} = \frac{1}{P(y)} = C'(y) \Rightarrow C(y) = \int\limits_{y_0}^{y} \frac{dt}{P(t)} \Rightarrow F(x, y) = \int\limits_{x_0}^{x} \frac{dt}{Q(t)} + \int\limits_{y_0}^{y} \frac{dt}{P(t)} = const.
\end{equation}

Точка $(x_0, y_0)$ -- произвольная точка в области определения функций $P$ и $Q$.

\begin{definition}
    Если дифференциальное уравнение вида $P_1(x, y)dx + Q_1(x, y)dy = 0$ может быть сведено к виду $P(y)dx + Q(x)dy = 0$, то такое уравнение называется уравнением с разделяющимися переменными.
\end{definition}

\begin{proposition}
    Задача Коши уравнения с разделяющимися переменными $P(y)dx + Q(x)dy = 0$ задаётся в виде $y(x_1) = y_1$, а её решение в виде 
    \begin{equation}
        \int\limits_{x_1}^{x} \frac{dt}{Q(t)} + \int\limits_{y_1}^{y} \frac{dt}{P(t)} = 0.
    \end{equation}
\end{proposition}

\subsubsection{Однородные уравнения}

Рассмотрим дифференциальное уравнение вида
\[ y' = g \left( \frac{y}{x} \right), \]
которое назовём уравнением с однородной правой частью, где $g(z)$ -- непрерывная функция на некотором промежутке. Сделаем замену $v(x) = \frac{y}{x}$, тогда $y(x) = v(x) \cdot x$, $y_x' = x \cdot v_x' + v = g(v)$, откуда имеем $x \frac{dv}{dx} = g(v) - v$. Если $\exists v_0: g(v_0) = v_0$, то $v_0$ -- решение уравнения $x \frac{dv}{dx} = g(v) - v$. Если же $v \neq g(v)$, тогда
\begin{equation}
    \frac{dv}{g(v) - v} = \frac{dx}{x} \Rightarrow \ln |x| + C = \int\limits_{v_0}^{v} \frac{dt}{g(t) - t}.
\end{equation}

Таким образом, найдено решение исходного уравнения с однородной правой частью в квадратурах.

\begin{definition}
    Функция $F(x^1, x^2, \dots, x^n)$ называется однородной степени $m$, если $\forall \lambda > 0 \longrightarrow F(\lambda x^1, \lambda x^2, \dots,  \lambda x^n) = \lambda^m F(x^1, x^2, \dots, x^n)$.
\end{definition}

\begin{example}
    Рассмотрим уравнение $P(x, y) dx = Q(x, y) dy$. Если $P(x, y)$ и $Q(x, y)$ -- однородные функции степени $m$, тогда
    \begin{equation}
        \frac{dy}{dx} = \frac{P(x, y)}{Q(x, y)} = \frac{x^m P(1, \frac{y}{x})}{x^m Q(1, \frac{y}{x})} = \frac{P(1, \frac{y}{x})}{Q(1, \frac{y}{x})} = g \left( \frac{y}{x} \right)
    \end{equation}
    Таким образом исходное уравнение свелось к уравнению с однородной правой частью.
\end{example}

\subsubsection{Линейные уравнения первого порядка}

\begin{definition}
    Дифференциальное уравнение вида $y' + a(x) y = f(x)$ -- линейное дифференциальное уравнение первого порядка. Дифференциальное уравнение вида $y' + a(x) y = 0$ -- линейное однородное дифференциальное уравнение первого порядка. При этом $a(x) \in C_{I(x)}$, $f(x) \in C_{I(x)}$, где $I(x)$ -- область, на которой определены функции $a(x)$ и $f(x)$. 
\end{definition}

Введём оператор $L = \frac{d}{dx} + a(x)$, который действует на множество непрерывно дифференцируемых функций $\varphi \in C^1_{I(x)}$. Тогда уравнение $y' + a(x) y = f(x)$ переписывается в виде $L(y) = f(x)$, а уравнение $y' + a(x) y = 0$ переписывается в виде $L(y) = 0$.

\begin{theorem}
    Введённый оператор $L = \frac{d}{dx} + a(x)$ -- линейный оператор.
\end{theorem}

\begin{proof}
    Рассмотрим линейную комбинацию $c_1 \varphi_1(x) + c_2 \varphi_2(x)$:
    \begin{equation}
        L(c_1 \varphi_1(x) + c_2 \varphi_2(x)) = (c_1 \varphi_1 + c_2 \varphi_2)' + a(x) (c_1 \varphi_1 + c_2 \varphi_2) = c_1 L(\varphi_1) + c_2 L(\varphi_2)
    \end{equation}
    Таким образом, $L(c_1 \varphi_1 + c_2 \varphi_2) = c_1 L(\varphi_1) + c_2 L(\varphi_2)$, то есть $L$ -- линейный оператор.
\end{proof}

\begin{proposition}
    Решением уравнения $y' + a(x) y = 0$ является
    \begin{equation}
        y = C e^{-\int\limits_{x_0}^{x} a(t) dt}, \; C \in \mathbb{R}.
    \end{equation}
\end{proposition}

\begin{proof}
    Найдём решение уравнения $y' + a(x) y = 0$: 
    \begin{equation}
        \frac{dy}{y} = -a(x) dx \Rightarrow \ln |y| = - \int\limits_{x_0}^{x} a(t) dt + \ln C \Rightarrow |y| = C e^{-\int\limits_{x_0}^{x} a(t) dt}, \; C > 0
    \end{equation}
    Раскрывая модуль и объединяя полученное решение с нулевым ($y \equiv 0$), имеем
    \begin{equation}
        y = C e^{-\int\limits_{x_0}^{x} a(t) dt}, \; C \in \mathbb{R}.
    \end{equation}
\end{proof}

\begin{proposition}
    Решением уравнения $y' + a(x) y = f(x)$ является
    \begin{equation}
        y = C_0 e^{-\int\limits_{x_0}^{x} a(t) dt} + \int\limits_{x_0}^{x} f(t) e^{- \int\limits_{t}^{x} a(s) ds} dt, \; C_0 \in \mathbb{R}.
    \end{equation}
\end{proposition}

\begin{proof}
    Найдём решение уравнения $y' + a(x) y = f(x)$: воспользуемся уже найденным решением однородного уравнения, применяя метод вариации постоянной. То есть будем искать решение в виде
    \begin{equation}
        y = C(x) e^{-\int\limits_{x_0}^{x} a(t) dt}.
    \end{equation}
    Подставим это решение в исходное уравнение:
    \begin{equation}
        C'(x) e^{-\int\limits_{x_0}^{x} a(t) dt} - a(x) C(x) e^{-\int\limits_{x_0}^{x} a(t) dt} + a(x) C(x) e^{-\int\limits_{x_0}^{x} a(t) dt} = f(x)
    \end{equation}
    \begin{equation}
        C'(x) e^{-\int\limits_{x_0}^{x} a(t) dt} = f(x) \Rightarrow C(x) = \int\limits_{x_0}^{x} f(t) e^{\; \int\limits_{x_0}^{t} a(s) ds} dt + C_0 
    \end{equation}
    Таким образом найден вид $C(x)$. Теперь подставим эту функцию:
    \begin{equation}
        y = C_0 e^{-\int\limits_{x_0}^{x} a(t) dt} + e^{-\int\limits_{x_0}^{x} a(t) dt} \int\limits_{x_0}^{x} f(t) e^{\; \int\limits_{x_0}^{t} a(s) ds} dt
    \end{equation}
    \begin{equation}
        y = C_0 e^{-\int\limits_{x_0}^{x} a(t) dt} + \int\limits_{x_0}^{x} f(t) e^{- \int\limits_{t}^{x} a(s) ds} dt
    \end{equation}
    
    Из полученного решения видно, что оно является суммой решения однородного уравнения и частного решения. 
\end{proof}

\begin{proposition}
    Если $\varphi_1(x)$ и $\varphi_2(x)$ -- некоторые решения уравнения $y' + a(x) y = f(x)$, то $z(x) = \varphi_1(x) - \varphi_2(x)$ -- решение однородного уравнения $y' + a(x) y = 0$.
\end{proposition}

\begin{proof}
    По условию $\varphi_1' + a(x) \varphi_1 = f(x)$, $\varphi_2' + a(x) \varphi_2 = f(x)$, откуда очевидно, что $(\varphi_1 - \varphi_2)' + a (\varphi_1 - \varphi_2) = 0$. Обозначив $z = \varphi_1 - \varphi_2$, получим $z' + a(x) z = 0$, то есть $z$ -- решение однородного уравнения.
\end{proof}
