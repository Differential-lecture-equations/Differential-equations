
\subsection{Функционалы, зависящие от высших производных}
Рассмотрим функционал 
\begin{equation} 
	\label{Issue15:2.3}
	\mathcal{J}(y) ~ = \int_a^bF(x,y(x),y'(x),\dots , y^{(n)}(x))d x
\end{equation}
с условием 
\begin{equation}
	\label{Issue15:2.4}
	y(a)=A_0,\dots ,y^{(n-1)}(a)=A_{n-1};\;\;y(b)=B_0,\dots , y^{(n-1)}(b)=B_{n-1}
\end{equation}

Будем считать, что $F(x,z_0,\dots , z_n)$ $n$ раз дифференцируема по совокупности всех переменных на $a\leqslant x\leqslant b$; $-\infty < z_0, \dots , z_n< \infty $. Пусть $y(x)\in C^n_{[a;b]}$. Норму на этом множестве функций определим как $$||y||= ~ \sum_{k=0}^{n}\underset{x \in [a;b]}{\max}|y^{(k)}(x)|$$
Пусть $y_0(x)$ является слабым минимумом $\eqref{Issue15:2.3}\wedge \eqref{Issue15:2.4}$.

Множество допустимых вариаций: $H_\delta(y_0)=\{ \delta y(x) \in C_{[a;b]}^n: \delta y^{(i)}(a)=\delta y^{i}(b)=0,\;\;i=\overline{1,n-1}\}\Rightarrow \mathcal{D}=\{y_0(x)+t\delta y(x):\delta y(x)\in H_\delta (y_0) \}$ (доказательство аналогично).

\[
    \mathcal{J}(y_0(x)+t\delta y(x)) = \int^b_a F(x, y_0(x)+t\delta y(x),\dots ,y_0^{(n)}(x)+t(\delta y(x))^{(n)} )dx =\mathcal{J}(t)=\mathcal{J}(0)+\delta\mathcal{J} \cdot t+o(t), 
\]

\[
    \delta \mathcal{J}= ~ \int\limits_a^b\left(\frac{\partial F}{\partial y}\delta y + \frac{\partial F}{\partial y'}\delta y' +\dots +  \frac{\partial F}{\partial y^{(n)}}\delta y^{(n)} \right) dx,
\]
 где $\cfrac{\partial F}{\partial y^{(i)}}= \cfrac{\partial F(x,y_0(x),\dots,y_0^{(n)}(x))}{\partial y^{(i)}}, \, i = \overline{0, n}$.

Аналогично доказывается, что если $y_0(x)$ -- слабый минимум $\eqref{Issue15:2.3}\wedge\eqref{Issue15:2.4}$, то $\forall\delta y(x)\in H_\delta (y_0) \mapsto \delta \mathcal{J} = 0$.

\begin{proof}
	Если $\delta y(x)\in H_\delta(y_0),$ то $\forall k=\overline{1,n}:$
	\begin{multline*}
		\int\limits_a^b\frac{\partial F}{\partial y^{(k)}}(\delta y)^{(k)}dx = \underbrace{\frac{\partial F}{\partial y^{(k)}}(b)(\delta y(b))^{(k-1)}}_{=0}-\underbrace{\frac{\partial F}{\partial y^{(k)}}(a)(\delta y(a))^{(k-1)}}_{=0}-\int\limits_a^b\frac{d}{dx}\frac{\partial F}{\partial y^{(k)}}(\delta y)^{(k-1)}dx= \\ 
		=[\text{аналогично}]=\int\limits_a^b \frac{d^2}{dx^2}\frac{\partial F}{\partial y^{(k)}}(\delta y)^{(k-2)}dx=\dots=\int\limits^b_a\left(-1\right)^k\frac{d^k}{dx^k}\frac{\partial F}{\partial y^{(k)}}\delta y dx.
	\end{multline*}
	Тогда, если $y_0(x)$ слабый минимум $\eqref{Issue15:2.3}\wedge\eqref{Issue15:2.4}$, то имеем: 
	\begin{multline}
		\label{Issue15:2.5}
		\delta\mathcal{J} = \int\limits_a^b\left(\sum_{k=0}^n(-1)^k\frac{d^k}{dx^k}\frac{\partial F}{\partial y^{(k)}}\delta y\right) dx = \\
		= \int\limits_a^b\left(\frac{\partial F}{\partial y}-\frac{d}{dx}\frac{\partial F}{\partial y'}+\frac{d^2}{dx^2}\frac{\partial F}{\partial y''}+\dots+ (-1)^k\frac{d^k}{dx^k}\frac{\partial F}{\partial y^{(k)}}\right)\delta y dx=0
	\end{multline}
\end{proof}

Тогда, если $y_0(x)\in C^{n+1}_{[a;b]}$ -- слабый экстремум $\eqref{Issue15:2.3}\wedge\eqref{Issue15:2.4}$, то из $\eqref{Issue15:2.5}$ и из основной леммы следует, что $y_0(x)$ удовлетворяет \textbf{уравнению Эйлера}:

\begin{equation}
	\label{Issue15:2.6}
	\frac{\partial F}{\partial y} - \frac{d}{dx}\frac{\partial F}{\partial y'}+\frac{d^2}{dx^2}\frac{\partial F}{\partial y''}+\dots+(-1)^n\frac{d^n}{dx^n}\frac{\partial F}{\partial y^{(n)}}=0
\end{equation}

\subsection{Условные вариационные принципы. Изопериметрическая задача.}
Среди функций $y(x)\in C^1_{[a;b]}$ найти такую, что $y(a)=A,\;\; y(b)=B$, которая дает экстремум функционалу 
\begin{equation}
	\mathcal{J}(y)=\int\limits_a^b F(x,y(x),y'(x))dx\;\;
	\label{Issue15:3.1}
\end{equation} 
и на которой функционал $\mathcal{G}(y) = \int\limits_a^b g(x,y(x),y'(x))dx$ принимает заданное значение $l$:
\begin{equation}
	\label{Issue15:3.2}
	\mathcal{G}(y)=\int\limits_a^b g(x,y(x),y'(x))d x=l
\end{equation}
Пусть $F$ и $g$ дважды непрерывно дифференцируемы по совокупности переменных. $D=\{y(x)\in C^1_{[a;b]}:y(a)=A, \, y(b)=B, G(y(x))=l\}$, 
$H_\delta (y_0)=\{\delta y \in C^1_{[a;b]}:\delta y (a) = \delta  y(b)=0\}$
\begin{theorem}
	Пусть $y_0(x)\in C^2_{[a;b]}$ и является слабым экстремумом $\eqref{Issue15:3.1}$ на множестве $D$ и $\exists \delta y_0\in H_\delta (y_0(x)): \delta \mathcal{G}(y_0,\delta y_0)\neq 0.$
	
	Положим $\Phi =\mathcal{J}+\lambda\mathcal{G}.$ Тогда $\exists \lambda \in \mathds{R}: \delta \Phi(y_0,\delta y)=0\;\;\;\forall \delta y\in H_\delta(y_0)$
\end{theorem}
\begin{proof}
	Пусть $y_0(x)\in D$ является слабым экстремумом $\eqref{Issue15:3.1}$ на $D$ и по условию теоремы $\exists \delta y_0\in H_\delta(y_0): \delta\mathcal{G}(y_0,\delta y_0)\neq 0$. Рассмотрим числа $t_1, t_2$ и $y(x)=y_0(x)+t_1\delta y_0 + t_2 \delta y \in D,$ где $\delta y_0$ зафиксировано. При фиксировании $\delta y: \mathcal{J}(y_0(x)+t_1\delta y_0+t_2 \delta y)=\mathcal{J}(t_1,t_2)$ и \begin{equation}
		\label{Issue15:3.3}
		\mathcal{G}(y_0(x)+t_1\delta y_0+t_2 \delta y)=\mathcal{G}(t_1,t_2)=l
	\end{equation}
	По условию $y_0(x)$ -- экстремум $\mathcal{J}(y(x))\Rightarrow$ при $t_1=t_2=0\;\;\;\mathcal{J}(t_1,t_2)$ имеет экстремум при условии $\eqref{Issue15:3.2}$.

Из $\eqref{Issue15:3.2}$ и условия теоремы: 
$$\delta \mathcal{G}(y_0,\delta y_0)=\frac{\partial \mathcal{G}}{\partial t_1} \bigg|_{t_1=t_2=0} = \frac{\partial \mathcal{G}}{\partial\overline{t_1}} = \int\limits_a^b \left( \frac{\partial g}{\partial y}\delta y_0+\frac{\partial g}{\partial y'}(\delta y_0)' \right) dx \neq 0$$

\begin{equation}
	\frac{\partial \mathcal{G}}{\partial t_2} \bigg|_{t_1 = t_2 = 0} = \frac{\partial \mathcal{G}}{\partial\overline{t_2}} = \int\limits_a^b \left( \frac{\partial g}{\partial y}\delta y+\frac{\partial g}{\partial y'}(\delta y)' \right) dx
	\label{Issue15:3.4}
\end{equation}

Так как в $\eqref{Issue15:3.4}$ $\cfrac{\partial \mathcal{G}}{\partial \overline{t_1}}\neq 0$, то по теореме о неявно заданной функции можно сказать, что $\eqref{Issue15:3.3}$ определяет неявную функцию $t_1=t_1(t_2)$. По теореме о неявной функции эта функция непрерывно дифференцируема в окрестности точки $(0;0)$ и 

\begin{equation}
    \cfrac{dt_1}{dt_2}\bigg|_{t_2 = 0} = -\cfrac{\partial \mathcal{G}/\partial \overline{t_2}}{\partial \mathcal{G}/\partial \overline{t_1}}
    \label{Issue15:3.5}
\end{equation}
Функция $\mathcal{J}(t_1(t_2);t_2)=\overline{\mathcal{J}}(t_2)$ при $t_2=0$ имеет экстремум по условию. Тогда 
\begin{equation}
    \frac{d\mathcal{J}(t_1(t_2);t_2)}{dt_2} \bigg|_{t_2=0}=\frac{\partial \mathcal{J}}{\partial \overline{t_2}}+ \frac{\partial \mathcal{J}}{\partial \overline{t_1}}\cdot \frac{d t_1}{d t_2}\bigg|_{t_2=0}=0
    \label{Issue15:3.6}
\end{equation}
В силу $\eqref{Issue15:3.5}$ продолжим $\eqref{Issue15:3.6}$: 

\[ 
    \frac{d\mathcal{J}}{d t_2} \bigg|_{t_2=0} = \frac{\partial \mathcal{J}}{\partial \overline{t_2}} - \frac{\partial \mathcal{J}}{\partial \overline{t_1}}\cdot \frac{ \partial \mathcal{G}/\partial \overline{t_2}}{\partial \mathcal{G}/\partial \overline{t_1}}=0 
\]

Обозначим через $\lambda = -\cfrac{\partial\mathcal{J}/\partial \overline{t_1}}{\partial\mathcal{G} / \partial \overline{t_1}}$
\begin{equation}
\label{Issue15:3.8}
\frac{\partial\mathcal{J}}{\partial\overline{t_2}}+\lambda \frac{\partial\mathcal{G}}{\partial\overline{t_2}}=0, ~\forall\delta y \in H_\delta(y_0) \end{equation}
\par
В $\eqref{Issue15:3.6}:$
\begin{equation}
    \label{Issue15:3.7}
    \frac{\partial\mathcal{J}}{\partial \overline{t_1}}=\frac{\partial \mathcal{J}}{\partial t_1}\bigg|_{t_1=t_2=0}=\int\limits_a^b \left( \frac{\partial F}{\partial y}\delta y_0 + \frac{\partial F}{\partial y'}(\delta y_0)' \right) dx,
    \end{equation}
    \begin{equation*}
    \frac{\partial\mathcal{J}}{\partial \overline{t_2}}=\frac{\partial \mathcal{J}}{\partial t_2}\bigg|_{t_1=t_2=0}=\int\limits_a^b \left( \frac{\partial F}{\partial y}\delta y+\frac{\partial F}{\partial y'}(\delta y)' \right) dx
\end{equation*}
Введем $\Phi = \mathcal{J}+\lambda\mathcal{G}\Rightarrow \delta \Phi\bigg|_{t_2=0}=\cfrac{d\mathcal{J}}{d\overline{t_2}}+\lambda\cfrac{d\mathcal{G}}{d\overline{t_2}}=0 ~\eqref{Issue15:3.8}$, тогда в силу $\eqref{Issue15:3.4}$ и $\eqref{Issue15:3.7}$ $$\int\limits_a^b\left(\frac{\partial F}{\partial y}\delta y + \frac{\partial F}{\partial y'}(\delta y)'\right)d x+\lambda\int\limits_a^b\left(\frac{\partial g}{\partial y}\delta y + \frac{\partial g}{\partial y'}(\delta y)'\right)d x=0\Rightarrow$$$$ 
\delta \Phi =\int\limits_a^b\left[\left(\frac{\partial F}{\partial y}+ \lambda\frac{\partial g}{\partial y}\right)\delta y + \left(\frac{\partial F}{\partial y'}+ \lambda\frac{\partial g}{\partial y'}\right)\delta y'\right]dx = 0, \, \forall \delta y \in H_\delta(y_0)\Rightarrow$$
аналогично получаем уравнение Эйлера
$$\frac{\partial(\mathcal{F}+\lambda\mathcal{G})}{\partial y}-\frac{d}{dx}\frac{\partial(\mathcal{F}+\lambda\mathcal{G})}{\partial y'}=0$$
В силу произвольности $\delta y\in H_\delta(y_0)$ теорема доказана
\end{proof}

\subsection{Задача Лагранжа}\par

Среди всех кривых $y=y(x)$, $z=z(x)$, лежащих на поверхности $g(x,y,z)=0$: $(g(x,y(x),z(x))=0)$ найти те, которые дают экстремум функционалу $\mathcal{J}=\int\limits_a^b F(x,y(x),y'(x),z(x),z'(x))d x$. Концы кривых закреплены, т.е. $y(a)=A_1$, $y(b)=B_1$, $z(a)=A_2$, $z(b)=B_2$.
Должно выполняться $g(a,A_1,A_2)= g(b,B_1,B_2)=0$. К обычным условиям на $F,\;y(x),\;z(x)$ добавляется условие, что $g(x,y,z)$ должна быть непрерывно дифференцируемой по совокупности переменных и $(g'_y)^2+(g'_x)^2\neq 0, \,\forall x \in [a;b]$, т.е $g$ -- простая гладкая поверхность без особых точек, назовем ее $S$.

Среди всех кривых, лежащих на $S$ и имеющих заданные концы, найти те, которые дают минимум $\mathcal{J}$.
\begin{theorem}
Пусть кривая $\gamma:a\leqslant x\leqslant b,\; y_0=y_0(x),\;z_0=z_0(x)$ является слабым экстремумом Лагранжа. Тогда $\exists \lambda=\lambda(x):$ первая вариация $F+\lambda g,$ т.е. $\delta(F+\lambda g)=0, \, \forall \delta y, \delta z$ ($\gamma$ является стационарной кривой для $\int\limits_a^b(F+\lambda g)d x$), т.е.
\begin{equation*}
\begin{cases}
\frac{\partial F}{\partial y} - \frac{d}{d x}\frac{\partial F}{\partial y'}+\lambda (x)g'_y=0 \text{ -- для $y(x)$} \\
\frac{\partial F}{\partial z} - \frac{d}{d x}\frac{\partial F}{\partial z'}+\lambda (x)g'_z=0 \text{ -- для $z(x)$}
\end{cases}\end{equation*}
\end{theorem}
\begin{proof}
$y(x)=y_0(x)+t\delta y;\;z(x)=z_0(x)+t \delta z.$ Рассматриваем кривые, лежащие на поверхности, т.е. $g(x,y_0+t\delta y,z_0+t \delta z)=0\Rightarrow \underbrace{g(x,y_0(x), z_0(x))}_{=0, \text{т.к. экстремаль лежит на } S}+g'_y\delta y \cdot t + g'_z\delta z \cdot t +o(t^2)=0 \underset{t \rightarrow 0}{\Rightarrow} $
\begin{equation}
    \label{Issue15:3.9}
    g'_y\delta y + g'_z\delta z = 0
\end{equation}
Таким образом в задаче Лагранжа допустимые вариации $\delta y$, $\delta z$ всегда связаны условием $\eqref{Issue15:3.9}$. Пусть $g'_z\neq 0$. Тогда $$\forall x: \delta z = -\frac{g'_y}{g'_z}\delta y\neq 0\Rightarrow (\delta z)'= -\left(\frac{g'_y}{g'_z}\right)'\delta y - \left(\frac{g'_y}{g'_z}\right)(\delta y)'$$

В таком случае:
\begin{multline*}
    \delta \mathcal{J}=\int\limits_a^b \left( \cfrac{\partial F}{\partial y}\delta y + \cfrac{\partial F}{\partial y'}(\delta y)' + \cfrac{\partial F}{\partial z}\delta z + \cfrac{\partial F}{\partial z'}(\delta z)' \right)d x = \\ 
    = \int\limits_a^b \left[\left(\frac{\partial F}{\partial y}-\frac{g'_y}{g'_z}\frac{\partial F}{\partial z}-\left(\frac{g'_y}{g'_z}\right)'\frac{\partial F}{\partial z'}\right)\delta y + \left(\frac{\partial F}{\partial y'}- \left(\frac{g'_y}{g'_z}\right)\frac{\partial F}{\partial z'}\right)\delta y'\right]d x = 
\end{multline*}

= [интегрируем по частям и учитываем закрепленные концы] = 
\[ 
    =\int\limits_a^b\left(\frac{\partial F}{\partial y}-\frac{d}{d x}\frac{\partial F}{\partial y'}-\left(\frac{g'_y}{g'_z}\right)\left(\frac{\partial F}{\partial z}-\frac{d}{d x}\frac{\partial F}{\partial z'}\right) \right)\delta y d x = 0 
\]
так как слабый экстремум $\forall \delta y, \delta z ~\eqref{Issue15:3.9}$ по основной лемме $\Rightarrow$

$$ \frac{\partial F}{\partial y} - \frac{d}{d x}\frac{\partial F}{\partial y'} - \left(\frac{g'_y}{g'_z}\right)\left(\frac{\partial F}{\partial z}-\frac{d}{d x}\frac{\partial F}{\partial z'}\right)=0$$
Обозначим $\lambda (x) = -\cfrac{\frac{\partial F}{\partial z}-\frac{d}{d x}\frac{\partial F}{\partial z'}}{g'_z}\Rightarrow$ уравнение для $y(x)$ принимает вид из условия.

Аналогично, выражая $\delta y$, $(\delta y)'$ 
$$-\lambda (x) g'_z-\left(\frac{\partial F}{\partial z}- \frac{d}{d x}\frac{\partial F}{\partial z'}\right)=0  \text{ -- уравнение для $z(x)$.}$$
\end{proof}