\subsection{Фундаментальная система и фундаментальная матрица решений линейной однородной
системы}

Будем рассматривать линейную однородную систему ДУ вида:

\begin{equation}
    \frac{d \overrightarrow x}{dt} = A(t) \overrightarrow x; ~~ \dot x^i = \sum^n_{k = 1} a^i_k(t) x^k; ~~ i, k = \overline{1, n}
    \label{4_2_lin_syst}
\end{equation}

\begin{proposition}
    Для однородных систем линейных уравнений верен принцип суперпозиции, т.е. если
    система функций $\varphi_1, \dots, \varphi_n$ -- решение системы уравнений, то любая их линейная комбинация тоже
    является решением.
\end{proposition}
\begin{proof}

    Введем оператор $L$ такой, что $L = \frac{d}{dt} - A$. Тогда однородная система ДУ $\frac{d \overrightarrow x}{dt} = A \overrightarrow x$ запишется в виде $L(\overrightarrow x) = 0$, неоднородная система ДУ $\frac{d \overrightarrow x}{dt} - A \overrightarrow x = \overrightarrow{q}(t)$ запишется в виде $L(\overrightarrow x) = \overrightarrow{q}(t)$.

    Пусть вектор-функции $\overrightarrow{\varphi}(t)$ и $\overrightarrow{\psi}(t)$ являются решениями системы $L(\overrightarrow x) = 0$, в таком случае справедливо
    \[ L(\overrightarrow{\varphi}(t)) = 0, \; L(\overrightarrow{\psi}(t)) = 0 \]
    Рассмотрим вектор-функцию $\overrightarrow{\chi}(t) = a \overrightarrow{\varphi}(t) + b \overrightarrow{\psi}(t)$, где $a$ и $b$ --  произвольные коэффициенты. Применим оператор $L$ к получившейся вектор-функции:
    
    \[ L(\overrightarrow{\chi}(t)) = \frac{d}{dt} \left(a \overrightarrow{\varphi}(t) + b \overrightarrow{\psi}(t) \right) - A \left(a \overrightarrow{\varphi}(t) + b \overrightarrow{\psi}(t) \right) = \]
    \[ = a \left(\frac{d}{dt} \overrightarrow{\varphi}(t) - A \overrightarrow{\varphi}(t) \right) + b \left(\frac{d}{dt} \overrightarrow{\psi}(t) - A \overrightarrow{\psi}(t) \right) = \]
    \[ = a L(\overrightarrow{\varphi}(t)) + b L(\overrightarrow{\psi}(t)) = 0 \]

\end{proof}

\begin{definition}
    Пусть имеется система вектор-функций $\overrightarrow \varphi_1(t), \dots, \overrightarrow \varphi_n(t)$
    \begin{equation}
        \overrightarrow \varphi_i(t) =
        \begin{pmatrix}
            \varphi_i^1(t) \\
            \dots \\
            \varphi_i^n(t) \\
        \end{pmatrix}           
    \end{equation}
    непрерывная на $I(x)$, тогда такая система называется
    линейно зависимой на $I$, если \[\exists ~ C_1, \dots, C_n : \sum^n_{i = 1} |C_i| \neq 0 ~ \& ~ \sum^n_{i = 1} C_i \overrightarrow \varphi_i(t) = 0 ~ \forall t \in I\]
    В противном случае, система вектор-функций называется линейно независимой, то есть условие
    \[\sum^n_{i = 1} C_i \overrightarrow \varphi_i(t) = 0 ~ \forall t \in I\] выполняется только при $C_1 = C_2 = \dots = C_n = 0$.
\end{definition}

\begin{definition}
    Пусть система вектор-функций $\overrightarrow \varphi_1(t), \dots, \overrightarrow \varphi_n(t)$ линейно независима на $I$ и каждая вектор-функция
    $\overrightarrow \varphi_i(t)$ является решением системы ДУ $\frac{d \overrightarrow x}{dt} = A \overrightarrow x$. Тогда такая система вектор-функций
    называется фундаментальной системой решений (ФСР) данной системы ДУ.
\end{definition}

\begin{theorem}
    Рассмотрим систему ДУ $\frac{d \overrightarrow x}{dt} = A(t) \overrightarrow x$. Если матрица $A(t)$ является непрерывной на отрезке $[a, b]$, то система
    имеет ФСР на этом отрезке.
\end{theorem}
\begin{proof}
    Матрица $A(t)$ является непрерывной на отрезке $[a, b]$, тогда, согласно основной теореме $\eqref{4_1_Cauchy}$, на отрезке $[a, b]$ существует единственное решение задачи Коши.
    
    Пусть система функций $\overrightarrow{\varphi_1}(t), \overrightarrow{\varphi_2}(t), \dots, \overrightarrow{\varphi_n}(t)$ является решением системы при следующих заданных условиях:
    \begin{equation}
        \overrightarrow{\varphi_1}(t_0) = 
        \begin{pmatrix}
            1 \\
            0 \\
            \dots \\
            0 \\
        \end{pmatrix}, ~
        \overrightarrow{\varphi_2}(t_0) = 
        \begin{pmatrix}
        0 \\
        1 \\
        \dots \\
        0 \\
        \end{pmatrix}, ~
        \dots, ~
        \overrightarrow{\varphi_n}(t_0) = 
        \begin{pmatrix}
        0 \\
        0 \\
        \dots \\
        1 \\
        \end{pmatrix}, ~
    \end{equation}
    тогда вронскиан такой системы в точке $t_0$ (про вронскиан и его свойства подробнее смотри следующие пункты):
    
    \begin{equation}
    W(t_0) = |\overrightarrow{\varphi_1}(t_0), \overrightarrow{\varphi_2}(t_0), \dots, \overrightarrow{\varphi_n}(t_0)| =
    \begin{vmatrix}
        1 ~ 0 ~ \dots ~ 0 \\
        0 ~ 1 ~ \dots ~ 0 \\
            ~~  \dots     \\
        0 ~ 0 ~ \dots ~ 1 \\
    \end{vmatrix} = 1 \neq 0.
    \end{equation}
    Тогда, из свойства вронскиана следует, что данная система функций является линейно независимой, а так как каждая функция является решением системы ДУ, эта система вектор-функций и есть ФСР системы ДУ.
\end{proof}

\begin{theorem}
    Пусть система вектор-функций $\overrightarrow \varphi_1(t), \dots, \overrightarrow \varphi_n(t)$ является ФСР системы ДУ, тогда
    любое решение этой системы ДУ можно представить, как линейную комбинацию компонентов ФСР: 
    $\overrightarrow x(t) = C_1 \overrightarrow \varphi_1(t) + \dots + C_n \overrightarrow \varphi_n(t)$, где $C_1, \dots, C_n$ -- произвольные постоянные.
\end{theorem}
\begin{proof}
    Так как для системы ДУ справедлив принцип суперпозиции, то вектор-функция $\overrightarrow{x(t)} = C_1 \overrightarrow \varphi_1(t) + \dots + C_n \overrightarrow \varphi_n(t)$ является решением системы ДУ.
    
    Предположим теперь, что существует функция $\overrightarrow{\chi}(t)$ такая, что она является решением системы ДУ, но не представима в виде $C_1 \overrightarrow \varphi_1(t) + \dots + C_n \overrightarrow \varphi_n(t)$. Пусть значение этой функции в точке $t_0$:
    
    \begin{equation}
        \overrightarrow{\chi}(t_0) = 
        \begin{pmatrix}
            \alpha_1 \\
            \alpha_2 \\
            \dots \\
            \alpha_n \\
        \end{pmatrix} = 
        \begin{pmatrix}
            \chi_1(t_0) \\
            \chi_2(t_0) \\
            \dots \\
            \chi_n(t_0) \\
        \end{pmatrix}
    \end{equation}
    
    Теперь составим следующую систему уравнений
    \begin{equation}
        \begin{cases}
            C_1 \varphi_1^1(t_0) + C_2 \varphi_2^1(t_0) + \dots + C_n \varphi_n^1(t_0) = \alpha_1 \\
            C_1 \varphi_1^2(t_0) + C_2 \varphi_2^2(t_0) + \dots + C_n \varphi_n^2(t_0) = \alpha_2 \\
            \dots \\
            C_1 \varphi_1^n(t_0) + C_2 \varphi_2^n(t_0) + \dots + C_n \varphi_n^n(t_0) = \alpha_3, \\
        \end{cases}
    \end{equation}
    где $C_1, ~ C_2, ~ \dots, ~ C_n$ -- являются неизвестными, которые надо найти. Определителем этой системы является
    
    \begin{equation}
        W(t_0) = 
        \begin{vmatrix}
        \varphi_1^1(t_0) ~ \varphi_2^1(t_0) ~ \dots ~ \varphi_n^1(t_0) \\
        \varphi_1^2(t_0) ~ \varphi_2^2(t_0) ~ \dots ~ \varphi_n^2(t_0) \\
        \dots \\
        \varphi_1^n(t_0) ~ \varphi_2^n(t_0) ~ \dots ~ \varphi_n^n(t_0) \\
        \end{vmatrix} \neq 0,
    \end{equation}
    который не равен $0$, а поскольку функции $\overrightarrow{\varphi_i}, i = \overline{1, n}$, являются ФСР системы ДУ, то числа $C_1, ~ C_2, ~ \dots, ~ C_n$ определяются однозначно.
    
    С этими числами рассмотрим решение исходной системы ДУ, назовем его $\overrightarrow{z}(t) = C_1 \overrightarrow \varphi_1(t) + \dots + C_n \overrightarrow \varphi_n(t)$. Поскольку $\overrightarrow{\chi}(t)$ и $\overrightarrow{z}(t)$ -- являются решениями системы ДУ, по принципу суперпозиции функция $\overrightarrow{\psi}(t) = \overrightarrow{z}(t) - \overrightarrow{\chi}(t)$ так же является решением этой системы ДУ.
    
    Заметим, что значение этой функции в точке $t_0$: $\overrightarrow{\psi}(t_0) = \overrightarrow{z}(t_0) - \overrightarrow{\chi}(t_0) = 0$, заметим так же, что $\overrightarrow{0}$ является решением однородной системы системы $\frac{d}{dt} \overrightarrow{x} - A \overrightarrow{x}$. Тогда, в силу теоремы о существовании и единственности решения задачи Коши, выполняется:
    
    \[ \overrightarrow{\psi}(t) = 0 ~ \forall ~ t \in I ~ \Rightarrow \]
    \[ \overrightarrow{\psi}(t) = \overrightarrow{z}(t) - \overrightarrow{\chi}(t) \equiv 0 ~ \forall ~ t \in I ~ \Rightarrow \]
    \[ \overrightarrow{z}(t) = \overrightarrow{\chi}(t) = C_1 \overrightarrow{\varphi_1}(t) + C_2 \overrightarrow{\varphi_2}(t) + \dots + C_n \overrightarrow{\varphi_n}(t).\]
    
    Мы получили противоречие с предположением о невозможности линейного представления решения $\overrightarrow{\chi}(t)$ через функции ФСР, таким образом, мы доказали, что любое решение системы ДУ можно представить как линейную комбинацию компонентов ФСР.
\end{proof}

\begin{definition}
    Решение системы ДУ вида $\overrightarrow x(t) = C_1 \overrightarrow \varphi_1(t) + \dots + C_n \overrightarrow \varphi_n(t)$, где $C_1, \dots, C_n$,
    называется общим решением системы ДУ.
\end{definition}

\subsection{Структура общего решения линейной однородной и неоднородной систем}
%% повторяется определение линейного оператора L
Введем оператор $L$ такой, что $L = \frac{d}{dt} - A$. Тогда однородная система ДУ $\frac{d \overrightarrow x}{dt} = A \overrightarrow x$ запишется в виде
$L(\overrightarrow x) = 0$, неоднородная система ДУ $\frac{d \overrightarrow x}{dt} - A \overrightarrow x = \overrightarrow{q}(t)$ запишется в виде $L(\overrightarrow x) = \overrightarrow{q}(t)$.

\begin{proposition}
    Общее решение неоднородной системы ДУ $\frac{d \overrightarrow x}{dt} - A \overrightarrow x = \overrightarrow{q}(t)$ представляет собой следующее выражение:
    \begin{equation}
        \overrightarrow x = \overrightarrow x^s + \overrightarrow x^{\text{об}}_0,
    \end{equation}
    где $\overrightarrow x^s$ является частным решением линейного неоднородного уравнение, т. е. $L(\overrightarrow x^s) = q(t)$, а
    $\overrightarrow x^{\text{об}}_0$ -- общее решение системы линейных \textbf{однородных} уравнений $L(\overrightarrow x^{\text{об}}_0) = 0$.
    Таким образом, получаем:
    \[L(\overrightarrow x) = L(\overrightarrow x^s + \overrightarrow x^{\text{об}}_0) = L(\overrightarrow x^s) + L(\overrightarrow x^{\text{об}}_0) = q(t) + 0.\]
\end{proposition}

\subsection{Определитель Вронского и его свойства}

\subsubsection{Определитель Вронского}

\begin{definition}
    Пусть на $I$ определена система вектор-функций $\overrightarrow \varphi_1(t), \dots, \overrightarrow \varphi_n(t)$, тогда определитель
    \begin{equation}
        W(t) = 
        \begin{vmatrix}
            \varphi^1_1(t) \dots \varphi^1_n(t) \\
            \dots ~~~~~~~~ \dots \\
            \varphi^n_1(t) \dots \varphi^n_n(t) \\
        \end{vmatrix}
    \end{equation}
    называется определителем Вронского, где
    \begin{equation}
        \overrightarrow{\varphi_i} = 
        \begin{pmatrix}
            \varphi_i^1 \\
            \dots \\
            \varphi_i^n
        \end{pmatrix}.
    \end{equation}
    Другими словами
    \begin{equation}
    W(t) = 
    \begin{vmatrix}
    \overrightarrow{\varphi_1}, \dots, \overrightarrow{\varphi_n}
    \end{vmatrix}.
    \end{equation}
\end{definition}

\begin{theorem}
    Если $\exists t_0 \in I : ~ W(t_0) \neq 0$, то система вектор-функций $\overrightarrow \varphi_1(t), \dots, \overrightarrow \varphi_n(t)$, которые являются решениями линейной систему ДУ $\eqref{4_2_lin_syst}$, является линейно независимой на $I$. Обратное неверно,
    например:
    \begin{equation}
        \varphi_1 = 
        \begin{pmatrix}
            t \\
            0
        \end{pmatrix}, ~
        \varphi_2 = 
        \begin{pmatrix}
            1 \\
            0
        \end{pmatrix} ~ \text{ЛНЗ решения, но} ~~ W(t) = 0.
    \end{equation}
    \label{4_2_wronskian_theorem}
\end{theorem}

\begin{proof}
    Будем доказывать от противного: пусть система является линейно зависимой, тогда $\exists ~ C_1, \dots, C_n: \sum^n_{i = 1} |C_i| \neq 0$,
    $C_1 \overrightarrow \varphi_1(t) + \dots + C_n \overrightarrow \varphi_n(t) = 0 ~ \forall t \in I$. Тогда в матрице Вронского $W(t)$
    есть хотя бы два линейно зависимых столбца, так как $\overrightarrow \varphi_i(t)$ являются столбцами матрицы, но тогда получим, что
    $W(t) = 0 ~ \forall t \in I$ (хотя предполагалось, что $\exists ~ t_0 \in I : ~ W(t_0) \neq 0$). Таким образом, мы получили противоречие,
    откуда следует, что система является линейно независимой на $I$.
\end{proof}

\subsubsection{Свойства Вронскиана}
\label{wr_properties}

\begin{enumerate}
    \item Если $\exists t_0 \in I : ~ W(t_0) \neq 0$, то система вектор-функций $\overrightarrow \varphi_1(t), \dots, \overrightarrow \varphi_n(t)$, которые являются решениями линейной систему ДУ $\eqref{4_2_lin_syst}$, является линейно независимой на $I$ (см. доказательство теоремы $\eqref{4_2_wronskian_theorem}$).
    \item Пусть вектор-функции $\overrightarrow \varphi_1(t), \dots, \overrightarrow \varphi_n(t)$ являются решениями линейной системы ДУ $\eqref{4_2_lin_syst}$, и $\exists \; t_0 \in I: ~ W(t_0) = 0$, тогда система $\overrightarrow \varphi_1(t), \dots, \overrightarrow \varphi_n(t)$ является линейно зависимой.

    \begin{proof}
        Поскольку $W(t_0) = 0$, то столбцы этой матрицы являются линейно зависимыми, то есть
        
        \[ \exists ~ C_1, C_2, \dots, C_n : ~ \sum_{i = 1}^n |C_i| \neq 0 ~ \& ~ \sum_{i = 1}^n C_i \overrightarrow{\varphi_i}(t_0) = 0 \]
        
        Используя данные коэффициенты, построим функцию-решение системы ДУ $\overrightarrow{x}(t) = \sum_{i = 1}^n C_i \overrightarrow{\varphi_i}(t)$. Заметим, что $\overrightarrow{x}(t_0) = 0$. Тогда, так как существует решение системы $x(t) \equiv 0$, то в силу теоремы о существовании и единственности решения задачи Коши выполняется: $\overrightarrow{x}(t) = \sum_{i = 1}^n C_i \overrightarrow{\varphi_i}(t) = 0 ~ \forall ~ t \in I$, что означает, что система $\overrightarrow{\varphi_i}$ является линейно зависимой.
    \end{proof}
    
\end{enumerate}
