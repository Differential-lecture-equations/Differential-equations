\subsection{Фундаментальная система и фундаментальная матрица решений линейной однородной
системы}

Будем рассматривать однородную систему ДУ вида:

\begin{equation}
    \frac{d \vec x}{dt} = A \vec x; ~~ \dot x^i = \sum^n_{k = 1} a^i_k x^k; ~~ i, k = \overline{1, n} 
\end{equation}

\begin{proposition}
    Для однородных систем линейных уравнений верен принцип суперпозиции, т.е. если
    система функций $\varphi_1, \dots, \varphi_n$ -- решение системы уравнений, то любая их линейная комбинация тоже
    является решением.
\end{proposition}
\begin{proof}
    %% возможно, имеет смысл вынести определение данного оператора в отдельное определение.
    Введем оператор $L$ такой, что $L = \frac{d}{dt} - A$. Тогда однородная система ДУ $\frac{d \vec x}{dt} = A \vec x$ запишется в виде
    $L(\vec x) = 0$, неоднородная система ДУ $\frac{d \vec x}{dt} - A \vec x = q(t)$ запишется в виде $L(\vec x) = q(t)$.
    
    Пусть вектор-функции $\vec{\varphi}(t)$ и $\vec{\psi}(t)$ являются решениями системы $L(\vec x) = 0$, в таком случае справедливо
    \[ L(\vec{\varphi}(t)) = 0; ~~ L(\vec{\psi}(t)) = 0 \]
    Рассмотрим вектор-функцию $\vec{\chi}(t) = a \vec{\varphi}(t) + b \vec{\psi}(t)$, где $a$ и $b$ --  произвольные коэффициенты. Применим оператор $L$ к получившейся вектор-функции:
    
    \[ L(\vec{\chi}(t)) = \frac{d}{dt} \left(a \vec{\varphi}(t) + b \vec{\psi}(t) \right) - A \left(a \vec{\varphi}(t) + b \vec{\psi}(t) \right) = \]
    \[ = a \left(\frac{d}{dt} \vec{\varphi}(t) - A \vec{\varphi}(t) \right) + b \left(\frac{d}{dt} \vec{\psi}(t) - A \vec{\psi}(t) \right) = \]
    \[ = a L(\vec{\varphi}(t)) + b L(\vec{\psi}(t)) = 0 \]
    
\end{proof}

\begin{definition}
    Пусть имеется система вектор-функций $\vec \varphi_1(t), \dots, \vec \varphi_n(t)$
    \begin{equation}
        \vec \varphi_i(t) =
        \begin{pmatrix}
            \varphi_i^1(t) \\
            \dots \\
            \varphi_i^n(t) \\
        \end{pmatrix}           
    \end{equation}
    непрерывна на $I(x)$, тогда такая система называется
    линейно-зависимой на $I$, если \[\exists ~ C_1, \dots, C_n : \sum^n_{i = 1} |C_i| \neq 0 ~ \& ~ \sum^n_{i = 1} C_i \vec \varphi_i(t) = 0 ~ \forall t \in I\]
    В противном случае, система вектор-функций называется линейно-независимой, то есть условие
    \[\sum^n_{i = 1} C_i \vec \varphi_i(t) = 0 ~ \forall t \in I\] выполняется только при $C_1 = C_2 = \dots = C_n = 0$.
\end{definition}

\begin{definition}
    Пусть система вектор-функций $\vec \varphi_1(t), \dots, \vec \varphi_n(t)$ линейно-независима на $I$ и каждая вектор-функция
    $\vec \varphi_i(t)$ является решением системы ДУ $\frac{d \vec x}{dt} = A \vec x$.Тогда такая система вектор-функций
    называется фундаментальной системой решений (ФСР) данной системы ДУ.
\end{definition}

\begin{theorem}
    Рассмотрим систему ДУ $\frac{d \vec x}{dt} = A \vec x$. Если матрица $A$ является непрерывной на отрезке $[a, b]$, то система
    имеет ФСР на этом отрезке.
\end{theorem}
\begin{proof}
    матрица $A$ является непрерывной на отрезке $[a, b]$, тогда, согласно основной теореме, на отрезке $[a, b]$ существует единственное решение задачи Коши.
    
    Пусть система функций $\vec{\varphi_1}(t), \vec{\varphi_2}(t), \dots, \vec{\varphi_n}(t)$ является решением системы при следующих заданных условиях:
    \begin{equation}
        \vec{\varphi_1}(t_0) = 
        \begin{pmatrix}
            1 \\
            0 \\
            \dots \\
            0 \\
        \end{pmatrix}, ~
        \vec{\varphi_2}(t_0) = 
        \begin{pmatrix}
        0 \\
        1 \\
        \dots \\
        0 \\
        \end{pmatrix}, ~
        \dots, ~
        \vec{\varphi_n}(t_0) = 
        \begin{pmatrix}
        0 \\
        0 \\
        \dots \\
        1 \\
        \end{pmatrix}, ~
    \end{equation}
    тогда вронскиан такой системы в точке $t_0$ (про вронскиан и его свойства подробнее смотри следующие пункты):
    
    \begin{equation}
    W(t_0) = |\vec{\varphi_1}(t_0), \vec{\varphi_2}(t_0), \dots, \vec{\varphi_n}(t_0)| =
    \begin{vmatrix}
        1 ~ 0 ~ \dots ~ 0 \\
        0 ~ 1 ~ \dots ~ 0 \\
            ~~  \dots     \\
        0 ~ 0 ~ \dots ~ 1 \\
    \end{vmatrix} = 1 \neq 0
    \end{equation}
    
    тогда, из свойства вронскиана следует, что данная система функций является линейно-независимой, а так как каждая функция является решением системы ДУ, эта система вектор-функций и есть ФСР системы ДУ.
\end{proof}

\begin{theorem}
    Пусть система вектор-функций $\vec \varphi_1(t), \dots, \vec \varphi_n(t)$ является ФСР системы ДУ, тогда
    любое решение этой системы ДУ можно представить, как линейную комбинацию компонентов ФСР: 
    $\vec x(t) = C_1 \vec \varphi_1(t) + \dots + C_n \vec \varphi_n(t)$, где $C_1, \dots, C_n$ -- произвольные постоянные.
\end{theorem}
\begin{proof}
    Так как для системы ДУ справедлив принцип суперпозиции, то вектор-функция $\vec{x(t)} = C_1 \vec \varphi_1(t) + \dots + C_n \vec \varphi_n(t)$ является решением системы ДУ.
    
    Предположим теперь, что существует функция $\vec{\chi}(t)$ такая, что она является решением системы ДУ, но не представима в виде $C_1 \vec \varphi_1(t) + \dots + C_n \vec \varphi_n(t)$. Пусть значение этой функции в точке $t_0$:
    
    \begin{equation}
        \vec{\chi}(t_0) = 
        \begin{pmatrix}
            \alpha_1 \\
            \alpha_2 \\
            \dots \\
            \alpha_n \\
        \end{pmatrix} = 
        \begin{pmatrix}
            \chi_1(t_0) \\
            \chi_2(t_0) \\
            \dots \\
            \chi_n(t_0) \\
        \end{pmatrix}
    \end{equation}
    
    Теперь составим следующую систему уравнений
    \begin{equation}
        \begin{cases}
            C_1 \varphi_1^1(t_0) + C_2 \varphi_2^1(t_0) + \dots + C_n \varphi_n^1(t_0) = \alpha_1 \\
            C_1 \varphi_1^2(t_0) + C_2 \varphi_2^2(t_0) + \dots + C_n \varphi_n^2(t_0) = \alpha_2 \\
            \dots \\
            C_1 \varphi_1^n(t_0) + C_2 \varphi_2^n(t_0) + \dots + C_n \varphi_n^n(t_0) = \alpha_3 \\
        \end{cases}
    \end{equation}
    
    где $C_1, ~ C_2, ~ \dots, ~ C_n$ -- являются неизвестными, который надо найти. Определителем этой системы является
    
    \begin{equation}
        W(t_0) = 
        \begin{vmatrix}
        \varphi_1^1(t_0) ~ \varphi_2^1(t_0) ~ \dots ~ \varphi_n^1(t_0) \\
        \varphi_1^2(t_0) ~ \varphi_2^2(t_0) ~ \dots ~ \varphi_n^2(t_0) \\
        \dots \\
        \varphi_1^n(t_0) ~ \varphi_2^n(t_0) ~ \dots ~ \varphi_n^n(t_0) \\
        \end{vmatrix} \neq 0
    \end{equation}
    данный определитель не равен 0, поскольку функции $\vec{\varphi_i} ~ i = \vec{1n}$ являются ФСР системы ДУ, поэтому числа $C_1, ~ C_2, ~ \dots, ~ C_n$ определяются однозначно.
    
    С этими числами рассмотрим решение исходной системы ДУ, назовем его $\vec{z}(t) = C_1 \vec \varphi_1(t) + \dots + C_n \vec \varphi_n(t)$. Поскольку $\vec{\chi}(t)$ и $\vec{z}(t)$ -- являются решениями системы ДУ, по принципу суперпозиции функция $\vec{\psi}(t) = \vec{z}(t) - \vec{\chi}(t)$ так же является решением этой системы ДУ. 
    
    Заметим, что значение этой функции в точке $t_0$: $\vec{\psi}(t_0) = \vec{z}(t_0) - \vec{\chi}(t_0) = 0$, заметим так же, что $\vec{0}$ является решением однородной системы системы $\frac{d}{dt} \vec{x} - A \vec{x}$. Тогда, в силу теоремы о существовании и единственности решения задачи Коши, выполняется:
    
    \[ \vec{\psi}(t) = 0 ~ \forall ~ t \in I ~ \Rightarrow \]
    \[ \vec{\psi}(t) = \vec{z}(t) - \vec{\chi}(t) \equiv 0 ~ \forall ~ t \in I ~ \Rightarrow \]
    \[ \vec{z}(t) = \vec{\chi}(t) = C_1 \vec{\varphi_1}(t) + C_2 \vec{\varphi_2}(t) + \dots + C_n \vec{\varphi_n}(t) \]
    
    Мы получили противоречие с предположением о невозможности линейного представления решения $\vec{\chi}(t)$ через функции ФСР, таким образом, мы доказали, что любое решение системы ДУ можно представить, как линейную комбинацию компонентов ФСР.
\end{proof}

\begin{definition}
    Решение системы ДУ вида $\vec x(t) = C_1 \vec \varphi_1(t) + \dots + C_n \vec \varphi_n(t)$, где $C_1, \dots, C_n$
    называется общим решением сисстемы ДУ.
\end{definition}

\subsection{Структура общего решения линейной однородной и неоднородной систем}
%% повторяется определение линейного оператора L
Введем оператор $L$ такой, что $L = \frac{d}{dt} - A$. Тогда однородная система ДУ $\frac{d \vec x}{dt} = A \vec x$ запишется в виде
$L(\vec x) = 0$, неоднородная система ДУ $\frac{d \vec x}{dt} - A \vec x = q(t)$ запишется в виде $L(\vec x) = q(t)$.

\begin{proposition}
    Общее решение неоднородной системы ДУ $\frac{d \vec x}{dt} - A \vec x = q(t)$ представляет собой следующее выражение:
    \begin{equation}
        \vec x = \vec x^s + \vec x^{\text{об}}_0
    \end{equation}
    где $\vec x^s$ -- частное решение линейного неоднородного уравнение, т. е. $L(\vec x^s) = q(t)$, а
    $\vec x^{\text{об}}_0$ -- общее решение системы линейных \textbf{однородных} уравнений $L(\vec x^{\text{об}}_0) = 0$.
    Таким образом, получаем:
    \[L(\vec x) = L(\vec x^s + \vec x^{\text{об}}_0) = L(\vec x^s) + L(\vec x^{\text{об}}_0) = q(t) + 0\]
\end{proposition}

\subsection{Определитель Вронского и его свойства}

\subsubsection{Определитель Вронского}

\begin{definition}
    Пусть на $I$ определена система вектор-функций $\vec \varphi_1(t), \dots, \vec \varphi_n(t)$, тогда определитель
    \begin{equation}
        W(t) = 
        \begin{vmatrix}
            \varphi^1_1(t) \dots \varphi^1_n(t) \\
            \dots ~~~~~~~~ \dots \\
            \varphi^n_1(t) \dots \varphi^n_n(t) \\
        \end{vmatrix}
    \end{equation}
    называется определителем Вронского, где
    \begin{equation}
        \vec{\varphi_i} = 
        \begin{pmatrix}
            \varphi_i^1 \\
            \dots \\
            \varphi_i^n
        \end{pmatrix}
    \end{equation}
    другими словами
    \begin{equation}
    W(t) = 
    \begin{vmatrix}
    \vec{\varphi_1}, \dots, \vec{\varphi_n}
    \end{vmatrix}
    \end{equation}
\end{definition}

\begin{theorem}
    Если $\exists ~ t_0 \in I : ~ W(t_0) \neq 0$, то система является линейно независимой на $I$. Обратное неверно,
    пример:
    \begin{equation}
        \varphi_1 = 
        \begin{pmatrix}
            t \\
            0
        \end{pmatrix}, ~
        \varphi_2 = 
        \begin{pmatrix}
            1 \\
            0
        \end{pmatrix} ~ \text{ЛНЗ, но} ~~ W(t) = 0
    \end{equation}
\end{theorem}

\begin{proof}
    Будем доказывать от противного: пусть система является линейно-зависимой, тогда $\exists ~ C_1, \dots, C_n:$
    $C_1 \vec \varphi_1(t) + \dots + C_n \vec \varphi_n(t) = 0 ~ \forall t \in I$. Тогда в определителе Вронского $W(t)$
    есть хотя бы два линейно-зависымих столбца, так как $\vec \varphi_i(t)$ являются столбцами определителя, но тогда получам, что
    $W(t) = 0 ~ \forall t \in I$ (хотя предпологалось, что $\exists ~ t_0 \in I : ~ W(t_0) \neq 0$). Таким образом, мы получили противоречие,
    откуда следует, что система является линейно независимой на $I$.
\end{proof}

\subsubsection{Свойства Вронскиана}

\begin{enumerate}
    \item Если $\exists ~ t_0 \in I : ~ W(t_0) \neq 0$, то система является линейно независимой на $I$ (см. доказательство теоремы).
    \item Пусть вектор-функции $\vec \varphi_1(t), \dots, \vec \varphi_n(t)$ являются решениями системы ДУ, и существует точка
    $t_0 \in I: ~ W(t_0) = 0$, тогда система $\vec \varphi_1(t), \dots, \vec \varphi_n(t)$ является линейно-зависимой.

    \begin{proof}
        Поскольку $W(t_0) = 0$ столбцы этой матрицы являются линейнозависимыми, то есть 
        
        \[ \exists ~ C_1, C_2, \dots, C_n : ~ \sum_{i = 1}^n C_i^2 \neq 0 ~ \& ~ \sum_{i = 1}^n C_i \vec{\varphi_i}(t_0) = 0 \]
        
        Используя данные коэффициенты, построим функцию $\vec{x}(t) = C_i \vec{\varphi_i}(t)$. Заметим, что во-первых $\vec{x}(t_0) = 0$, а во-вторых данная функция является решением системы ДУ в силу теоремы о суперпозиции. Тогда, в силу теоремы о существовании и единственности решения задачи Коши выполняется: $\vec{x}(t) = C_i \vec{\varphi_i}(t) = 0 ~ \forall ~ t \in I$, что означает, что система $\vec{\varphi_i}$ является линейнозависимой.
    \end{proof}
\end{enumerate}