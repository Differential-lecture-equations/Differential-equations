
\section{Билет 3. Линейные дифференциальные уравнения и линейные системы дифференциальных уравнений с постоянными коэффициентами}

\subsection{Вводная часть}
\subsubsection{Понятие кольца. Рассмотрение понятия многочленов}

\begin{definition}
Кольцом $K$ называют множество, на котором определены две операции: сложение и умножение, сопоставляющее упорядоченным парам элементов их $"$сумму$"$, $"$произведение$"$, являющимся элементами этого же множества.
\end{definition}
Рассмотрим кольцо, в котором действия $+$ и $\cdot$ удовлетворяют следующим условиям (первые $6$ -- определение кольца):
\begin{enumerate}
    \item $(a+b)+c=a+(b+c)\;\;\;\forall a, b, c \in K$
    \item $a+b=b+a\;\;\;\forall a, b\in K$
    \item $\exists 0\in K:\;\;\; a+0=a \;\;\;\forall a\in K$
    \item $\forall ~ a ~ \in ~ K ~ \exists -a ~ \in ~ K: a+(-a)=0 ~ \forall a ~\in ~ K$
    \item $(a+b) \cdot c = ac + bc \;\;\; \forall a, b, c \in K$
    \item $c\cdot(a+b)=ca+cb\;\;\; \forall a, b, c\in K$
    \item $(ab)c=a(bc)\;\;\; \forall a, b, c\in K$
    \item $ab=ba \;\;\;\forall a, b\in K$
    \item $\exists 1\in K: a\cdot 1 =1\cdot a = a\;\;\;\forall a\in K$
    \item $\exists a^{-1} \in K: a^{-1}a=aa^{-1}=1\;\;\;\forall a\in K$
\end{enumerate}
\begin{proposition}
Если $a+x=a+y$, то $x=y$
\end{proposition}
\begin{proof}
$$(-a)+(a+x)=(-a)+(a+y)\Rightarrow \big((-a)+a\big) +x= \big((-a)+a\big) +y=0+x=x=0+y=y$$
Отсюда следует единственность нуля и противоположного элемента:
$$(-a)\neq (-a)'$$$$0=a+(-a)=a+(-a)'\Rightarrow(-a)=(-a)'$$
\end{proof}
\begin{proposition}
$a\cdot 0=0\cdot a=0 \;\;\;\forall a$
\end{proposition}
\begin{proof}
$a \cdot 0+0=a \cdot 0 = a(0+0) \Rightarrow a \cdot 0 =0$; аналогично $0+0 \cdot a=0 \cdot a=(0 + 0) \cdot a =0 \cdot a + 0 \cdot a \Rightarrow 0 \cdot a =0$
\end{proof}
\begin{proposition}
Единица единственна
\end{proposition}
\begin{proof}
Пусть $1\neq1': 1=1\cdot 1'=1'\cdot 1=1'$
\end{proof}
\begin{itemize}
    \item Кольцо называется ассоциативным, если выполнено условие 7; коммутативным, если выполненно 8. Если выполнено условие 9, то говорят о кольце с единицей.
    \item Ассоциативное кольцо называется областно целостным, если из $ab=0\Rightarrow a=0\bigvee b=0$
    \item Полем называется коммутативное ассоциативное кольцо с единицей, в котором каждый отличный от нуля элемент имеет обратный. 
    
\end{itemize}
\begin{proposition}
Любое поле является областно целостным
\end{proposition}
\begin{proof}
$ab=0, \;\;a \neq 0 \Rightarrow a^{-1} \cdot (ab)=a^{-1} \cdot 0=0=(a^{-1}a) \cdot b=1 \cdot b =b \Rightarrow b=0$
\end{proof}
\subsubsection{Многочлен}
Пусть $A~-~$коммутативное ассоциативное кольцо с единицей. Одночленом от $x$ с коэффициентом из $A$ называется выражение $ax^m, \;\;a\in A, \;\;m\in \mathbb{N}$. По определению положим, что $ax^0=a$. Выражение $ax^m$ будем рассматривать как символ, для которого выпоняется по определению: $$ax^m+bx^m=(a+b)x^m$$
$$ax^m\cdot bx^n= a \cdot b x^{m+n}$$
Выражение, состоящее из нескольких одночленов, соединенных знаком $+$ назовем многочленом от $x$ с коэффициентами из $A$. Без нарушения общности, в силу коммутативности сложения запишем в каноническом виде: $P_n(x)=a_0+a_1x+\dots+a_nx^n$
\begin{enumerate}
    \item\label{1} Многочлены $P_n(x)=a_0+a_1x+\dots+a_nx^n$ и $Q_m(x)=b_0+b_1x\dots+b_mx^m $ считаем равными в том и только в том случае,  если $n=m$ и $a_k=b_k, \;\;\;k=\overline{1, n}$
    \item Суммой двух многочленов $P_n(x)$ и $Q_n(x)$ называется многочлен, получившейся посредством объединения одночленов соответствующих слагаемых:$$P(x)+Q_m(x)=(a_0+a_1x+\dots+a_nx^n)+(b_0+b_1x^1+\dots+b_mx^m)=a_0+b_0+ (a_1+b_1)x+\dots+c_sx^s$$ 
    $$s=max\{n, m\} $$
    $c_s=a_s+b_s, a_s=0, $ если $s>n$ и $b_s=0, $ если $s>m$\par Так определенное сложение многочленов коммутативно и ассоциативно.\par Имеется нулевой элемент $0=0 \cdot x + \dots + 0 \cdot x^n, $ а также противоположный $\big(-P_n(x)\big)=(-a_0)+(-a_1)x++\dots+(-a^n)x^n$
    \item Произведением двух многочленов называют многочлен, составленный их произведения всех членов первого сомножителя на все члены второго.
    $$P_n(x)\cdot Q_m(x)=a_0b_0+(a_0b_1+a_1b_0)x+\dots+\Big(\displaystyle\sum_{j=k+l}a_kb_l \Big)x^j+\dots+a_nb_mx^{n+m}$$
    
\end{enumerate}
\begin{itemize}
    \item Покажем, что так определенное умножение будет коммутативно и ассоциативно:
    $$P_n(x)\cdot Q_m(x)=a_0b_0+(a_0b_1+a_1b_0)x+\dots+\Big(\displaystyle\sum_{j=k+l}a_kb_l \Big)x^j+\dots+a_nb_mx^{n+m}$$
    В сумме $\displaystyle\sum_{j=k+l}a_kb_l $ заменим $k\leftrightarrow l\Rightarrow \displaystyle\sum_{j=k+l}b_ka_l=\displaystyle\sum_{j=k+l}b_la_k=\displaystyle\sum_{j=k+l}a_lb_k\underset{(\ref{1})}{\Rightarrow}P_n(x)\cdot Q_m(x)=Q_m(x)\cdot P_n(x)\Rightarrow$ коммутативно.\par
    Пусть $R_s(x)=c_0+c_1x+\dots+c_sx^s\Rightarrow(P_n(x)\cdot Q_m(x))\cdot R_s(x)= \big((a_0b_0)c_0\big)+\Big(\displaystyle\sum_{\gamma=j+\sigma}\Big(\displaystyle\sum_{j=k+l}a_kb_l\Big)c_\sigma\Big)x^\gamma + (a_nb_m)c_Sx^{n+m+s}), \;\;\;j=1, \dots, n+m+s-1$. Так как $\displaystyle\sum_{\gamma=j+\sigma}\Big(\displaystyle\sum_{j=k+l}a_kb_l\Big)c_\sigma = \displaystyle\sum_{\gamma=k+l+\sigma}a_k(b_lc_\sigma)$.
    
    Пусть $l'=l+\sigma\Rightarrow\displaystyle\sum_{\gamma=k+l+\sigma}a_k(b_lc_\sigma) =\displaystyle\sum_{\gamma=k+l'}a_k\Big(\displaystyle\sum_{l'=l+\sigma}b_lc_\sigma\Big)$ 
    $\underset{(\ref{1})}{\Rightarrow} (P_n(x)\cdot Q_m(x) )R_s(x)=P_n(x)\cdot(Q_m(x)\cdot R_s(x))~-~$ ассоциативно.
    \item Дистрибутивность аналогично (везде используются свойства одночленов)\par Таким образом так построенное множество многочленов от $x$ над $A$ будет ассоциативным и коммутативным кольцом $A(x)$. Роль единицы в $A(x)$ играет единица из $A$.\par
    При построении кольца многочленов вместо $x$ положим $p=\frac{d}{dx}~-~$ оператор дифференцирования, который действует на множестве бесконечно дифференцируемых комплекснозначных функций. $p\cdot f(x)=p(f(x))=\frac{df}{dx}=f', \;\;p^2(f)=f'', \dots, p^n(f)=f^{(n)};$ Справедлива формула $p^s\cdot p^m(f)=p^s\cdot (p^m(f))=p^s\cdot (f^{(m)})=f^{(m+s)}=p^{m+s}(f)$.\par
    По определению, множество бесконечно дифференцируемых комплекснозначных функций $\varPhi$ является кольцом, содержащим поле $\mathds{C}$. В качестве элементов кольца $A$ будем брать  числа из $\mathds{C}$.
    Роль операторного одночлена в таком случае будет играть $ap^m, \;\;a\in\mathds{C};\;\;ap^m=p^ma, $ так как $ap^m(f)=af^{(m)}=f^{(m)}\cdot a = p^m(f)\cdot a;$ По определению положим $ap^0=a, $ что корректно, так как $ap^0f=ap^0(f)=af=a\cdot f=a(f).$ Приведение подобных слагаемых для одночленов определим как $ap^m+bp^m=(a+b)p^m$, поскольку $(ap^{(m)})(f)+bp^{(m)})(f)=af^{(m)}+bf^{(m)}=(a+b)f^{(m)}=((a+b)p^m)(f)$.\par
    Аналогично вводим выражение, состоящее из нескольких операторных одночленов, соединенных знаком $+$, называемое операторным многочленом от $p$ с коэффициентом из $\mathds{C}$. Из свойств дифференцирования следует, что в общем виде можно записать $L_n(p)=a_0+a_1p+\dots+a_np^n$\par
    Абсолютно аналогично доказываем, что замена $x$ на $p$ дает множество операторных многочленов от $p$, которое будет кольцом из $\mathds{C}$
    \item Пусть $x \in \mathds{C}$. Значение многочлена $P_n(x)$ на $\mathds{C}$ определим как число $P_n(x) = a_0 + a_1 x + \dots + a_n x^{n} \in \mathds{C}$. \par
    Понятие значения многочлена можно обобщить на случай, когда $B$ является ассоциативным кольцом, содержащим кольцо $A$, в случае, когда элементы $A$ коммутируют с элементами из $B$.\par
    В таком случае можно определить степень элемента кольца $B$. Пусть $a\in B, \;\;a^1=a, \;\;a^2=a\cdot a, \dots, a^n=a^{n-1}\cdot a$
    \begin{theorem}
    $\forall k\in \mathbb{N}, \forall m \in \mathbb{N} \rightarrow a^k\cdot a^m= a^{k+m}$
    \end{theorem}
    \item Значение операторного многочлена $L_n(p) $ определим на коммутативном и ассоциативном кольце $\varPhi~-~$ бесконечно дифференцируемой комплекснозначной функцией от $x \in \mathbb{R}: \;\;f(x)$$$L_n(F)=L_n(p)(f)=a_0f+a_1f'+\dots+a_nf^{(n)}\in \varPhi$$
    \item Если $F(p)=L_n(p)+M_m(p)$ определим сумму на множеcтве дифф. операторов:$$F(p)=(a_0+b_0)f+(a_1+b_1)f'+\dots+c_sf=L_n(p)(f)+M_m(p)(f)\Rightarrow (L_n(p)+M_m(p))(f) =$$ 
    $$=(M_m(p)+L_m(p))(f)$$ коммутативно, ассоциативность аналогично.
    \item $(L_n(p) M_m(p))(f)=(a_0 b_0 p^0 + (a_0 b_1 + a_1 b_0)p+ \dots + \left(\displaystyle \sum_{j=k+l}a_kb_l \right)p^j+ \dots +a_n b_m p^{m+n})(f)=a_0 b_0 f +(a_0 b_1 + a_1 b_0)f'+ \dots +(\displaystyle \sum_{j=k+l} a_k b_l)f^{(j)}+ \dots +a_n b_m f^{(n+m)} = (a_0 p^0 + a_1 p+ \dots + a_n p^n) \cdot (b_0 f + b_1 f'+ \dots + b_m f^{(m)})=\;L_n(p) \cdot (M_m(f))~-~$ определение действия произведения операторов на множестве $\varPhi$. Так как $a_0b_0 f+(a_0b_1+a_1b_0)f'+\dots+ \left(\displaystyle \sum_{j=k+l}a_kb_l \right)f^{(j)}+ \dots + a_n b_m f^{(m+n)} = M_m(p) \cdot (a_0 f+a_1 f' + \dots a_n f^{(n)})\Rightarrow (L_n(p)\cdot M_m(p))=(M_m(p)\cdot L_n(p)) ~-~$ коммутативность.
    \item Покажем ассоциативность и дистрибутивность
    $$ (L_n(p)\cdot M_m(p))K_s(p)(f) = (L_n(p)\cdot M_m(p))(K_s(p)(f)) = $$ $$ = L_n(p)(M_m(p)(K_s(p)(f)))= L_n(p)(Q_m(p)R_s(p))(f) $$
    
    ассоциативность. $$(L_n(p)+M_m(p))K_s(p)(f)=L_n(p)(K_s(p)(f))+M_m(p)(K_s(p)(f))= $$ 
    $$ = (L_n(p)K_s(p))(f)+(M_m(p)K_s(p))(f)$$дистрибутивность $\cdot $ и $+$.\par
    Таким образом, множество значений операторных многочленов является кольцом, которое содержится в $\varPhi$.
    \item Если для $P_n(x)$ и $Q_m(x)$ из $A(x)\;\;\;\exists R_s(x)\in A(x): \;\;P_n(x)=Q_m(x)\cdot R_s(x), $ то говорят, что $P_n(x)$ делится на $Q_m(x)$.
    \begin{theorem}
    $$P_n(x) = a_n x^n + a_{n-1} x^{n-1} + \dots + a_0 \in A(x), \;c \in A \Rightarrow \exists ! Q_m(x),  r\in \mathds{C}:P_n(x)=(x-c)Q_m(x)+r$$
    \end{theorem}
    \begin{theorem}
    (Безу) $P_n(x)$ делится на $x-c\Leftrightarrow P_n(c)=0$.
    \end{theorem}
    \begin{theorem}
    Если кольцо $A$ является областью целостности, то число корней $P_n(x)$ не превосходит $n$.
    \end{theorem}
    \begin{theorem}
    Основная теорема алгебры \par
    Любой многочлен $P_n(x)$ над $\mathds{C}$ имеет хотя бы один корень.
    \end{theorem}
    \begin{proposition}
    Из $3$ и $5$ теоремы \begin{equation}
        \forall P_n(x)\rightarrow P_n(x)=a_n(x-c_1)^{l_1}\cdot \dots
    \cdot (x-c_k)^{l_k} 
    \end{equation}
    \end{proposition}
    \item Взаимнооднозначное соответствие $\varphi$ кольца $K$ на кольцо $K'$ называется изоморфизмом, если $\varphi : K \rightarrow K',\;\;a,b\in K \rightarrow$
    \begin{equation}
        \label{3}
        \varphi(a+b)=\varphi(a)+\varphi(b)\;\;\;\;\;\;\;\varphi(a\cdot b)=\varphi(a)\cdot\varphi(b)
    \end{equation}
    Причем знаки $+$ и $\cdot$ внутри $\varphi$ относятся у кольцу $K$, а знаки $+$ и $\cdot$ снаружи $\varphi$ относятся к $K'$\par
    Из $\eqref{3}$ следует, что образом нуля кольца $K$ будет нуль $K'$: $\varphi(a)=a'\in K'$ и $\varphi(0)=c',\;\;\varphi(a)=a'=\varphi(a+o) = \varphi(a)+\varphi(0)=a'+c'\Rightarrow c'=0$\par
    Если кольцо $K$ имеет единицу, то $\varphi(1)$ будет единицей кольца $K'$: $\varphi(a)=a'=\varphi(1\cdot a )= \varphi(1)\cdot \varphi(a) = \varphi(1) a'\Rightarrow\varphi(1)~-~$ единица $K'$.
    \item Обратное отображение $\varphi^{-1}$ кольца $K'$ на $K$ существует и будет изоморфно.\par
    Рассмотрим отображение $\varphi,$ которое множеству значений $P_n(x)$ над $\mathds{C}$ ставит в соответствие множество значений $L_n(p)$ на множестве бесконечно дифференцируемых комплекснозначных функций $\varPhi$ по принципу:
    $$\varphi(P_n(z))=\varphi(a_nz^n+a_{n-1}z^{n-1}+\dots+a_0z^0)=L_n(p)(f)=a_nf^{(n)}+a_{n-1}f^{(n-1)}+\dots+a_0f;$$
    Покажем, что отображение является изоморфизмом.\par Отображение взаимнооднозначно по построению.
    $$\varphi(P_n(z)+Q_m(z))=\varphi(a_0+a_1z+\dots+a_nz^n+b_0+b_1z+\dots+b_mz^m)=$$
    $$=\varphi(a_0+b_0+(a_1+b_1)z+\dots+(a_s+b_s)z^s)=$$
    $$=(a_0+b_0+(a_1+b_1)p+\dots+(a_s+b_s)p^s)(f)=(L_n(p)+L_m(p))(f)$$
    $$\varphi(P_n(z)\cdot Q_m(z))=\varphi(a_0b_0+\dots+ \displaystyle\sum_{j=k+l}a_kb_lz^j+\dots+a_nb_mz^{m+n})=$$
    $$=(a_0b_0p+\dots+\displaystyle\sum_{j=k+l}a_kb_lp^j+\dots+a_nb_mp^{m+n})(f)=L_n(p)\cdot Q_m(p)(f)$$
    Таким образом, $\varphi~-~$изоморфизм. Тогда из $(\ref{3})$: $$\varphi(P_n(x))=\varphi(a_n(z-c_1)^{l_1}\cdot ...\cdot (z-c_k)^{l_k})= L_n(p)(f)=a_n\cdot (p-c_1)^{l_1}\cdot ... \cdot я(p-c_k)^{l_k}(f)$$
    В итоге  $L_n(p)=a_n\cdot (p-c_1)^{l_1}\cdot ... \cdot (p-c_k)^{l_k},$ где $c_1, \dots, c_k ~-~$ корни $P_n(z)$.
    
\end{itemize}
\subsection{Линейные уравнения с поcтоянными коэффициентами}
Рассмотрим ДУ вида: $a_n y^{(n)}+a_{n-1}y^{(n-1)} + \cdots +a_0y=0, \;\;\;a_n \neq 0, $ где $a_i=const\;\;\forall i=\overline{1, n}.$ Через введенный ранее дифференциальный оператор $L_n(p)=a_np^n+ \dots + a_0 p^0$ уравнение записывается в виде \begin{equation}\tag{2.1}
\label{2.1}
    L_n(p)(y(x))=0
\end{equation}
Было доказано, что $L_n(p)$ является изоморфизмом характеристического многочлена $(\ref{2.1})$: $P_n(\lambda)=a_n\lambda^n+\dots+a_0=a_n(\lambda-\lambda_1)^{l1}\cdot ... \cdot (\lambda-\lambda_{k})^{l_k}$ и поэтому для $L_n(p)$ справедливо разложение \begin{equation}
    \tag{2.2}
    \label{2.2}
    L_n(p)=a_n(p-\lambda_1)^{l1}\cdot ... \cdot (p-\lambda_{k})^{l_k}, \;\;\;p=\frac{d}{dx}
\end{equation}
Задача: найти ФСР $(\ref{2.1})$. Из записи $L_n(p)$ ясно, что решением$ (\ref{2.1})$ будут функции из $\varPhi$, котрые являются корнями $L_n(p)$.
\begin{lemma}
Для любой $n$ раз дифференцируемой на промежутке функции $f(x)$, $\lambda\in\mathds{C}$ выполняется "формула сдвига" \begin{equation}
    \tag{2.3}
    \label{2.3}
    L_n(p)(e^{\lambda x}f)=e^{\lambda x}\cdot L_n(p+\lambda)(f)
\end{equation}  
\end{lemma}
\begin{proof}
Докажем по индукции. База $n=1$: $$L_1(p)(e^{\lambda x}f)=(a_1p^1+a_0)(e^{\lambda x}f)=e^{\lambda x}(a_0f+a_1(\lambda f+ f'))=e^{\lambda x}(a_1(p+\lambda)+a_0)(f)=e^{\lambda x}L_1(p+\lambda)(f)$$
Пусть $(\ref{2.3})$ справедлива для $k=n-1$, то есть $L_{n-1}(p)(e^{\lambda x} f)=e^{\lambda x} L_{n-1}(p+\lambda)(f)$\par
Обозначим $L_n(p)=(p-\lambda_1)\cdot L_{n-1}(p),$ тогда по формуле $(\ref{2.2}):$
$$L_n(p)=a_n(p-\lambda_1)\cdot (p-\lambda_1)^{l_1-1}\cdot ...\cdot (p-\lambda_m)^{l_m} \cdot ... \cdot (p-\lambda_k)^{l_k}=L_1(p)\cdot L_{n-1}(p)=L_{n-1}(p)\cdot L_1(p)$$
Тогда $L_n(p)(e^{\lambda x }f) = L_{n-1}(p) \cdot L_1(p)(e^{\lambda x}f(x))=L_{n-1}(p)(L_1(p)(e^{\lambda x}f))\underset{\text{база}}{=} L_{n-1}(p)(e^{\lambda x}\cdot L_1(p+\lambda) (f))$\par
Обозначим через $g(x)=L_1(p+\lambda)(f(x)),$ имеем:

$$L_n(p)(e^{\lambda x}f) = L_{n-1}(p)(e^{\lambda x}g(x))\underset{\text{индукция}}{=}e^{\lambda x}L_{n-1}(p+\lambda)(g) = $$ $$ = e^{\lambda x}L_{n-1}(p+\lambda)(L_1(p+\lambda)(f))=e^{\lambda x}(L_{n-1}(p+\lambda)\cdot L_{1}(p+\lambda))(f)=e^{\lambda x}\cdot L_n(p+\lambda)(f(x))$$

\end{proof}
\begin{theorem}
Если $\lambda_m$ является корнем $L_n(\lambda)$ кратности $l_m,$ то функции $e^{\lambda_m x}, xe^{\lambda_m x}, \dots$, $x^{l_m-1}e^{\lambda_m x}$ являются решениями $(\ref{2.1})$
\end{theorem}
\begin{proof}
Из коммутативности и ассоциативности кольца операторных многочленов и формулы $(\ref{2.3})$: $L_n(p)=a_n(p-\lambda_1)^{l_1}\cdot ... \cdot (p-\lambda_m)^{l_m}\cdot ... \cdot (p-\lambda_k)^{l_k}=L_{n-l_m}(p)(p-\lambda_m)^{l_m}$\par
Воспользуемся формулой сдвига для $x^se^{\lambda_m x}:$
$$L_n(p)(x^se^{\lambda_m x})=e^{\lambda_m x}\cdot L_{n-l_m}(p+\lambda_m) \cdot p^{l_m}(x^s)=$$
$$=e^{\lambda_m x}\cdot L_{n-l_m}(p+\lambda_m)(x^s)^{(l_m)}=\left\{\begin{gathered}
    0, \;\;\forall s\leqslant l_m-1\hfill
    \\
    e^{\lambda_m x}\cdot P_{n-l_m}(x), s\geqslant l_m \hfill
    \\
\end{gathered}
\right.,$$
где $P_{n-l_m}$ многочлен степени не ниже $s-l_m$.\par
Таким образом $x^se^{\lambda_{m} x}, \;\;\;s=\overline{0, l_{m}-1}$ являются корнями $L_n(p),$ а значит и решениями$(\ref{2.1})$
\end{proof}
Из доказанной теоремы следует:
\par
Все функции из набора:
\begin{equation}
\tag{2.4}
\label{2.4}
    \Bigg\{\{e^{\lambda_1 x}, \dots, x^{l_1-1}e^{\lambda_1 x}\}, \dots, \{e^{\lambda_mx}, \dots, x^{l_m-1 }e^{\lambda_mx}\}, \dots, \{e^{\lambda_k x}, \dots, x^{l_k-1}e^{\lambda_k x}\}\Bigg\}
\end{equation}
будут решениями $(\ref{2.1})$. Всего таких функций $n$ штук. Докажем линейную независимость систем функций $(\ref{2.4})$.
\begin{lemma}\label{lem1}
Система $1, x, \dots, x^m$ линейно независима. 
\end{lemma}

\begin{proof}
Рассмотрим линейную комбинацию функций $C_0+C_1x+\dots+C_nx^n=0$\par От противного: пусть $\exists C_0, \dots, C_n:\;\;\;\displaystyle\sum_{i=0}^n C_i^2\neq0: C_0+C_1x+\dots+C_nx^n=0\;\;\;\forall x $\par Так как у многочлена степени $n$ не более чем $n$ нулей, то получаем противоречие.
\end{proof}
\begin{theorem}
Система функций $P_{n1}(x)e^{\lambda_1 x},\dots, P_{ns}(x)e^{\lambda_s x}, $ где $P_{ni}(x)$ является многочленом степени $n_i$,  а все $\lambda_i \in \mathds{C}$ разные,  является ЛНЗ.
\end{theorem}
\begin{proof}
Выражение $P_n(x)e^{\lambda x}~-~$квазимногочлен степени $n$, $\lambda\in\mathds{C}$, коэффициенты $P_n(
x)\in\mathds{C}$
Рассмотрим $(P_n(x)e^{\lambda x})'=\lambda\cdot P_n(x)e^{\lambda x}+e^{\lambda x}\overline{P}_{n-1}(x)=e^{\lambda x }(\lambda P_n(x)+\overline{P}_{n-1}(x))= \widetilde{P}_n(x)\cdot e^{x\cdot \lambda}$.\par То есть, если будем дифференцировать квазимногочлены степени $n$, то останемся в множестве квазимногочленов степени $n$.\par
Докажем по индукции. База $n=1~-~$ выполнена по Лемме ($\ref{lem1}$). Пусть выполнено для $n=s-1:$ система из $s-1$ квазимногочленов является ЛНЗ системой: $P_{n_1}(x)e^{\lambda_1x}, \dots, P_{n_{s-1}}e^{\lambda_{s-1}\cdot x}~-~$ ЛНЗ.\par
Для $n$. От противного: пусть система $P_{n_1}(x)e^{\lambda_1x}, \dots, P_{n_{s-1}}e^{\lambda_{s-1}x}, P_{n_s}(x)e^{\lambda_s x}$ является линейно зависимой, тогда $\exists C_1, \dots, C_l, \dots, C_s:$
\begin{equation}
    \tag{2.5}
    \label{2.5}
    C_1P_{n_1}(x)e^{\lambda_1x}+C_2P_{n_2}(x)e^{\lambda_2x}+\dots+C_lP_{n_l}(x)e^{\lambda_lx}+\dots+C_sP_{n_s}(x)e^{\lambda_sx}=0
\end{equation}
и хотя бы одна константа, например $C_l\neq0$ Из $(\ref{2.5})$, перенося $C_l$ вправо и деля на $C_le^{\lambda_lx}$ получаем:
$$\overline{C_1}P_{n_1}(x)e^{\omega_1x}+\dots+\overline{C_{l-1}}P_{n_{l-1}}(x)e^{\omega_{l-1}x}+\overline{C_{l+1}}P_{n_{l+1}}(x)e^{\omega_{l+1}x}+\dots+\overline{C_s}P_{n_s}(x)e^{\omega_sx}=-P_{n_l}(x)$$ где $\overline{C_i}=\frac{C_i}{C_l\neq0}, \omega_i=\lambda_i-\lambda_l$.\par Продифференцируем $n_{l}+1$ раз последнее тождество. Перенумеровав $s-1$ слагаемое в левой части получим $\widetilde{C}_1\cdot \widetilde{P}_{n_1}(x)e^{\omega_1x}+\dots+\widetilde{C}_{s-1} \cdot \widetilde{P}_{n_{s-1}}(x)e^{\omega_{s-1}x}=0$.\par По определению индукции последнее равенство возможно, только если  все $\widetilde{C}_i=0, \;\;\Rightarrow C_i =0, i=1, \dots, l-1, l+1, \dots, s\underset{(\ref{2.5})}{\Rightarrow} C_l=0~-~$ противоречие предположению индукции о линейной независимости системы $P_{n_1}(x)e^{\lambda_1x}, \dots, P_{n_{s-1}}(x)e^{\lambda_{s-1}x}$.
\end{proof}
Таким образом ФСР дифференциального уравнения $(\ref{2.1})$ будет состоять из функций набора
$$ \Bigg\{\{e^{\lambda_1 x}, \dots , x^{l_1-1}e^{\lambda_1 x}\}, \dots, \{e^{\lambda_mx}, \dots, x^{l_m-1 }e^{\lambda_mx}\}, \dots, \{e^{\lambda_k x}, \dots, x^{l_k-1}e^{\lambda_k x}\}\Bigg\},$$
где $\lambda_1, \dots, \lambda_m, \dots, \lambda_k~-~$ корни характеристического многочлена $P_n(\lambda)$ кратности $l_1, \dots$, $l_m, \dots, l_k$.\par
Общее решение $(\ref{2.1})$ будет иметь вид
\begin{equation}
    \tag{2.6}
    \label{2.6}
    y_0=e^{\lambda_1x}\Big(\displaystyle\sum_{m=0}^{l_1-1}C_m^1x^m\Big)+\dots+e^{\lambda_kx}\Big(\displaystyle\sum_{m=0}^{l_k-1}C_m^kx^m\Big)
\end{equation}
Фигурирующие в $(\ref{2.6})$ константы $C_i^j$, вообще говоря, могут быть комплексными, если корни $P_n(\lambda)$ являются комплекснозначными. Если изначально ставится задача $~-~$ найти решение ДУ во множестве действительных функций действительного переменного, то в случае комплексных корней возникает задача выделить из множества комплексных решений действительное. Это осуществимо, так как коэффициенты $P_n(\lambda)$ являются действительными числами.\par
Пусть $\lambda_m=\alpha+\beta i ~-~$ корень характеристического многочлена кратности $l$. Ему соответствуют $\varphi_m^i=x^ie^{\alpha x}(\cos{\beta x}+i\sin{\beta x}) $
\par

Комплексные корни идут парами, поэтому $\lambda_m=\alpha-\beta i$ тоже корень, и ему соответствует 
$\overline{\varphi}_m^i=x^ie^{\alpha x}(\cos{\beta x}-i\sin{\beta x})\;\;\;\varphi^i_m, \overline{\varphi}_m^i~-~$ ЛНЗ, $i=\overline{0, l-1}$\par
Рассмотрим функции $$\Psi_m^i =\frac{\varphi_m^i+\overline{\varphi}_m^i}{2}=e^{\alpha x}\cdot x^i\cos{\beta x}= Re(\varphi^i_m)$$
$$\chi_m^i =\frac{\varphi_m^i-\overline{\varphi}_m^i}{2i}=e^{\alpha x}\cdot x^i\sin{\beta x}= Im(\varphi^i_m)$$
Так как любая суперпозиция решений $(\ref{2.1})$ в силу его линейности тоже является решением, то $\chi_m^i$ и $\Psi_m^i$ являются линейно независимыми и действительными решениями $(\ref{2.1})$. Таким образом, чтобы получить действительную ФСР, необходимо все $\varphi_m^i$ и $\overline{\varphi_m^i}, \;\;\; i=\overline{0, l_m-1}, \; m=\overline{1, k} $, отвечающих паре комплексных корней характеристического многочлена $\alpha \pm i\beta $ кратности $l$, заменить на вещественные $Re(\varphi_m^i)$ и $(\varphi_m^i)$. Если считать, что $\lambda_i=\alpha_i\pm i\beta_i~-~$ корень $P_n(\lambda)$ кратности $l_i$, то общее решение $(\ref{2.1})$ имеет вид:

\begin{equation}
    \tag{2.7}
    \label{2.7}
    y_0=e^{\alpha_1x}\Big(\displaystyle\sum_{j=0}^{l_1-1}x^j(A^1_j\cos{\beta_1x}+B^1_j\sin{\beta_1x})\Big)+\dots +e^{\alpha_kx}\Big(\displaystyle\sum_{j=0}^{l_k-1}x^j(A_j^k\cos{\beta_kx}+B_j^k\sin{\beta_kx})\Big)
\end{equation}
\subsection{Неоднородные линейные уравнения}
Рассмотрим уравнение вида: $L_n(p)(y(x))=f(x)$.
\begin{lemma}
Пусть неоднородность имеет вид $f(x)=\displaystyle\sum_{k=1}^mf_k(x)$ и $y_k^s(x)~-~$ частное решение\par $L_n(p)(y(x))=f_k(x), \;\;\;k=\overline{1, m},$ то есть $L_n(p)(y^s_k(x))=f_k(x)$\par
Тогда частное решение уравнения имеет вид $y^s(x)=\displaystyle\sum_{k=1}^my_k^s(x).$
\end{lemma}
\begin{proof}
$L_n(p)\Big(\displaystyle\sum_{k=1}^my_k^s(x)\Big)\underset{\text{линейность L}}{=} \displaystyle\sum_{k=1}^mL_n(p)(y^s_k(x))=\displaystyle\sum_{k=1}^mf_k(x)=f(x)$
\end{proof}
\begin{remark}
Утверждение леммы остается верным и в случае переменных коэффициентов в $L_n(p).$
\end{remark}
\begin{definition}
Пусть $f(x)=\displaystyle\sum_{i=1}^nP_{n_i}(x)e^{\lambda x},$ где $P_{n_i}~-~$ многочлен степени $n_i$ с комплексными коэффициентами, $\lambda \in \mathds{C}$. Тогда $f(x)$ называется квазимногочленом.
\end{definition}
Рассмотрим ДУ:
\begin{equation}
    \tag{1}
    \label{1}
    y^{(n)}+a_1y^{(n-1)}+\dots+ a_{n-1} y' + a_ny=L_n(p)(y(x))=(b_kx^k+b_{k-1}x^{k-1}+\dots + b_0)e^{\lambda x}=P_k(x)e^{\lambda x}
\end{equation}
\begin{theorem}
Частное решение $(\ref{1})$ можно найти в виде \begin{equation}
    \tag{2}
    \label{2}
    y^s(x)=x^r(C_kx^k+C_{k-1}x^{k-1}+\dots + C_0)e^{\lambda x}
\end{equation}
где $r=l_m,$ если $\lambda=\lambda_m, \;\;m=\overline{1, s}~-~$ корень $P_n(\lambda)$\par
$r=0,$ если $\lambda\neq \lambda_m;$ Неопределенные константы $C_k\dots, C_0$ находятся из системы с треугольной матрицей.
\end{theorem}
\begin{proof}
\begin{itemize}
    \item $\lambda_m=\lambda$ \par
    Подставим $(\ref{2})$  в $(\ref{1})$ и воспользуемся формулой сдвига.\par
    $y^s(x)=x^r(C_kx^k+C_{k-1}x^{k-1}+\dots + C_0)e^{\lambda x}$\par
    Оператор примет вид:
    $$L_n(p)(y^s(x))=(a_n(p-\lambda_1)^{l_1}\cdot...\cdot (p-\lambda_s)^{l_s})(y^s(x))=
    L_{n-l_m}(p)\cdot \underset{\text{(к оперетору)}}{(p-\lambda_m)^{l_m}}(y^s(x))\underset{\text{формула сдвига}}{=}
    $$$$=e^{\lambda_m x}L_{n-l_m}(p+\lambda_m)\frac{d^{l_m}}{dx^{l_m}}(C_kx^{r+k}+C_{k-1}x^{r+k-1}+\dots+C_0x^r)$$
    Уравнение в таком виде имеет вид:
    $$e^{\lambda x}L_{n-l_m}(p+\lambda_m)\frac{d^{l_m}}{dx^{l_m}}(C_kx^{r+k}+C_{k-1}^{r+k-1}+\dots+C_0x^r)\equiv e^{\lambda_m x}(b_kx^k+b_{k-1}x^{k-1}+\dots+b_0)$$
    где $L_{n-l_m}(p+\lambda_m)=a_0(p+\lambda_m)^0+\dots+ a_{n-l_m}(p+\lambda_m)^{n-l_m}=d_0p^0+\dots+ d_{n-l_m}p^{n-l_m}$\par
    Сократим на $e^{\lambda_m x}$ и выполним дифференцирование $\frac{d^{l_m}}{dx^{l_m}}$ с учетом того, что $r=l_m$
    $$(d_0p^0+\dots + d_{n-l_m}p^{n-l_m})(A_kC_kx^k+A_{k-1}C_{k-1}x^{k-1}+\dots)=$$
    $$=A_kC_kd_0x^k+(kA_kC_kd_1+A_{k-1}C_{k-1}d_0)x^{k-1}+\dots \equiv$$$$\equiv b_kx^k+b_{k-1}x^{k-1}+\dots$$ где $A_k=(k+l_m)(k+l_m-1)\cdot ... \cdot (k+1)$\par
    Приравнивая коэффициенты при одинаковых степенях $x$ и получим систему 
    \begin{equation}
       \text{система с треугольной матрицей}
        \begin{cases}
            A_kC_kd_0=b_k\\
            A_{k-1}C_{k-1}d_0+kA_kC_kd_1=b_{k-1}\\
            \dots
        \end{cases}
    \end{equation}
    \item $\lambda\neq \lambda_m$
    $$y^s=e^{\lambda x}(C_kx^k+C_{k-1}x^{k-1}+\dots+ C_0)$$
    После формулы сдвига $e^{\lambda x}L_n(p+\lambda)(f)\Rightarrow$
    $$L_n(p+\lambda_m)=(a_0(p+\lambda_m)^0+\dots+a_n(p+\lambda_m)^n)=d_0p^0+d_1p+\dots+d_np^n\Rightarrow$$
    уравнение примет вид:
    $$e^{\lambda x}(d_0p^0+d_1p+\dots+d_np^n)(C_kx^k+C_{k-1}x^{k-1}+\dots+C_0)\equiv (b_kx^k+b_{k-1}x^{k-1}+\dots +b_0)e^{\lambda x}\Rightarrow$$
    $$C_kd_0x^k+(kC_kd_1+C_{k-1}d_0)x^{k-1}+\dots\equiv b_kx^k+b_{k-1}x^{k-1}+\dots$$
    После приравнивая коэффициентов при одинаковых степенях $x$:
    \begin{equation}
    \text{Система с треугольной матрицей}
        \begin{cases}
            C_kd_0=b_k\\
            C_{k-1}d_0+kC_kd_1=b_{k-1}\\
            \dots
        \end{cases}
    \end{equation}
\end{itemize}
\end{proof}
\subsection{Уравнение Эйлера}
\begin{remark}
Источник: В. М. Ипатова, О. А. Пыркова, В. Н. Седов "Дифференциальные уравнения. Методы решений"
\end{remark}
\begin{definition}
Уравнением Эйлера называется линейное дифференциальное уравнение с переменными 
коэффициентам вида $a_k(x)=b_kx^{n-k}, \;\;k=\overline{0, n},$ где $b_0, b_1, \dots, b_n~-~$заданные числа, причем $b_0\neq 0:$
\begin{equation}
\tag{3.1}
\label{3.1}
    b_0x^ny^{(n)}+b_1x^{n-1}y^{(n-1)}+\dots+b_{n-1}xy'+b_ny=f(x)
\end{equation}
\end{definition}
Заменой $x=e^t\;\;(t=ln x)\;\;\;(\ref{3.1})$ сводится к линейному дифференциальному уравнению с постояннными коэффициентами. Действительно, $$\frac{dy}{dx}=\frac{dy}{dt}\cdot\frac{dt}{dx}=\frac{1}{x}\cdot \frac{dy}{dt}=e^{-t}\frac{dy}{dt}, \;\;\;\frac{d^2y}{dx^2}=\frac{d}{dx}\Big(\frac{dy}{dx}\Big)=e^{-t}\frac{d}{dt}\Big(e^{-t}\frac{dy}{dt}\Big)=e^{-2t}\Big(\frac{d^2y}{dt^2}-\frac{dy}{dt}\Big)$$
Допустим, что $k-$я производная имеет вид $$\frac{d^ky}{dx^k}=e^{-kt}\Big(\frac{d^ky}{dt^k}+\alpha_1\frac{d^ky}{dt^k}+\dots+\alpha_{k-1}\frac{dy}{dt}\Big)=\frac{1}{x^k}\Big(\frac{d^ky}{dt^k}+\alpha_1\frac{d^{k-1}y}{dt^{k-1}}+\dots+\alpha_{k-1}\frac{dy}{dt}\Big)$$ где $\alpha_1, \alpha_2, \dots, \alpha_{k-1}~-~\text{постоянные}$
Тогда $(k+1)-$я производная будет равна \begin{equation}
    \frac{d^{k+1}y}{dx^{k+1}}=\frac{d}{dx}\Big(\frac{d^ky}{dx^k}\Big)=e^{-t}\frac{d}{dt}\Big(\frac{d^ky}{dx^k}\Big)=e^{-(k+1)t}\Big(\frac{d^{k+1}y}{dt^{k+1}}+(\alpha_1-k)\frac{d^ky}{dt^k}+\dots+k\alpha_{k-1}\frac{dy}{dt}\Big) =
\end{equation}
    
    \begin{equation}
= \frac{1}{x^{k+1}} \Big(\frac{d^{k+1} y}{dt^{k+1}}+(\alpha_1 - k) \frac{d^k y}{dt^k} + \dots + k \alpha_{k-1} \frac{dy}{dt}\Big)
\end{equation}
Так как в преобразованном уравнении, в случае отсутствия
кратных корней характеристического уравнения, решения имеют
вид $y=e^{\lambda t},$ следовательно, в исходном уравнении они имеют вид $y=x^\lambda$. Поэтому можно непосредственно подставить его в уравнение Эйлера $(\ref{3.1})$. Поскольку $x^k\frac{d^kx^\lambda}{dx^k}=\lambda(\lambda-1)\dots(\lambda-k+1)x^\lambda$ при $k\leqslant\lambda$, то характеристическое уравнение имеет вид \begin{equation}
\tag{3.2}
\label{3.2}
    b_0\lambda(\lambda-1)\dots(\lambda-n+1)+\dots+b_{n-2}\lambda(\lambda-1)+b_{n-1}\lambda+b_n=0
\end{equation}
Каждому простому корню $\lambda$ уравнения $(\ref{3.2})$ соответствует частное решение однородного уравнения Эйлера $x^\lambda$; каждому действительному корню $\lambda$ кратности $l\;\;(l\geqslant2)$ соответсвует $l$ линейно
независимых частных решений однородного уравнения Эйлера $x^\lambda, x^\lambda\ln{x}, \dots, x^\lambda(\ln{x})^{l-1}$. В случае невещественных корней $\lambda$ надо
учитывать, что $x^{i\beta}=e^{i\beta\ln{x}}$, таким образом паре комплексно сопряженных
корней $\alpha\pm i\beta$ уравнения $(\ref{3.2})$ будут соответствовать два решения
однородного уравнения Эйлера $x^\alpha\cos{(\beta\ln x)}$ и $x^\alpha\sin{(\beta\ln{x})}$.
