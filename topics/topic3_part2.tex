\subsection{Матричная экспонента, ее свойства и применение к решению нормальных линейных систем}

\subsubsection{Матричная экспонента}
Необходимо решить ОЛДУ вида:

\begin{equation}
  	\frac{d\overrightarrow{x}}{dt} = A\overrightarrow{x},\ \overrightarrow{x}(t_0) = \overrightarrow{x_0},
  	\label{Issue5_1}
\end{equation}

Матрица $A = ||a_j^i||,\ a_j^i \in \mathds{R},\ i, j = \overline{1,n}$. При доказательстве теоремы существования и единственности была использована следующая итерационная процедура:

\begin{equation*}
	\overrightarrow{x^{0}} = \overrightarrow{x_{0}}, \; \overrightarrow{x^{1}}(t) = \overrightarrow{x_{0}} + \int \limits_{t_0}^{t} A \overrightarrow{x^{0}} ds, \; \dots, \overrightarrow{x^{n}}(t) = \overrightarrow{x_{0}} + \int \limits_{t_0}^{t} A \overrightarrow{x^{n - 1}} ds
\end{equation*}

Отсюда можно получить:

\begin{equation*}
\begin{gathered}
         \overrightarrow{x^{0}} = E\overrightarrow{x_0},\ \overrightarrow{x^1} = E\overrightarrow{x_0} + \frac{t-t_0}{1!}A\overrightarrow{x_0} = \left(E + \frac{t-t_0}{1!}A\right) \overrightarrow{x_0}, \\  
         \overrightarrow{x^n} = \left(E + \frac{t-t_0}{1!}A +\ \dots\ + \frac{(t-t_0)^n}{n!}A^n\right)\overrightarrow{x_0},	 
\end{gathered}
\end{equation*}

При доказательстве теоремы существования и единственности Коши было получено, что решением является

\begin{equation*}
   	\overrightarrow{x} = \left(E + \frac{t-t_0}{1!}A +\ \dots\ + \frac{(t-t_0)^n}{n!}A^n+\ \dots \right)\overrightarrow{x_0} = \left(\sum\limits_{n = 0}^{\infty} \frac{(t-t_0)^n}{n!}A^n\right)\overrightarrow{x_0},	 
\end{equation*}

при условии, что $A^0 = E.$

\begin{definition}
	Матричной экспонентой называют следующий степенной ряд:
	\begin{equation*}
		e^{(t-t_0)A} = E + \frac{t-t_0}{1!}A +\ \dots\ + \frac{(t-t_0)^n}{n!}A^n+\ \dots\ = \sum\limits_{n = 0}^{\infty} \frac{(t-t_0)^n}{n!}A^n.
	\end{equation*}
\end{definition}

\subsubsection{Свойства матричной экспоненты}

Матричная экпонента это квадратная матрица, по размерам аналогична матрице $A$, и каждый элемент этой матрицы представляет из себя степенной ряд с радиусом сходимости $+\infty$.

\begin{enumerate}
	\item $e^{(t_1+t_2)A} = e^{t_1A}e^{t_2A} \Rightarrow e^{t_1A}e^{t_2A} = e^{t_2A}e^{t_1A}$ (коммутативность).
		
	\item $ e^{0 A} = E.$	
	
	\item $\left(e^{tA}\right)^{-1} = e^{-tA}.$
		
	\item $(e^{tA})^{'} = A e^{tA} = e^{tA}A.$

\end{enumerate}

\begin{proof}

Так как квадратные матрицы составляют определенное кольцо, то \\

\begin{enumerate}

	\item Рассматриваем $(\ref{Issue5_1})$, если $\overrightarrow{x}(t)$ -- решение этого ДУ, то $\overrightarrow{x}(t+t_0)$ тоже решение этого ДУ: пусть $u = t + t_0$, тогда

	\[ \frac{d\overrightarrow{x}(t+t_0)}{dt} = \frac{d\overrightarrow{x}}{du}\frac{du}{dt} = \frac{d\overrightarrow{x}}{du} = A\overrightarrow{x}(u) = A\overrightarrow{x}(t+t_0).\]

	Рассмотрим $(\ref{Issue5_1})$ с задачей Коши $\overrightarrow{x}(0) = \overrightarrow{x_0}$: система имеет решение

	\[ \overrightarrow{x}(t) =  e^{tA}\overrightarrow{x_0},\]
	\[ \overrightarrow{x}(t+t_0) = e^{(t+t_0)A}\overrightarrow{x_0}\text{ -- решение }\frac{d\overrightarrow{x}}{dt} = A\overrightarrow{x}. \]

	Рассмотрим тогда то же самое уравнение для функции $\overrightarrow{z}(t)$:

	\[ \frac{d\overrightarrow{z}}{dt} = A\overrightarrow{z}, \text{ с задачей Коши } \overrightarrow{z}(0) = e^{t_1A}\overrightarrow{x_0} \Rightarrow \overrightarrow{z}(t) = e^{tA} (e^{t_1A}\overrightarrow{x_0}) = (e^{tA}e^{t_1A})\overrightarrow{x_0}.\]

	Получаем:

	\[ \overrightarrow{x}(t_1) = e^{t_1A}\overrightarrow{x_0} = \overrightarrow{z}(0),\]

	из основной теоремы следует, что $\overrightarrow{x}(t+t_1) = \overrightarrow{z}(t)\ \forall t$.

	Тогда и получается основная формула:

	\[ \overrightarrow{x}(t+t_1) = e^{(t+t_1)A}\overrightarrow{x_0} = (e^{tA}e^{t_1A})\overrightarrow{x_0} = \overrightarrow{z}(t)\]

	\item $ e^{tA} = E + \frac{t-t_0}{1!}A +\ \dots\ + \frac{(t-t_0)^n}{n!}A^n+\ \dots $, если $t = 0$ :
	\[ e^{0A} = E + 0 + \dots = E\]

	\item $E = e^{0A} = e^{(t-t)A} = e^{tA}e^{-tA} = E \Rightarrow \left(e^{tA}\right)^{-1} = e^{-tA}.$

	\item Берем представление матричной экспоненты в виде степенного ряда, который можно дифференцировать, тогда получаем:

	\[ (e^{tA})^{'} = A + tA^2 +\ \dots\ +\frac{t^{n-1}}{(n-1)!} A^n +\ \dots = A\left(E + tA +\ \dots\ + \frac{t^{n-1}}{(n-1)!}A^{n-1}\right),\]

	\[ (e^{tA})^{'} = Ae^{tA} = e^{tA}A.\]

\end{enumerate}

\end{proof}

\textbf{Примечание.} Формула $e^{t(A+B)} = e^{tA}e^{tB}$ не имеет места, кроме случая, если $AB = BA$ (т.е. матрицы коммутативны).

\subsubsection{Применение к решению нормальных линейных систем}

\begin{theorem}

Пусть $S$ -- матрица перехода от исходного базиса к новому базису. Тогда в новой базисе $\overline{A} = S^{-1}AS$, или $A = S\overline{A}S^{-1}$. И главное:

\[e^{t\overline{A}} = S^{-1}e^{tA}S.\]

\end{theorem}

\begin{proof}
	\begin{equation*}
		\begin{gathered}
 e^{t\overline{A}} = \left(E + t\overline{A} +\ \dots\ + \frac{t^{n}}{n!}\overline{A}^{n} \dots \right) = \left(E + tS^{-1}AS +\ \dots\ + \frac{t^{n}}{n!}(S^{-1}AS)^{n}\right), \\
 (S^{-1}AS)^n = S^{-1}A^nS,\ SES^{-1} = SS^{-1} = E \\
 e^{t\overline{A}} = S^{-1}e^{tA}S.
		\end{gathered}
	\end{equation*}
\end{proof}

Для решения нормальных линейных систем методом матричной экспоненты мы будем находить собственные вектора.

Матрица $A$ в базисе из собственных векторов (если они соответствуют действительным собственным значениям) будет иметь диагональный вид. Произведение диагональной матрицы на диагональную -- диагональная. Тогда для случая без кратных корней:

\[ e^{tA} = E + t\cdot diag(\lambda_1, \dots, \lambda_n) + \dots + \frac{t^n}{n!}\cdot diag(\lambda_1^n, \dots , \lambda_n^n) + \dots\]

\[ e^{tA} = diag(e^{t\lambda_1}, \dots, e^{t\lambda_n}).\]

Если $\lambda$ -- корень кратности $l$, то матрица $C = A|_{L}$ (где L - корневое подпростанство размером $l$)  приводится к Жордановой клетке (диагональная матрица с одиннаковыми значениеми на диагонали и с единицами над главной диагональю).

\[ C = \lambda E + B \Rightarrow B = C - \lambda E. \]

\[ e^{tC} = e^{t(\lambda E + B)} = e^{t\lambda E}e^{tB},\ e^{t\lambda E} = diag(e^{t\lambda}, \dots, e^{t\lambda}), e^{tB} = E + tB + \dots + \frac{t^{l-1}}{(l-1)!}B^{l-1} + 0 \]

\begin{equation*}
	\text{тогда } e^{tC} = e^{\lambda t}
 	\begin{pmatrix}
            1\ \ t\ \ \ \ \  \dots\ \ \ \ \ \frac{t^{l-1}}{(l-1)!} \\
            0\ \ 1\ \ t\ \ \ \dots\ \ \ \ \  \frac{t^{l-2}}{(l-2)!} \\
            \dots \\
            0\ \ \ \ \ \ \ \ \ \dots\ \ \ \ \ \ \ \ \  1
    \end{pmatrix}
\end{equation*}

\textbf{Метод решения линейных неоднородных уравнений с постоянными коэффициентами} (матричный метод вариации постоянной)

\[ \frac{d\overrightarrow{x}}{dt} = A\overrightarrow{x} + \overrightarrow{f}(t),\ \ \text{ решение будем искать в виде } \ \overrightarrow{x}(t) = e^{tA}\overrightarrow{C}(t), \]

\[ \text{ тогда } Ae^{tA}\overrightarrow{C}(t) + e^{tA}\dot{\overrightarrow{C}}(t) = Ae^{tA}\overrightarrow{C} + \overrightarrow{f}(t),\]

\[ e^{tA}\dot{\overrightarrow{C}}(t) = \overrightarrow{f}(t)\ \Rightarrow \dot{\overrightarrow{C}}(t) = (e^{tA})^{-1}\overrightarrow{f}(t) = e^{-tA}\overrightarrow{f}(t). \]

\newpage
