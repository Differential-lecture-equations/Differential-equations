
\subsection{Групповые свойства автономных систем}

\begin{enumerate}
    \item $	\vec{\varphi}(t_1 + t_2, \vec{x}_0) = \varphi (t_2, \vec{\varphi}(t_1, \vec{x}_0)) = \vec{\varphi}(t_1, \vec{\varphi}(t_2, \vec{x}_0))$
    \begin{proof}
        \ \\
        Рассмотрим $ \vec{\varphi}(t, \vec{\varphi}(t_1, \vec{x}_0)) $ -- решение \eqref{avt_sys}; $\vec{\varphi}(t + t_1; \vec{x}_0)$ -- тоже решение \eqref{avt_sys}
        \[
            \left. 
                \begin{aligned}
                    &\vec{\varphi}(0, \vec{\varphi}(t_1, \vec{x}_0)) = \vec{\varphi}(t_1, 	\vec{x}_0) \\
                    &\vec{\varphi}(0 + t_1, \vec{x}_0) = \vec{\varphi}(t_1, \vec{x}_0) \\
                \end{aligned}
            \right\rbrace \stackrel{\text{основная 	теорема}}{=\joinrel=\joinrel=\joinrel=\joinrel=\joinrel=\joinrel\Rightarrow} \vec{\varphi}(t + t_1; \vec{x}_0) \equiv \vec{\varphi} (t, \vec{\varphi}(t_1, \vec{x}_0))
        \]
        Аналогично, $ \vec{\varphi}(t + t_2, \vec{x}_0) \equiv \vec{\varphi}(t, \vec{\varphi}(t_2, \vec{x}_0))$
    \end{proof}
    \item $ \vec{\varphi} (-t; \vec{\varphi}(t, \vec{x}_0)) = \vec{x}_0 $
    \begin{proof}
        \ \\
        Из 1):  $\ \vec{\varphi} (t + \tau, \vec{x}_0) = \vec{\varphi}(\tau, \vec{\varphi}(t, \vec{x}_0)) $. В силу произвольности $ \tau $ при $ \tau = -t $: $\\ \vec{\varphi}(-t, \vec{\varphi}(t, \vec{x}_0)) \stackrel{1)}{=} \vec{\varphi}(0, \vec{x}_0) = \vec{x}_0 $
    \end{proof}
\end{enumerate}

\subsection{Понятия фазового потока и фазового объема}

\begin{definition}
Рассматриваем давно привычную нам систему $\dfrac{d\vec{x}}{dt} = \vec{f}(\vec{x})$.

Пусть $\mathscr{D} \subset \mathbb{R}^n$ -- это область в ее фазовом пространстве. Возьмем произвольную точку $\vec{x}_0 \in \mathscr{D}$ и выпустим из нее фазовую траекторию. Таким образом, с течением времени $t$ мы будем двигаться по этой траектории. Обозначим точку на данной траектории в момент времени $t$ как $g^t \vec{x}_0$. 

Теперь можно определить преобразование области $\mathscr{D}$: $\forall \vec{x}_0 \in \mathscr{D}$ сделаем отображение $\vec{x}_0 \rightarrow g^t \vec{x}_0$. Получаем $\mathscr{D} \rightarrow g^t\mathscr{D}$. Другими словами, каждую точку $\mathscr{D}$ сносим по фазовой траектории на время $t$.



\tikzset{every picture/.style={line width=0.75pt}}        
\begin{tikzpicture}[x=0.75pt,y=0.75pt,yscale=-1,xscale=1]
    \draw    (247.53,281.36) .. controls (246.65,147.81) and (519.67,340.56) .. 	(539.08,208.1) ;

    \draw  [line width=2.25]  (147.34,109.16) .. controls (256.61,189.1) and 	(333.31,22.38) .. (383.41,102.31) .. controls (433.5,182.25) and (557.18,294.15) .. (464.81,371.8) .. controls (372.45,449.45) and (178.65,314.7) .. (147.34,246.19) .. controls (116.03,177.68) and (38.07,29.23) .. (147.34,109.16) -- cycle ;

    \draw  [fill={rgb, 255:red, 0; green, 2; blue, 0 }  ,fill opacity=1 ] (245.37,281.36) 	.. controls (245.37,279.81) and (246.34,278.55) .. (247.53,278.55) .. controls (248.72,278.55) and (249.68,279.81) .. (249.68,281.36) .. controls (249.68,282.91) and (248.72,284.17) .. (247.53,284.17) .. controls (246.34,284.17) and (245.37,282.91) .. (245.37,281.36) -- cycle ;

    \draw  [fill={rgb, 255:red, 20; green, 4; blue, 0 }  ,fill opacity=1 ] 	(371.82,242.13) .. controls (371.82,240.49) and (372.74,239.16) .. (373.86,239.16) .. controls (374.98,239.16) and (375.89,240.49) .. (375.89,242.13) .. controls (375.89,243.77) and (374.98,245.09) .. (373.86,245.09) .. controls (372.74,245.09) and (371.82,243.77) .. (371.82,242.13) -- cycle ;

    \draw [color={rgb, 255:red, 74; green, 144; blue, 226 }  ,draw opacity=1 ]   	(239.76,272.04) .. controls (221.69,230.23) and (304.05,180.28) .. (364.71,231.08) ;
    \draw [shift={(365.63,231.85)}, rotate = 220.6] [color={rgb, 255:red, 74; green, 144; 	blue, 226 }  ,draw opacity=1 ][line width=0.75]    (10.93,-3.29) .. controls (6.95,-1.4) and (3.31,-0.3) .. (0,0) .. controls (3.31,0.3) and (6.95,1.4) .. (10.93,3.29)   ;
    

    \draw (156.76,157.66) node [anchor=north west][inner sep=0.75pt]    {$\mathscr{D} 	\subset \mathbb{R}^{n}$};

    \draw (258.56,283.08) node [anchor=north west][inner sep=0.75pt]    {$x_{0}$};

    \draw (363.97,267.92) node [anchor=north west][inner sep=0.75pt]    {$g^{t} x_{0}$};

    \draw (492.88,273.91) node [anchor=north west][inner sep=0.75pt]  [rotate=-317.27] 	[align=left] {{\small траектория}};

    \draw (486.03,248.66) node [anchor=north west][inner sep=0.75pt]  [rotate=-316.21] 	[align=left] {{\small Фазовая}};

    \draw (264.11,188.54) node [anchor=north west][inner sep=0.75pt]    	{$g^{t}$};			
\end{tikzpicture}


Так вот преобразование $g^t$ и называется фазовым потоком.
\end{definition}


Перечислим несколько полезных свойств введенного нами фазового потока:
\begin{itemize}
\item $g^{t_1 + t_2} = g^{t_1} \cdot g^{t_2} = g^{t_2} \cdot g^{t_1}$;

\item $g^{t} \cdot g^{-t} = g^{-t} \cdot g^t = \text{Id}$ -- тождественное преобразование;

\item Фазовый поток является группой;

\item И еще сильнее, фазовый поток -- однопараметрическая группа, то есть каждому числу $t \in \mathbb{R}$ соответствует единственное преобразование $g^t: \; \mathscr{D} \rightarrow g^t \mathscr{D}$.
\end{itemize}


\begin{definition}
Пусть у нас опять есть область $\mathscr{D}$ фазового пространства $\mathbb{R}^n$. Подействуем на $\mathscr{D}$ фазовым потоком $g^t$. Тогда $\mathscr{D}(t) = g^t \mathscr{D}$ и $\vec{x} = g^t \vec{x}_0$. Определим следующую величину как фазовый объем:
\begin{equation*}
    V_\mathscr{D}(t) = \int\limits_{\mathscr{D}(t)} d\vec{x} = \int\limits_{g^t 	\mathscr{D}} d(g^t \vec{x}_0).
\end{equation*}
\end{definition}

\subsection{Теорема Лиувилля}

\begin{theorem}
В автономной системе дифференциальных уравнений $\dfrac{d\vec{x}}{dt} = \vec{f}(\vec{x})$ производная фазового объема $V_\mathscr{D}(t)$ области $\mathscr{D} \subset \mathbb{R}^n$ фазового пространства может быть вычислена по формуле:
\begin{equation*}
    \frac{dV_\mathscr{D}(t)}{dt} = \int\limits_{\mathscr{D}} \diverg\vec{f}\cdot d\vec{y},
\end{equation*}
где $\displaystyle \diverg \vec{f} = \sum\limits_{i = 1}^n \frac{\partial f^i}{\partial x^i}$ -- дивергенция $\vec{f}$, а $\vec{y} = \vec{x}(0)$.
\end{theorem}

\begin{proof}
\ \\
Докажем, что производная равна этому при $t = 0$, а в силу автономности системы это будет верно в каждой точке.

Пишем производную по определению: $\displaystyle \frac{dV_\mathscr{D}}{dt}(0) = \lim\limits_{t \rightarrow 0} \frac{V_\mathscr{D}(t) - V_\mathscr{D}(0)}{t}$.

Из системы имеем $\displaystyle \vec{x} = \vec{y} + \int\limits_{0}^t \vec{f}(\tau)d\tau$.

При малых значениях $t$ получаем следующее: $\displaystyle x^i = y^i + f^i(\vec{y})t + o(t), t\rightarrow 0$.

На все это дело можно смотреть как на замену координат $x^i \longrightarrow y^i$. Тогда получаем следующее выражение для фазового объема:
\begin{equation*}
    V_\mathscr{D}(t) = \int\limits_{\mathscr{D}(t)} d\vec{x} \stackrel{\mathscr{D}(0) = 	\mathscr{D}}{=\joinrel=\joinrel=\joinrel=\joinrel=} \int\limits_{\mathscr{D}} |J| d\vec{y},
\end{equation*}
где $J = \dfrac{\partial (x^1, x^2, \dots, x^n)}{\partial (y^1, y^2, \dots, y^n)}$ -- якобиан преобразования.

Посчитаем этот якобиан:
\begin{equation*}
    J = \begin{vmatrix}
        1 + \dfrac{\partial f^1}{\partial y^1}t & \dfrac{\partial f^1}{\partial y^2}t & 	\cdots & \dfrac{\partial f^1}{\partial y^n}t \\
        \dfrac{\partial f^2}{\partial y^1}t & 1 + \dfrac{\partial f^2}{\partial y^2}t & 	\cdots & \dfrac{\partial f^2}{\partial y^n}t \\
        \vdots & \vdots & \ddots & \vdots \\
        \dfrac{\partial f^n}{\partial y^1}t & \dfrac{\partial f^n}{\partial y^2}t & 	\cdots & 1 + \dfrac{\partial f^n}{\partial y^n}t \\
    \end{vmatrix} = \left(1 + \dfrac{\partial f^1}{\partial y^1}t\right)\left(1 + 	\dfrac{\partial f^2}{\partial y^2}t\right)\dots\left(1 + \dfrac{\partial f^n}{\partial y^n}t\right) + o(t).
\end{equation*}
Здесь мы все, что имеет множители $t^2, t^3, \dots, t^n$, завернули в $o(t)$. Однако если раскрыть скобки, то такие слагаемые все еще остаются. Раскроем эти скобки и опять впихнем все ненужное в $o(t)$:
\begin{equation*}
    J = 1 + \left(\frac{\partial f^1}{\partial y^1} + \frac{\partial f^2}{\partial y^2} + 	\dots + \frac{\partial f^n}{\partial y^n}\right)t + o(t) = 1 + t\diverg\vec{f} + o(t).
\end{equation*}

Ну, а теперь считаем эту производную:
\begin{equation*}
    \frac{dV_\mathscr{D}}{dt} = \lim\limits_{t \rightarrow 0} \frac{V_\mathscr{D}(t) - 		V_\mathscr{D}(0)}{t} = \lim\limits_{t \rightarrow 0} \dfrac{\displaystyle \int_{\mathscr{D}} \left(1 + t\diverg\vec{f} + o(t)\right) d\vec{y} - \int_{\mathscr{D}} d\vec{y}}{t} = \int\limits_{\mathscr{D}} \diverg\vec{f}\cdot d\vec{y}.
\end{equation*}
\end{proof}

\subsection{Теорема Пуанкаре}

\begin{theorem}	
Пускай $g^t$ -- непрерывное взаимнооднозначное отображение, сохраняющее фазовый объем и переводящее ограниченную область $\mathscr{D}$ саму в себя, то есть $g^t\mathscr{D} = \mathscr{D}$. Тогда:
\begin{equation*}
    \forall x_0 \in \mathscr{D} \longmapsto \forall U(x_0) \;\; \exists \overline{x} \in 	U(x_0):\; g^n \overline{x} \in U(x_0) \;\;\; (n = t_0),
\end{equation*}
где $U(x_0)$ -- некоторая окрестность точки $x_0$.

Другими словами, для любой окрестности $U$ любой точки $x_0$ области $\mathscr{D}$ найдется точка $\overline{x}$, возвращающаяся обратно в эту же окрестность.
\end{theorem}