\documentclass[a4paper, 12pt]{article}
%%% Работа с русским языком
\usepackage{cmap}					% поиск в PDF
\usepackage{mathtext} 				% русские буквы в формулах
\usepackage[T2A]{fontenc}			% кодировка
\usepackage[utf8]{inputenc}			% кодировка исходного текста
\usepackage[russian]{babel}	% локализация и переносы

%%% Дополнительная работа с математикой
\usepackage{amsmath,amsfonts,amssymb,amsthm,mathtools} % AMS
\usepackage{icomma} % "Умная" запятая: $0,2$ --- число, $0, 2$ --- перечисление

%% Номера формул
%\mathtoolsset{showonlyrefs=true} % Показывать номера только у тех формул, на которые есть \eqref{} в тексте.
%\usepackage{leqno} % Немуреация формул слева

%% Шрифты
\usepackage{euscript}	 % Шрифт Евклид
\usepackage{mathrsfs} % Красивый матшрифт

%%% Свои команды
\DeclareMathOperator{\sgn}{\mathop{sgn}}

%% Поля
\usepackage[left=2cm,right=2cm,top=2cm,bottom=2cm,bindingoffset=0cm]{geometry}

%% Русские списки
\usepackage{enumitem}
\makeatletter
\AddEnumerateCounter{\asbuk}{\russian@alph}{щ}
\makeatother

%%% Работа с картинками
\usepackage{graphicx}  % Для вставки рисунков
\graphicspath{{images/}{images2/}}  % папки с картинками
\setlength\fboxsep{3pt} % Отступ рамки \fbox{} от рисунка
\setlength\fboxrule{1pt} % Толщина линий рамки \fbox{}
\usepackage{wrapfig} % Обтекание рисунков и таблиц текстом

%%% Работа с таблицами
\usepackage{array,tabularx,tabulary,booktabs} % Дополнительная работа с таблицами
\usepackage{longtable}  % Длинные таблицы
\usepackage{multirow} % Слияние строк в таблице

%% Красная строка
\setlength{\parindent}{2em}

%% Интервалы
\linespread{1}
\usepackage{multirow}

%% TikZ
\usepackage{tikz}
\usetikzlibrary{graphs,graphs.standard}

%% Верхний колонтитул
% \usepackage{fancyhdr}
% \pagestyle{fancy}

%% Перенос знаков в формулах (по Львовскому)
\newcommand*{\hm}[1]{#1\nobreak\discretionary{}
	{\hbox{$\mathsurround=0pt #1$}}{}}

%% дополнения
\usepackage{float} %Добавляет возможность работы с командой [H] которая улучшает расположение на странице
\usepackage{gensymb} %Красивые градусы
\usepackage{caption} % Пакет для подписей к рисункам, в частности, для работы caption*

% подключаем hyperref (для ссылок внутри  pdf)
\usepackage[unicode, pdftex]{hyperref}

%%% Теоремы
\theoremstyle{plain}                    % Это стиль по умолчанию, его можно не переопределять.
\renewcommand\qedsymbol{$\blacksquare$} % переопределение символа завершения доказательства

\newtheorem{theorem}{Теорема}[section] % Теорема (счетчик по секиям)
\newtheorem{proposition}{Утверждение}[section] % Утверждение (счетчик по секиям)
\newtheorem{definition}{Определение}[section] % Определение (счетчик по секиям)
\newtheorem{corollary}{Следствие}[theorem] % Следстиве (счетчик по теоремам)
\newtheorem{problem}{Задача}[section] % Задача (счетчик по секиям)
\newtheorem*{remark}{Примечание} % Примечание (можно переопределить, как Замечание)
\newtheorem{lemma}{Лемма}[section] % Лемма (счетчик по секиям)

\newtheorem{example}{Пример}[section] % Пример
\newtheorem{counterexample}{Контрпример}[section] % Контрпример
\newcommand{\defeq}{\stackrel{def}{=}} % по определению
\newcommand{\defarr}{\stackrel{def}{\Rightarrow}} % следует из определения

\makeatletter
\newcommand{\eqnum}{\refstepcounter{equation}\textup{\tagform@{\theequation}}}
\makeatother % создание метки и нумерация формулы одновременно

\newcommand{\deflimk}{\lim\limits_{k\rightarrow \infty}} % лимит при k -> бесконечности
\DeclareMathOperator{\Tr}{trace} % след матрицы
\usepackage{indentfirst}


\begin{document}
    \section{Билет номер 5}
    \subsection{Устойчивость по Ляпунову. Асимптотическая устойчивость}
    Рассматривается общая система дифференциальных уравнений
    \begin{equation}
    	\frac{d\bar{x}}{dt} = \bar{f}(t, x^1, \dots, x^n)
    \end{equation}
    Пусть $\bar{x} = \bar{\varphi}(t, \bar{x}_0)$ -- решение этой системы, такое что $\bar{\varphi}(t_0, \bar{x}_0) = \bar{x}_0$. А $\psi(t, \bar{\tilde{x}}_0)$ -- произвольное решение, такое что $\psi(t, \bar{\tilde{x}}_0) = \bar{\tilde{x}}_0$.
    \begin{definition}
    	Решение $\bar{x} = \bar{\varphi}(t, \bar{x}_0)$ называется \textbf{устойчивым по Ляпунову}, если
    	\[
    		\forall \varepsilon >0\exists \delta > 0:\forall \bar{x}_0: |\bar{\tilde{x}}_0 - \bar{x}_0|<\delta \to |\bar{\psi}(t, \bar{\tilde{x}}_0) -  \bar{\varphi}(t_0, \bar{x}_0)|<\varepsilon\text{ } \forall t\in[t_0, +\infty]
    	\]
    \end{definition}
		
	\begin{definition}
		Решение $\bar{x} = \bar{\varphi}(t, \bar{x}_0)$ называется \textbf{асимптотически устойчивым}, если оно устойчиво по Ляпунову, а так же
		\[
			\forall \varepsilon > 0\exists \delta>0:\forall \bar{x}_0: |\bar{\tilde{x}}_0 - \bar{x}_0|<\delta \to \lim_{t\to\infty}|\bar{\psi}(t, \bar{\tilde{x}}_0) -  \bar{\varphi}(t_0, \bar{x}_0)|=0
		\]
	\end{definition}

	\subsection{Автономные линейные системы}
	
	Пусть в конечномерном линейном пространстве $B$ линейный оператор задается матрицей $A=||a_{ij}(t)||$. Если $a_{ij}$ ограничены, тогда норма матрицы 
	\[
		||A|| = \max_{i,j =\overline{1,n}}|\sup_{t\in I(t)}(a_{ij}(t))|
	\]
	Можно записать следующее неравенство:
	\[
		||A\bar{x}||\leq ||A||\cdot||\bar{x}||
	\]
	
	Теперь рассмотрим систему однородных уравнений, где $A$ постоянна
	\begin{equation}
		\frac{d\bar{x}}{dt} = A\bar{x}
		\label{equ:issue12sys}
	\end{equation}
	Тогда $\bar{x} = 0$ -- решение.
	
	\begin{lemma}
		Если однородная линейная система имеет неограниченное решение, то нулевое решение не устойчиво.
	\end{lemma}

	\begin{proof}
		Будем рассматривать систему (\ref{equ:issue12sys}). Пусть решение $\bar{\varphi}(t, \bar{x}_0)$ неограниченно. То есть 
		\[
			\forall M>0 \exists t^*:|\bar{\varphi}(t^*, \bar{x}_0)|>M
		\]
		
		Обратим определение устойчивости нулевого приближения 
		\[
			\exists \varepsilon_0>0: \forall \delta>0 \exists \bar{x}_0: |\bar{x}_0|<\delta, \exists t^*\in[t_0, +\infty): |\bar{\varphi}(t^*, \bar{x}_0)|>\varepsilon
		\]
		Воспользуемся неограниченностью решения
		\[
			\forall C>0 \to \bar{\psi}(t, \bar{x}_0) = C\cdot\bar{\varphi}(t, \bar{x}_0) \text{ -- неограниченно} 
		\]
		Теперь для произвольного $\delta >0 $ возьмем $C = \dfrac{\delta}{2|\bar{x}_0|}$
		\[
			|\bar{\psi}(t_0, \bar{x}_0)| = C\cdot|\bar{\varphi}(t_0, \bar{x}_0)| = \frac{\delta |\bar{x}_0|}{2|\bar{x}_0|} = \frac{\delta}{2}
		\]
		Таким образом
		\[
			\exists \varepsilon_0, \exists t: 	|\bar{\psi}(t, \bar{x}_0)| > \varepsilon_0
		\]
	\end{proof}
	
	\begin{theorem}
		Пусть $\lambda_1, \dots, \lambda_k$ -- собственные числа матрицы $A$ кратности $l_1, \dots, l_k$ соответственно. Тогда
		\begin{enumerate}
			\item Если $Re(\lambda_i)<0, i=\overline{1, k}$, то нулевое решение асимптотически устойчиво.
			\item Пусть $Re(\lambda_i)<0, i\neq l, Re(\lambda_l)=0$. И существует базис из собственных векторов $e_{l_1}, \dots, e_{l_k}$. Тогда нулевое решение устойчиво по Ляпунову.
			\item Если $\exists l: Re(\lambda_l) > 0$, или $Re(\lambda_l) = 0$, но собственные вектора не образуют базис, тогда нулевое решение не устойчиво
		\end{enumerate}
	\end{theorem}
	
	\begin{proof}
		Рассмотрим решение $\bar{\varphi}(t, \bar{x}_0)$, такое что $\bar{\varphi}(0, \bar{x}_0)=\bar{x}_0$. Тогда
		\[
			\bar{x}(t) = e^{tA}\cdot \bar{x}_0
		\]
		где 
		\[
			e^{tA} = S\begin{Vmatrix}
			e^{\lambda_1t}P_{ij}^1(t) &0&0&\dots\\
			0&e^{\lambda_2t}P_{ij}^2(t)&0&\dots\\
			&&\dots\\
			\dots&0&0& e^{\lambda_kt}P_{ij}^k(t)
			\end{Vmatrix}S^{-1} = ||e^{\lambda_st}P_{ij}(t)||
		\]
		$S$ -- матрица перехода к Жорданову базису. $P_{ij}$ -- многочлены степени $m$ $$m \leq \max_{s=\overline{1,k}}(l_s-1)$$
		Рассмотрим случаи по порядку:
		\begin{enumerate}
			\item $e^{(\alpha_s+i\omega_s)t}P_{ij}^s(t)$ -- элемент $e^{tA}$. $e^{(\alpha_s+i\omega_s)t}P_{ij}^s(t) = e^{\alpha_st}(\cos{(\omega_s t)}+i\sin{(\omega_s t)})P_{ij}^s(t)$. Тогда $|e^{(\alpha_s+i\omega_s)t}P_{ij}^s(t)| = e^{\alpha_st}|P_{ij}^s(t)|$. Положим $\alpha = \inf_{i=\overline{1,k}}|\alpha_i|$. Распишем
			\[
				e^{tA} = e^{-\alpha t}(e^{\alpha t}e^{tA}) = e^{-\alpha t}\Phi(t)
			\]
			Произвольный элемент матрицы $\Phi(t)$
			\[
				\Phi_{ij}(t) = e^{-rt}P_{ij}(t)
			\]
			где $r>0$. Отсюда видно, что 
			\[
				\lim_{t\to\infty}e^{-rt}P_{ij}(t) = 0
			\]
			Тогда все элементы матрицы $\Phi(t)$ ограничены. Обозначим норму этой матрицы
			\[
				m = ||\Phi(t)||
			\]
			
			Для произвольного $\varepsilon$ возьмем $\delta = \frac{\varepsilon}{m}$. Теперь возьмем норму решения $\bar{x}(t)$.
			\[
				||\bar{x}(t)|| = ||e^{tA}\cdot \bar{x}_0|| \leq ||e^{tA}||\cdot|| \bar{x}_0|| = e^{-\alpha t}||\Phi(t)||\cdot|| \bar{x}_0||\leq e^{-\alpha t}m|| \bar{x}_0||\leq e^{-\alpha t}m\delta\leq e^{-\alpha t}\varepsilon
			\]
			\[
				\lim_{t\to\infty}e^{-\alpha t}\varepsilon = 0
			\]
			
			\item В данном случае $P_{ij}^l = const$, тогда $e^{-\alpha t}$ не будет. Следовательно $||\bar{x}(t)||\leq \varepsilon$ -- устойчивость по Ляпунову.
			\item $Re(\lambda_s)> 0$. Тогда решение 
			\[
				\bar{\varphi}(t, \bar{x}_0) = e^{(\alpha_s+i\omega_s)t}\cdot C \text{ -- неограниченно}
			\]
			А если $Re(\lambda_s)= 0$, но в базисе присутствуют присоединенные вектора, тогда решение принимает вид $P_{ij}(t)$ -- неограниченно при $t\to+\infty$ 
		\end{enumerate}
		
	\end{proof}

	
	
	
\end{document}






















