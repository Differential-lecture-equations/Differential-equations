\subsection{Теорема Штурма}

Рассмотрим на промежутке $I = I(x)$ следующее уравнение:

\begin{equation}\label{diff-eq_issue12}
y'' + a(x)y' + b(x)y = 0,
\end{equation}

где $a(x) \in C^1_{I(x)}$, $b(x) \in C^1_{I(x)}$.

Решение (\ref{diff-eq_issue12}) такое, что $y(x)$ тождественно не равно нулю на $I(x)$ называется нетривиальным , а точка $x_0 \in I$ такая, что $y(x_0) = 0$ называется нулём нетривиального решения $y(x)$.

Уравнение (\ref{diff-eq_issue12}) приводится к виду:

\begin{equation}\label{diff-eq_z-form_issue12}
z'' + q(x)z = 0.
\end{equation}

Для этого сделаем замену $y(x) = c(x) \cdot z(x)$, где $z(x)$ - решение уравнения выше (далее будем считать, что $c = c(x)$ и $z = z(x)$:

\[z''\cdot c + 2c' \cdot z' + c'' \cdot z + a(x) (c' \cdot z + z' \cdot c) + b(x) \cdot c \cdot z = 0,\]

здесь выберем $c \neq 0$ так, что бы для $z'$ выполнялось:

\[z' (2c' + a(x) c) = 0.\]

Тогда получаем линейное однородное уравнение $\Rightarrow \: 2c' + a(x)c = 0$, которое можно преобразовать в:

\begin{equation}\label{c-x-solution_issue12}
\frac{dc}{c} = - \frac{a(x)}{2x}dx\: \Rightarrow \: c(x) = c_0 \cdot exp \left[-\frac{1}{2} \int a(x)dx\right] > 0.
\end{equation}

Возьмем $c_0 = 1 \: \Rightarrow \: c \cdot z'' + (c'' + c'a + bc)z = 0$, тогда можем ввести $q(x)$ такое, что:

\[q(x) = \frac{c'' + c'a}{c} + b.\]

Также заметим, что из (\ref{c-x-solution_issue12}) следует, что $c(x) > 0$. Тогда в силу замены $y = c(x) \cdot z$, $x_0 \in I$ является нулём $y(x)$ тогда и только тогда, когда $x_0$ является нулём $z(x)$.

\begin{definition}
Точка $x_0$ является нулём $f(x) \in C^{\infty}$ кратности $k$, если $f(x_0) = f'(x_0) = ... = f^{(k-1)}(x_0) = 0$, а $f^{(k)}(x_0) \neq 0$.
\end{definition}

\begin{lemma}\label{sol-zeros_issue12}
Все нули нетривиального решения (\ref{diff-eq_z-form_issue12}) (также как и для (\ref{diff-eq_issue12})) являются простыми, т.е. $k = 1$.
\end{lemma}

\begin{proof}
От противного: пусть $x_0$ является нулём кратности 2, тогда $z(x_0) = z'(x_0) = 0$. Тогда в силу основной теоремы $z(x) = 0 \forall x \in I$ - противоречие, т.к. $z(x)$ - нетривиальное решение по условию.
\end{proof}

\begin{lemma}\label{sol-set_issue12}
Пусть M - множество нулей нетривиального решения $y(x)$ на нечетном промежутке $[x_1;x_2]$. Множество M не имеет предельной точки.
\end{lemma}

\begin{proof}
От противного: пусть M - множество нулей. Пусть $x_0$ - предельная точка и $\exists {x_k} :$ 

\[\deflimk x_k = x_0 \in [x_1;x_2], \: y(x_k) = 0, \: k = 1,2,...\]

Так как $y(x)$ - непрерывно, то $\deflimk y(x_k) = 0 = y (\deflimk x_k) = y(x_0) \: \Rightarrow \: y(x_0) = 0$.

Рассмотрим $[x_k;x_{k+1}]$ и $y(x)$ на нём, т.к. $y(x_k) = y(x_{k+1}) = 0$, то по теореме Ролля $\exists c_k: x_k \leq c_k \leq x_{k+1}: y'(c_k) = 0$ и т.к. $\deflimk x_k = \deflimk x_{k+1} = x_0 \Rightarrow \deflimk c_k = x_0$. Из этого может получить, что так как $y'(x)$ - непрерывна, то:

\[\deflimk y' (c_k) = 0 = y' (\deflimk c_k) = y'(x_0) = 0\]

Так как по предложению $x_0 \in [x_1;x_2]$ и $y_0(x_0) = 0$, $y'(x_0) = 0$ - получим задачу Коши для $x_0 \in [x_1;x_2] \Rightarrow$ в силу теорем существования и единственности решения задачи Коши: $y\equiv 0$ - единственное решение на $[x_1;x_2]$ - получим противоречие с нетривиальным решением.
\end{proof}

\begin{theorem}
[Теорема Штурма]

Рассмотрим уравнения:

\begin{equation}\label{fast-eq_issue12}
y'' + q(x) y = 0
\end{equation}

\begin{equation}\label{slow-eq_issue12}
z'' + Q(x)z = 0,
\end{equation}

где уравнение (\ref{fast-eq_issue12}) будем называть быстрым, а (\ref{slow-eq_issue12}) - медленным.

Пусть 

\[q(x)\in C^1_{I(x)}, Q(x) \in C^1_{I(x)}, \forall x \in I \rightarrow q(x) \leq Q(x).\]

Пусть $y(x)$ - нетривиальное решение (\ref{fast-eq_issue12}), z(x) - нетривиальное решение (\ref{slow-eq_issue12}). Если $x_1, x_2 \in I$ - последовательное нули $y(x)$, то либо $\exists x_0 \in (x_1;x_2)$, в которой $z(x_0) = 0$, либо $z(x_1) = z(x_2) = 0$.
\end{theorem}

\begin{proof}
Пусть $x_1, x_2$ - два соседних нуля $y(x)$, т.е. $y(x) \neq 0$ на $(x_1;x_2)$, пусть для определённости $y(x) > 0$.

По определению:

\[y'(x_1) = \lim\limits_{x\rightarrow x_1} \frac{y(x) - y(x_1)}{x - x_1} \geq 0; \: y'(x_2) = \lim\limits_{x\rightarrow x_2} \frac{y(x) - y(x_2)}{x - x_2} \leq 0 .\]

В силу Леммы \ref{sol-zeros_issue12} нули $x_1$ и $x_2$ должны быть однократными, т.е. $y'(x_1) \neq 0, \: y'(x_2) \neq 0$. Таким образом $y'(x_1) > 0, \: y'(x_2) < 0$.

Умножим (\ref{slow-eq_issue12}) на $z(x)$, а (\ref{fast-eq_issue12}) на y(x) и вычтем из первого второе:

\[zy'' + qyz - yz''' - Qyz = 0; \: zy'' - yz'' = (zy'-yz')' = (Q - q)zy.\]

Проинтегрируем полученное тождество на $[x_1;x_2]$:

\[(zy' - yz')\Big\vert_{x_1}^{x_2} = \int\limits_{x_1}^{x_2} (Q(x) - q(x))zydx;\]

\[z(x_2)y'(x_2) - y(x_2)z'(x_2) - z(x_1)y'(x_1) + y(x_1)z'(x_1) = \int\limits_{x_1}^{x_2}(Q(x) - q(x))zydx \Rightarrow\]

\begin{equation}\label{add-diff_issue12} % additional differential equation
\Rightarrow z(x_2)y'(x_2) - z(x_1)y'(x_1) = \int\limits_{x_1}^{x_2}(Q(x) - q(x))zydx,
\end{equation}

здесь $z(x_2)y'(x_2) < 0$, $z(x_1)y'(x_1) > 0$, $(Q(x) - q(x)) > 0$ и $y > 0$.

Предположим противное - пусть теорема Штурма не верна. Тогда возможны варианты:

\begin{enumerate}
\item $z > 0 \forall x \in [x_1;x_2].$ Тогда левая часть (\ref{add-diff_issue12}) отрицательна, а правая положительна - противоречие.

\item $z > 0 \forall x \in [x_1;x_2), \: z(x_2) = 0$ - аналогично.

\item $z > 0 \forall x \in (x_1;x_2]б \: z(x_1) = 0$ - аналогично.
\end{enumerate}

Таким образом $\exists x_0 \in (x_1;x_2): \: z(x_0) = 0$. Если $z(x_1) = z(x_2)$, то может быть, что $Q(x) \equiv q(x) \Rightarrow z(x) = const \cdot y(x)$, либо:

\[\exists x* \in (x_1;x_2): \: Q(x*) > q(x*),\]

в силу непрерывности $Q(x)$ и $q(x) \exists \bigtriangleup:$

\[\int\limits_{x* -\bigtriangleup}^{x* + \bigtriangleup} (Q(x) - q(x))z(x)y(x)dx = 0,\]

значит $\exists x_0$, где $z(x)$ меняет знак $\Rightarrow z(x_0) = 0$
\end{proof}

\subsection{Следствия из теоремы Штурма}

\begin{corollary}\label{zero-quant_in-diff_issue12}
Пусть есть уравнение:

\[y'' + q(x)y = 0; \: q(x) \leq0 \forall x \in I(x),\]

тогда любое нетривиальное решение (\ref{fast-eq_issue12}) на $I$ имеет не более одного нуля.
\end{corollary}

\begin{proof}
В качестве второго уравнения можно взять $z'' + Q(x)z = 0$, здесь Q(x) = 0. Пусть решение уравнения (\ref{fast-eq_issue12}) имеет нули $x_1$ и $x_2$ $Q(x) \geq q(x) \Rightarrow 0 \geq q(x)$. Тогда по теореме Штурма любое решение (\ref{slow-eq_issue12}) должно иметь ноль на $(x_1;x_2)$. В качестве решения можем вщять $z\equiv 1$, которое не и имеет нулей $\Rightarrow$ противоречие $\Rightarrow$ для решения (\ref{fast-eq_issue12}) не может быть больше одного нуля.
\end{proof}

\begin{corollary}\label{line-indep-sol-zeros_issue12}
Пусть $\varphi (x)$ и $\psi (x)$ - два линейно независимых нетривиальных решения (\ref{fast-eq_issue12}), $x_1, x_2 \in I$ - два соседних нуля $\varphi (x)$, тогда $\psi (x)$ имеет только один нуль на $(x_1;x_2)$.
\end{corollary}

\begin{proof}
Применим теорему Штурма к двум одинаковым уравнениям $(q(x) \equiv Q(x))$. По теореме Штурма $\psi (x)$ на $(x_1;x_2)$ имеет хотя бы один нуль. Общих нулей $\varphi (x)$ и $\psi (x)$ иметь не могут, так как они линейно независимые ($W(x_1) = 0$, если бы $\varphi (x_1) = \psi (x_1) = 0$, что означало бы, что $\varphi(x)$ и $\psi(x)$ - ЛЗ). Итак, $\psi (x)$ имеет нуль $x_0$ на $(x_1;x_2)$.

Докажем, что такой нуль единственный - от противного: пусть нулей два для $\psi (x): x*$ и $\overline{x}$. Если нулей $\psi (x)$ два, то по теореме Штурма для $\varphi (x)$ будет ноль между $x*$ и $\overline{x}$ - противоречие тому, что $x_1$ и $x_2$ соседние нули $\varphi (x)$.
\end{proof}

Из \ref{line-indep-sol-zeros_issue12} очевидно получим следующий результат:

\begin{corollary}\label{infinity-zeros-amount_issue12}
    Пусть $\varphi(x)$ некторое нетривиальное решение уравнения \eqref{fast-eq_issue12}. Если Пусть $\varphi(x)$
    имеет бесконечное число нулей, то и каждое нетривиальное решение $\psi(x)$ уравнения \eqref{fast-eq_issue12} имеет бесконечное число нулей.
\end{corollary}

\begin{example}
    Пара функций $\sin x$ и $\cos x$, являющиеся нетривиальными решениями уравнения $y^{''} + y = 0$ 
    иллюстрирует следствия \ref{line-indep-sol-zeros_issue12} и \ref{infinity-zeros-amount_issue12}. 
\end{example}