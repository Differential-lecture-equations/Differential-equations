\section{Билет 2. Задача Коши}
\subsection{Принцип сжимающих отображений}

Работаем в $E =\mathbb{R}^n$ -- пространство точек с $n$ координатами. $E$ -- аффинное пространство, а $\vec{E}$ -- его присоединенное линейное пространство, состоящее из векторов, натянутых на точки $E$.

\begin{definition}
	Пусть $L$ -- это векторное пространство, и на нем задано отображение $\|\cdot\|:L\longrightarrow \mathbb{R}$ такое, что:
	\begin{enumerate}
		\item $\forall x \in L \longmapsto \|x\| \geqslant 0$. А также $\|x\| = 0 \Longleftrightarrow x = 0$;
		
		\item $\forall x \in L \; \& \; \forall \lambda \in \mathbb{R} \longmapsto \|\lambda x\| = |\lambda| \cdot \|x\|$;
		
		\item $\forall x, y \in L \longmapsto \|x+y\| \leqslant \|x\| + \|y\|$ -- неравенство треугольника.
	\end{enumerate}
	Тогда данное отображение называется нормой, а пространство $L$ нормированным.
\end{definition}

\begin{example}
	Приведем пример норм. Пусть $a(x_1, x_2, \dots, x_n) \in \mathbb{R}^n$. Тогда норму можно определить, допустим, так:
	\begin{equation}
		\|a\|_1 = \sqrt{\sum_{j = 1}^{n} x_j^2}.
	\end{equation}
Или так:
	\begin{equation}
		\|a\|_2 = \max_{j = 1, \dots, n}|x_j|.
	\end{equation}
\end{example}

И тогда можно ввести понятие эквивалентности норм.

\begin{definition}
	Пусть снова $L$ -- линейное пространство. Тогда нормы $\| \cdot \|_1$ и $\| \cdot \|_2$ на $L$ называются эквивалентными, если $\exists C_1, C_2 > 0: \; \forall x \in L \longmapsto C_1\|x\|_1 \leqslant \|x\|_2 \leqslant C_2\|x\|_1$.
\end{definition}

Как видно, для определенных выше двух норм это соотношение удовлетворяется.

\begin{proposition}
	В конечномерном линейном пространстве все нормы эквивалентны.
\end{proposition}

Рассмотрим множество функций, непрерывных на отрезке $[a; b]$ для некоторых неравных $a, b \in \mathbb{R}$ и обозначим данное множество $C[a; b]$. Понятно, что $C[a; b]$ является линейным пространством. Тогда введем на нем норму.

\begin{definition}
	Нормой функции $f(x) \in C[a; b]$ будем называть число $$\|f(x)\| = \max_{x \in [a; b]}|f(x)|.$$
\end{definition}

\begin{definition}
	Набор функций $f_1(x), f_2(x), \dots, f_n(x) \in C[a; b]$ будем называть вектор-функцией и обозначать $f(x) = \vec{f}(x) = \left(f_1(x), f_2(x), \dots, f_n(x)\right)^T$.
\end{definition}

\begin{definition}
	Вектор-функция $f(x)$ называется непрерывной (дифференцируемой, непрерывно дифференцируемой и т.п.), если все ее компоненты непрерывны (дифференцируемы, непрерывно дифференцируемы и т.п.).
\end{definition}

\begin{definition}
	Модулем вектор-функции $f(x)$ назовем число
	\begin{equation}
		|f(x)| = \sqrt{\sum_{j = 1}^{n} f_j^2(x) }.
	\end{equation}
\end{definition}

Норму вектор-функции можно определить как $$\|f(x)\|_1 = \max_{x \in [a; b]}|f(x)|.$$

Или же как $$\|f(x)\|_2 = \max_{j = 1, \dots, n} \max_{x \in [a; b]} f_j(x).$$

Понятно, что эти две нормы эквивалентны.

\begin{definition}
	Пусть имеется функциональная последовательность $\{f_n(x)\}_{n = 1}^{\infty}$, где $f_n(x) \in C[a; b]$ -- линейное пространство функций с нормой (1 или 2 -- неважно). Тогда говорят, что данная последовательность сходится к функции $f(x)$ по норме, если:
	\begin{equation}
		\lim\limits_{n\rightarrow \infty} \|f_n(x) - f(x)\| = 0.
	\end{equation}

Аналогично все то же самое и точно так же определяется и для вектор-функций $f(x) = \vec{f}(x) \in C^n[a; b]$.
\end{definition}

\begin{definition}
	Функциональная последовательность $\{f_n(x)\}_{n = 1}^{\infty}$ называется фундаментальной, если:
	\begin{equation}
		\forall \varepsilon > 0 \; \exists N \in \mathbb{N}: \; \forall n \geqslant N \; \& \; \forall m \geqslant N \longmapsto \|f_n(x) - f_m(x)\| < \varepsilon.
	\end{equation}
\end{definition}

\begin{definition}
	Функциональное пространство $L$ называется полным по [данной] норме, если любая фундаментальная функциональная последовательность данного пространства сходится по норме к функции из этого же пространства $L$.
\end{definition}

\begin{theorem}
	Функциональное пространство $C[a; b]$ с нормой $\| \cdot \|_1$ является полным.
\end{theorem}
\begin{proof}
	Возьмем произвольную функциональную последовательность $\{f_n(x)\}_{n = 1}^{\infty}$ из нашего пространства непрерывных функции. Тогда из определения фундаментальности следует, что $\|f_n(x) - f_m(x)\| < \varepsilon$.
	
	Однако $|f_n(x) - f_m(x)| \leqslant \|f_n(x) - f_m(x)\| < \varepsilon \; \forall x \in [a; b]$.
	
	А значит, последовательность $f_n(x)$ сходится к некоторой $f(x)$, причем равномерно на $[a; b]$ (числовая последовательность $\|f_n(x)\|$ мажорирует функциональную последовательность $f_n(x)$).
	
	Так как $f_n(x) \in C[a; b]$ -- непрерывны $\forall n \in \mathbb{N}$, и последовательность сходится равномерно на $[a; b]$, то предельная функция $f(x)$ также является непрерывной на $[a; b]$, а значит, $f(x) \in C[a; b]$. 
	
	Таким образом, последовательность $\{f_n(x)\}_{n = 1}^{\infty}$ сходится к $f(x) \in C[a; b]$. В силу произвольности $\{f_n(x)\}_{n = 1}^{\infty}$ заключаем, что функциональное пространство $C[a; b]$ с нормой $\| \cdot \|_1$ является полным.
\end{proof}

\begin{definition}
	Полное нормированное линейное пространство называется Банаховым. Обозначается $B$.
\end{definition}

\begin{definition}
	Функциональный ряд $\sum\limits_{k = 1}^{\infty} f_k(x)$ называется сходящемся по норме, если последовательность его частичных сумм $S_n(x) = \sum\limits_{k = 1}^{n} f_k(x)$ является сходящейся по норме.
\end{definition}

\begin{definition}
	Пусть $\forall x \in M \subseteq B$ определен элемент $Ax \in B$. Тогда говорят, что на множестве $B$ задан оператор $A$ с областью определения $M$.
\end{definition}

Будем рассматривать уравнение $x = Ax$.

\begin{definition}
	Множество $M \subseteq B$ называется ограниченным, если $\exists C > 0$ такое, что $\forall x \in M \longmapsto \|x\| \leqslant C$.
\end{definition}

\begin{definition}
	Оператор $A$ называется сжатием на $M$, если:
	\begin{enumerate}
		\item $\forall x \in M \longmapsto Ax \in M$;
		
		\item $\exists k \in (0; 1): \; \forall x, y \in M \longmapsto \|Ax - Ay\| \leqslant k\|x -y\|$.
	\end{enumerate}
\end{definition}

\begin{theorem}
	[Принцип сжимающих отображений]
	
	Пусть множество $M \subseteq B$, причём $M \neq \varnothing$, является ограниченным и замкнутым, а оператор $A$ является сжатием. Тогда решение уравнения $x = Ax$ существует и единственно.
\end{theorem}

\begin{proof}
	Будем использовать итерационный метод, согласно которому мы выбираем начальное $x_0$, а затем строим последовательность $x_n = Ax_{n-1}$. Тогда, если $\exists \lim\limits_{n \rightarrow \infty} x_n = x$ и $\exists \lim\limits_{n \rightarrow \infty} Ax_n = Ax$, то $x = Ax$.
	
		Пусть $x_n = S_n = x_0 + (x_1 - x_0) + \ldots + (x_n - x_{n-1})$.
		Докажем, что $\|x_{n+1} - x_n\| \leqslant 2Ck^n$ для некоторого $C > 0$, ограничивающего последовательность $x_n$. Сделаем это по индукции.
		
		База индукции: $\|x_1 - x_0\| \leqslant \|x_1\| + \|x_0\| \leqslant 2C$.
		
		Предположим, что $\|x_n - x_{n-1}\| \leqslant 2Ck^{n-1}$. Тогда получаем, что $\|x_{n+1} - x_n\| = \|Ax_n - Ax_{n-1}\| \leqslant k \|x_n - x_{n-1}\| \leqslant 2Ck^n$.
		
		И получаем, что $\|x_0 + \sum\limits_{j = 1}^{\infty} (x_j - x_{j-1})\| \leqslant \|x_0\| + \sum\limits_{j = 1}^{\infty} 2Ck^{n-1} < \infty$. 
		
		А значит $\exists \lim\limits_{n \rightarrow \infty} x_n = x$. А поскольку $M$ замкнуто, то $x \in M$. 
		
		Теперь рассмотрим разность $\|Ax_n - Ax\| \leqslant k\|x_n - x\| \underset{n \rightarrow \infty}{\longrightarrow} 0$. Это означает, что $\exists \lim\limits_{n \rightarrow \infty} Ax_n = Ax$. 
		
		Учитывая, что $x_{n+1} = Ax_n$, то, перейдя к пределу с обеих частей равенства, мы получаем, что итерационный метод сходится к решению уравнения $x = Ax$. И таким образом, доказано существование решения. Теперь докажем его единственность.
		
		Пойдем от противного: пусть $x$ и $y$ -- два разных решения. Тогда $\|x - y\| = \|Ax - Ay\| \leqslant k \|x - y\|$. Учитывая, что $k \in (0; 1)$, то данная ситуация возможна тогда и только тогда, когда $\|x - y\| = 0$. Следовательно, $x = y$, что противоречит тому, что это два разных решения. Итак, теорема доказана.
\end{proof}