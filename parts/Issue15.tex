
\section{Билет 7. Элементы вариационного исчисления}
\subsection{Основные понятия}
\begin{definition}
    Пусть $M$ -- множество функций $y(x)$, а $\mathcal{J}$ -- отображение $M$ в $\mathbb{R}$ такое, что $\mathcal{J} = \{ \mathcal{J}(y(x)) \in \mathbb{R}: \forall y(x) \in M \}$. 
    Такое отображение называется \textbf{функцианалом}, а $M$ -- область его определения. 
\end{definition}

$\forall y(x) \in C^1_{[a;b]}$ рассмотрим функционал \fbox{$\mathcal{J}(y(x)) = \int \limits^b_a F(x, y(x), y'(x)) dx$}

Будем считать, что $F(x, y(x), y'(x))$ как функцию трех независимых переменных $x_1 = x$, $x_2 = y(x)$, $x_3 = y'(x)$, непрерывна вместе с $\cfrac{\partial{F}}{\partial{x_i}}$,  $\cfrac{\partial^2{F}}{\partial{x_i}\partial{x_j}}$, $i,j = \overline{1, 3}$ 

\subsubsection*{Постановка вариационной задачи}

Вариационная задача состоит в том, чтобы среди функций $y(x) \in D \subset C^1_{[a;b]}$ (в случае наличия дополнительного условия) найти такую функцию $y_0(x)$, что $\mathcal{J}(y_0(x))$ принимает минимальное (максимальное) значение. Будет рассматривать $y(x) \in C^1_{[a;b]}$.

\begin{definition}
    Множество функций $D$, которые удовлетворяет свойствам, которые мы наложим, называется \textbf{множеством варьируемых функций}.
\end{definition}

\begin{definition}
    $y_0(x)$ такое что $\mathcal{J}(y_0(x)) \leq \mathcal{J}(y(x)) [\mathcal{J}(y_0(x)) \geq \mathcal{J}(y(x))] \, \forall y(x) \in D$ называется \textbf{абсолютным экстремумом} $\mathcal{J}$.  
\end{definition}

Введём норму на $C^1_{[a;b]}$ для определения типа экстремумов: $\| y(x) \| = \max \limits_{x \in [a;b]} |y(x)| + \max \limits_{x \in [a;b]} |y'(x)|$ -- все свойства нормы выполнены.

\begin{definition}
    Пусть $y(x) \in D$. Функцию $\delta y(x) \in C^1_{[a;b]}$ будем называть \textbf{допустимый вариацией} $y(x)$, если $\forall y$: $y + \delta y \in D$ 
\end{definition}

\begin{definition}
    Множество функций $V_{\varepsilon}(y_0(x)) = \{ y(x) \in C^1_{[a;b]}: \| y(x) - y_0(x)\| \leq \varepsilon \}$ будем называть \textbf{$\varepsilon$-окрестностью $\mathbf{y_0(x)}$}
\end{definition}

\subsubsection*{Основной принцип}

Пусть $y_0(x) \in D$ фиксирована, а $\delta y(x)$ -- какая-то фиксированная допустимая вариация такая, что $\forall t \in [-1;1] \mapsto y_0(x) + t \delta y(x) \in D \Rightarrow $ 
\[\mathcal{J}(y(x)) = \mathcal{J}(y_0(x) + t \delta y(x)) = \int \limits_a^b F(x, y_0(x) + t \delta y(x), y_0'(x) + t (\delta y(x))') dx = \mathcal{J}(t) \]

В силу определения $F$, у него существуют 1 и 2 непрерывные производные по $t$, т.е $\mathcal{J}(t)$ -- дважды непрерывно дифференцируемая по $t$ функция. Следовательно из формулы Тейлора:
\begin{equation*}
    \mathcal{J}(y_0(x) + t \delta y(x)) = \mathcal{J}(0) + \frac{d\mathcal{J}}{dt}(0) \cdot t + \frac{1}{2} \frac{d^2 \mathcal{J}}{dt^2}(0) \cdot t^2 + o(t^2) = \left[\text{~обозначим } (\delta y(x))' = \delta y' ~\right] \fbox{=}
\end{equation*}

\[ \frac{d \mathcal{J}}{dt}(t) = \int \limits_a^b \left[  \frac{\partial F}{\partial y}(x, y_0 + t \delta y, y'_0 + t \delta y') \delta y + \frac{\partial F}{\partial y'} (x, y_0 + t \delta y, y'_0 + t \delta y') \delta y' \right] dx \] 
\begin{equation} \label{issue15:FJ} 
        \frac{d \mathcal{J}}{dt}(0) = \int \limits_a^b \left[  \frac{\partial F}{\partial y}(x, y_0, y'_0) \delta y + \frac{\partial F}{\partial y'} (x, y_0, y'_0) \delta y' \right] dx = \delta \mathcal{J} \text{- первая вариация}
\end{equation}


\begin{multline*}
    \frac{d^2 \mathcal{J}}{dt^2}(t) = \int \limits_a^b \left[  \frac{\partial^2 F}{\partial y^2}(x, y_0 + t \delta y, y'_0 + t \delta y') \delta y^2 + \frac{\partial^2 F}{\partial y \partial {y'}} (x, y_0 + t \delta y, y'_0 + t \delta y') \delta y \delta {y'} \right. + \\ 
    \left. + \frac{\partial^2 F}{\partial {y'}^2} (x, y_0 + t \delta y, y'_0 + t \delta y') \delta {y'}^2  \right] dx 
\end{multline*}

\begin{equation} \label{issue15:SJ}
    \frac{d^2 \mathcal{J}}{dt^2}(0) = \int \limits_a^b \left[  \frac{\partial^2 F}{\partial y^2}(x, y_0, y'_0) \delta y^2 
    + \frac{\partial^2 F}{\partial y \partial {y'}} (x, y_0, y'_0) \delta y \delta {y'} + \frac{\partial^2 F}{\partial {y'}^2} (x, y_0, y'_0) \delta {y'}^2  \right] dx 
    = \delta^2 \mathcal{J}
\end{equation}

\[ \delta^2 \mathcal{J} \text{- вторая вариация} \]

\[ \fbox{=}~~ \mathcal{J}(0) + \delta \mathcal{J}\cdot t + \delta^2 \mathcal{J} \cdot t^2 + o(t^2)\]

\begin{definition}
    Функция $y_0(x) \in D$ называется слабым экстремумом функцианала $\mathcal{J}$, если $\exists \varepsilon > 0: \mathcal{J}(y_0(x)) \leq \mathcal{J}(y(x)) \left[ \mathcal{J}(y_0(x)) \geq \mathcal{J}(y(x)) \right] \forall y(x) \in V_{\varepsilon} (y_0(x))$, т.е. $\forall y(x): \| y(x) - y_0(x)\| \leq \varepsilon$. 
\end{definition}

\begin{theorem}[Основная теорема]
    Пусть $y_0(x) \in D \subset C^1_{[a;b]}$ является слабым экстремумом функцианала $\mathcal{J}(y(x))$. Тогда первая вариация $\delta \mathcal{J}(y_0, \delta y) = 0$ \fbox{$\forall \,\text{допустимой } \delta y$}.
\end{theorem}
\begin{proof}
    Не нарушая общности рассуждений докажем для минимума.
    
    При $\delta y = 0$ из \eqref{issue15:FJ} следует, что $\delta \mathcal{J}(y_0, \delta y) = 0$. Пусть какая-либо допустимая $\delta y \neq 0$. Т.к. $y_0(x)$ -- cлабый экстремум $\mathcal{J}$, то $\exists \varepsilon > 0: \forall y(x) = y_0(x) + t \delta y(x): \| y(x) - y_0(x) \| \leq \varepsilon \mapsto \mathcal{J}(y_0) \leq \mathcal{J}(y)$. 
    Зафиксируем $\delta y \neq 0$. Т.к. $\| y(x) - y_0(x) \| = \| y_0 + t \delta y(x) - y_0(x)\| \leq \varepsilon$, то $\| t \cdot \delta y \| \leq \varepsilon $. Таким образом $t \in \left[ - \cfrac{\varepsilon}{\| \delta y(x)\|}; \cfrac{\varepsilon}{\| \delta y(x)\|}\right]$. 

    Т.к $y_0(x)$ -- локальный минимум, то $\mathcal{J}(y_0) \leq \mathcal{J}(y)$ или $\mathcal{J}(0) \leq \mathcal{J}(t)~\forall t \in \left[ - \cfrac{\varepsilon}{\| \delta y(x)\|}; \cfrac{\varepsilon}{\| \delta y(x)\|} \right]$. 

    Таким образом $\mathcal{J}(t)$ является непрерывно дифференцируемой функцией $t$, достигающий минимум при $t = 0$. Следовательно по теореме Ферма $\frac{d \mathcal{J}}{dt} (0) = 0 = \delta \mathcal{J}$
    
    Ввиду произвольности $\delta y$ теорема доказана. 
\end{proof}

\begin{lemma}[Основная лемма вариационного исчисления]
    Пусть $f(x) \in C^1_{[a;b]}$ и $\int \limits_a^b f(x) \cdot h(x) dx = 0,~ \forall h \in C^1_{[a;b]}$ и такой, что $h(a) = h(b) = 0$. Тогда $f(x) = 0 ,\, \forall x \in [a;b]$.
\end{lemma}

\begin{proof}
    От противного: пусть $\exists \bar{x} \in [a;b]: f(\bar{x}) \neq 0$. Тогда в силу непрерывности функции $f(x)$, $\exists x_0 \in (a, b): f(x_0) \neq 0 ~\&~ \exists \delta > 0$ такое, что $\forall x \in (x_0 - \delta; x_0 + \delta) \mapsto f(x) \neq 0$. 
    Для определенности рассмотрим $f(x) > 0, \, \forall x \in (x_0 - \delta; x_0 + \delta)$. 
    Если так случилось, что $(x_0 - \delta; x_0 + \delta) \not\subset [a;b]$, то уменьшим $\delta$, не нарушив при этом это условие: $f(x) > 0$ на отрезке ненулевой длины. 

    Обозначим $I_{\delta} = (x_0 - \delta; x_0 + \delta)$ и рассмотрим 
    \begin{equation} \label{issue15:SY}
        h_{\delta}(x) = \begin{cases}
            \left[(x - x_0 + \delta)(x - x_0 - \delta)\right]^2 & x \in I_{\delta}\\
            0 & x \not\in I_{\delta}
            \end{cases}
    \end{equation}
    
    Т.к. $h_{\delta}(x) > 0, \, \forall x \in I_{\delta}$, то $\int \limits_a^b f(x) \cdot h_{\delta}(x) dx  = \int \limits_{x_0 - \delta}^{x_0 + \delta} f(x) \cdot h_{\delta}(x) dx > 0$ 
    -- противоречие с условием $\int \limits_a^b f(x) \cdot h(x) dx = 0  \Rightarrow \nexists \bar{x} \in [a;b]: f(\bar{x}) \neq 0$. 
    \end{proof}

    \begin{remark}
    
        Лемма остаётся в силе, если в условии леммы $\int \limits_a^b f(x) \cdot h(x) dx = 0, \, \forall h \in C^n_{[a;b]}$ и $h^{(i)}(a) = h^{(i)}(b) = 0, \, i = \overline{0, n-1}$. В \eqref{issue15:SY} достаточно взять 

        \begin{equation} \label{issue15:SY}
            h_{\delta}(x) = \begin{cases}
                \left[(x - x_0 + \delta)(x - x_0 - \delta)\right]^{2n} & x \in I_{\delta}\\
                0 & x \not\in I_{\delta}
                \end{cases} 
                \text{  (Модифицированная лемма)}
        \end{equation}
    \end{remark}
    
    \subsection{Простейшие задачи вариационного исчисления}
    \subsubsection{Задача с закрепленными концами}

    Требуется найти экстремум функционала $\mathcal{J} (y) = \int \limits^b_a F(x, y(x), y'(x)) dx$ среди функций $y(x) \in C^1_{[a;b]}$ таких, что $y(a) = A$, $y(b) = B$, а где $A$ и $B$ являются заданными константами.
    Таким образом экстремум ищется на множестве $D = \{y(x): y(a) = A, y(b) = B \} \subset C^1_{[a;b]}$. Пусть $y_0(x)$ -- экстремум нашего функционала. 
    Обозначим за $H_{\delta}(y_0) = \{ \delta y(x) \in C^1_{[a;b]}$: $\delta y(a) = \delta y(b) = 0 \}$.
    Покажем, что $H_{\delta}(y_0)$ является множеством допустимых вариаций: $\forall \delta y(x) \in H_{\delta}(y_0)$ для $y(x) = y_0(x) + \delta y(x) \mapsto y(a) = A, y(b) = B \Rightarrow y_0(x) + \delta y \in D$
    
    \begin{theorem}
        Пусть $y_0(x) \in C^2_{[a;b]}$ является слабым экстремумом функцианала $\mathcal{J}$ на $D$. Тогда $y_0(x)$ удовлетворяет уравнению Эйлера-Лагранжа.
        
        \begin{equation} \label{issue15:IF}
            \cfrac{\partial F}{\partial y}(x, y_0(x), y'_0(x)) - \cfrac{d}{dx} \cfrac{\partial F}{\partial y'} (x, y_0(x), y'_0(x)) = 0
        \end{equation}

        Обозначение: $\cfrac{\partial F}{\partial y}(x, y_0(x), y'_0(x)) = \cfrac{\partial F}{\partial y}$; $\cfrac{\partial F}{\partial y'} (x, y_0(x), y'_0(x)) = \cfrac{\partial F}{\partial y'}$

    \end{theorem}

    \begin{proof}
        Т.к. $y_0(x)$ является слабым экстремумом, то $\forall \delta y(x) \in H_{\delta}(y_0) \mapsto$ $$\delta \mathcal{J} = { \Large \int \limits_a^b}  \left( \cfrac{\partial F}{\partial y} \delta y + \underbrace{\cfrac{\partial F}{\partial y'} \delta y'}_{\text{проинтегрируем по частям}}\right) dx = 0 \text{ (по основной теореме)} $$

        Концы закреплены:
        \[ \int \limits_a^b \frac{\partial F}{\partial y'} \delta y' dx = \frac{\partial F}{\partial y'} \delta y \bigg|_a^b - \int \limits_a^b \cfrac{d}{dx} \left( \cfrac{\partial F}{\partial y'}\right) \delta y dx = - \int \limits_a^b \cfrac{d}{dx} \left( \cfrac{\partial F}{\partial y'}\right) \delta y dx \]

        \[ \delta \mathcal{J} = \int \limits_a^b \left( \cfrac{\partial F}{\partial y} - \cfrac{d}{dx}\left(\cfrac{\partial F}{\partial y'}\right) \right) \delta y dx = 0 \;\; \forall \delta y \in H_{\delta}(y_0)\]

        Заметим, что $\forall \delta y \in H_{\delta}(y_0)$ и $\delta \mathcal{J}$ удовлетворяют уcловиям основной леммы $\Rightarrow $
        \[ \cfrac{\partial F}{\partial y} - \cfrac{d}{dx} \cfrac{\partial F}{\partial y'} = 0 \]
    \end{proof}
    
    \begin{remark}
        Требование $y_0(x) \in C^2_{[a;b]}$ является естественным, т.к. \eqref{issue15:IF} для $y_0(x)$ является ДУ второго порядка: $\cfrac{d}{dx} \cfrac{\partial F}{\partial y'}(x, y_0(x), y'_0(x)) = \cfrac{\partial^2 F}{\partial x \partial y'} + \cfrac{\partial^2 F}{\partial y \partial y'} y' + \cfrac{\partial^2 F}{\partial {y'}^2} y''$
    \end{remark}

    \begin{definition}
        Функцию $y_0(x)$, удовлетворяющую уравнению Эйлера и условиям множества $D$ будем называть \textbf{допустимой экстремалью}.
    \end{definition}

    \subsubsection{Функционалы, зависящие от вектор-функции}

    Рассмотрим 
    \begin{equation} \label{issue15:IF1}
        \mathcal{J}(\vec{y}) = \int \limits_a^b F(x, y_1(x), y_2(x), ..., y_n(x), y'_1(x), y'_2(x), ..., y'_n(x)) dx = \int \limits_a^b F(x, \vec{y}(x), \vec{y}\,'(x)) dx,
    \end{equation} где $\vec{y}(x) = \| y_1, ..., y_n\|$, $ \vec{y}\,'(x) = \| y'_1, ..., y'_n \|$
    
    Рассмотрим задачу с закрепленными концами: 
    \begin{equation} \label{issue15:IF2}
        \vec{y}(a) = \vec{A} = \| y_1(a), ..., y_n(a)\| = \| A_1, ..., A_n\|,\, \vec{y}(b) = \vec{B} = \| y_1(b), ..., y_n(b)\| = \| B_1, ..., B_n \|
    \end{equation}
    
    Считаем, что $F(x, y_1, ..., y_n, z_1, ..., z_n)$ - дважды непрерывно дифференцированна по совокупности переменных $a \leq x \leq b$, $-\infty < y_1, ..., y_n, z_1, ..., z_n < +\infty$. 
    Минимум \eqref{issue15:IF1}$\wedge$\eqref{issue15:IF2}, без нарушения общности, будем искать в классе $y_i(x) \in C^1_{[a;b]}, \, i = \overline{1, n}$. 
    Введём $|\vec{y}| = \sqrt{\sum \limits_{k=1}^{n} y^2_k}$ и $\|\vec{y}\| = \max \limits_{x \in [a;b]} |\vec{y}| + \max \limits_{x \in [a;b]} | \vec{y}' |$ 

    Множество допустимых вариаций $H_{\delta}(\vec{y}_0) = \{\delta \vec{y}(x) = \| \delta y_1 (x), ..., \delta y_n(x)\|: \delta \vec{y}(a) = \delta \vec{y}(b) = 0 \}$

    Пусть $\vec{y}_0(x) \in C^1_{[a;b]}$ -- слабый минимум ($\Rightarrow \delta \mathcal{J} = 0$), \eqref{issue15:IF1}$\wedge$\eqref{issue15:IF2}. При условии \eqref{issue15:IF2} получаем:
    
    \begin{multline*}
            \mathcal{J}(\vec{y}_0(x) + t \delta \vec{y}(x)) = \int \limits_a^b F(x, \vec{y}_0(x) + t\delta \vec{y}(x), \vec{y}\,'_0(x) + t(\delta \vec{y}(x))')dx = \mathcal{J}(t) = \\ 
            = \mathcal{J}(0) + \delta\mathcal{J}\cdot t + o(t) = \mathcal{J}(0) + \int \limits_a^b \left( \sum \limits_{k =1}^n \frac{\partial F}{\partial y_k} \delta y_k + \sum \limits_{k=1}^n \frac{\partial F}{\partial y'_k} (\delta y_k)' \right) dx \cdot t + o(t) = \\ 
            = \mathcal{J}(0) + \int \limits_a^b \left( \sum \limits_{k = 1}^n \left( \frac{\partial F}{\partial y_k} - \frac{d}{dx} \frac{\partial F}{\delta y'_k}  \right) \delta y_k \right) dx \cdot t
            + \sum \limits_{k = 1}^n \left( \frac{\partial F}{\partial y'_k}(b) \underbrace{(\delta y_k (b))}_{= 0} \right) \cdot t - \\ 
            - \sum \limits_{k = 1}^n \left( \frac{\partial F}{\partial y'_k}(a) \underbrace{(\delta y_k (a))}_{= 0} \right) \cdot t + o(t) 
    \end{multline*}

    \[ \delta \mathcal{J} = \int \limits_a^b \left(\sum \limits_{k = 1}^n \left( \frac{\partial F}{\partial y_k} - 
    \frac{d}{dx} \frac{\partial F}{\partial y'_k} \right) \delta y_k \right) dx = 0 \,\, 
    \text{\fbox{  $\forall \delta \vec{y}(x) \in H_{\delta}(\vec{y}_0) $}}
    \]

    Итак, $\delta \mathcal{J} = 0,\, \forall \delta \vec{y}(x) \in H_{\delta}(\vec{y}_0)$, тогда в силу произвольности выбора $\delta \vec{y}(x)$:
    пусть $\delta y_1 = \delta y_2 = ... = \delta y_{k - 1} = 0$, $\delta y_k  = ((x-a)(x-b))^2, \delta y_{k+1} = ... = \delta y_n = 0$. 
    
    Тогда 
    \[ \delta \mathcal{J} = 0 + \int \limits_a^b \sum \limits_{k = 1}^n \left( \frac{\partial F}{\partial y_k} - 
    \frac{d}{dx} \frac{\partial F}{\partial y'_k} \right) \delta y_k dx + 0 = 0 \] 
    $\Rightarrow$ Основная лемма $ \Rightarrow $ проходим все $k = \overline{1, n}$ 

    \[ \text{\fbox{$\frac{\partial F}{\partial y_k} - \frac{d}{dx} \frac{\partial F}{\partial y'_k} = 0, \,\, k = \overline{1, n} \text{ (Система уравнений Эйлера-Лагранжа)} $}}\]

    \subsubsection{Задача со свободными концами}
    Рассмотрим нахождение экстремума функцианала $\mathcal{J}(y) = \int \limits_a^b F(x, y(x), y'(x)) dx$ среди $y(x) \in C^1_{[a,b]}.$ В этом случае $D = C^1_{[a;b]}$, $H_{\delta}(y_0) = \{ \delta y(x) \in C^n_{[a;b]} \}$,
    т.е на $\delta y(x)$ не наложено условий. На $F$ наложены обычные условия: дважды непрерывной дифференцируемости всех переменных в совокупности. 

    Пусть $y_0(x) \in C^2_{[a;b]}$ является минимум функционала. $y = y_0 + t \cdot \delta y$

    \begin{multline*}
        \delta  \mathcal{J} = \int \limits_a^b \left( \frac{\partial F}{\partial y} \delta y + \frac{\partial F}{\partial y'} (\delta y)' \right) dx = \int \limits_a^b \left( \frac{\partial F}{\partial y} - \frac{d}{dx} \frac{\partial F}{\delta y'} \right) \delta y dx  + \frac{\partial F}{\partial y'}(y_0(b)) \delta y(b) - \\ 
        - \frac{\partial F}{\partial y'}(y_0(a)) \delta y(a) = 0
    \end{multline*}     
    По основной теореме $\forall \delta y(x) \in C^1_{[a;b]}$ 

    В силу произвольности $\delta y$:

    \[ \begin{cases}
        \cfrac{\partial F}{\partial y} - \cfrac{d}{dx} \cfrac{\partial F}{\partial y'} = 0, & (1)\\
        \cfrac{\partial F}{\partial y'} (b; y_0(b); y'_0(b)) = 0, & (2)\\
        \cfrac{\partial F}{\partial y'} (a; y_0(a); y'_0(a)) = 0. & (3)\\
    \end{cases}
    \]

    Таким образом, если $y_0(x) \in C^2_{[a;b]}$ является слабым экстремумом функцианала со свободными концами, то $y_0(x)$ удовлетворяет уравнению Эйлера (1) с граничными условиями (2 и 3)
