\documentclass[a4paper, 12pt]{article}
%%% Работа с русским языком
\usepackage{cmap}					% поиск в PDF
\usepackage{mathtext} 				% русские буквы в формулах
\usepackage[T2A]{fontenc}			% кодировка
\usepackage[utf8]{inputenc}			% кодировка исходного текста
\usepackage[russian]{babel}	% локализация и переносы

%%% Дополнительная работа с математикой
\usepackage{amsmath,amsfonts,amssymb,amsthm,mathtools} % AMS
\usepackage{icomma} % "Умная" запятая: $0,2$ --- число, $0, 2$ --- перечисление

%% Номера формул
%\mathtoolsset{showonlyrefs=true} % Показывать номера только у тех формул, на которые есть \eqref{} в тексте.
%\usepackage{leqno} % Немуреация формул слева

%% Шрифты
\usepackage{euscript}	 % Шрифт Евклид
\usepackage{mathrsfs} % Красивый матшрифт

%%% Свои команды
\DeclareMathOperator{\sgn}{\mathop{sgn}}

%% Поля
\usepackage[left=2cm,right=2cm,top=2cm,bottom=2cm,bindingoffset=0cm]{geometry}

%% Русские списки
\usepackage{enumitem}
\makeatletter
\AddEnumerateCounter{\asbuk}{\russian@alph}{щ}
\makeatother

%%% Работа с картинками
\usepackage{graphicx}  % Для вставки рисунков
\graphicspath{{images/}{images2/}}  % папки с картинками
\setlength\fboxsep{3pt} % Отступ рамки \fbox{} от рисунка
\setlength\fboxrule{1pt} % Толщина линий рамки \fbox{}
\usepackage{wrapfig} % Обтекание рисунков и таблиц текстом

%%% Работа с таблицами
\usepackage{array,tabularx,tabulary,booktabs} % Дополнительная работа с таблицами
\usepackage{longtable}  % Длинные таблицы
\usepackage{multirow} % Слияние строк в таблице

%% Красная строка
\setlength{\parindent}{2em}

%% Интервалы
\linespread{1}
\usepackage{multirow}

%% TikZ
\usepackage{tikz}
\usetikzlibrary{graphs,graphs.standard}

%% Верхний колонтитул
% \usepackage{fancyhdr}
% \pagestyle{fancy}

%% Перенос знаков в формулах (по Львовскому)
\newcommand*{\hm}[1]{#1\nobreak\discretionary{}
	{\hbox{$\mathsurround=0pt #1$}}{}}

%% дополнения
\usepackage{float} %Добавляет возможность работы с командой [H] которая улучшает расположение на странице
\usepackage{gensymb} %Красивые градусы
\usepackage{caption} % Пакет для подписей к рисункам, в частности, для работы caption*

% подключаем hyperref (для ссылок внутри  pdf)
\usepackage[unicode, pdftex]{hyperref}

%%% Теоремы
\theoremstyle{plain}                    % Это стиль по умолчанию, его можно не переопределять.
\renewcommand\qedsymbol{$\blacksquare$} % переопределение символа завершения доказательства

\newtheorem{theorem}{Теорема}[section] % Теорема (счетчик по секиям)
\newtheorem{proposition}{Утверждение}[section] % Утверждение (счетчик по секиям)
\newtheorem{definition}{Определение}[section] % Определение (счетчик по секиям)
\newtheorem{corollary}{Следствие}[theorem] % Следстиве (счетчик по теоремам)
\newtheorem{problem}{Задача}[section] % Задача (счетчик по секиям)
\newtheorem*{remark}{Примечание} % Примечание (можно переопределить, как Замечание)
\newtheorem{lemma}{Лемма}[section] % Лемма (счетчик по секиям)

\newtheorem{example}{Пример}[section] % Пример
\newtheorem{counterexample}{Контрпример}[section] % Контрпример
\newcommand{\defeq}{\stackrel{def}{=}} % по определению
\newcommand{\defarr}{\stackrel{def}{\Rightarrow}} % следует из определения

\makeatletter
\newcommand{\eqnum}{\refstepcounter{equation}\textup{\tagform@{\theequation}}}
\makeatother % создание метки и нумерация формулы одновременно

\newcommand{\deflimk}{\lim\limits_{k\rightarrow \infty}} % лимит при k -> бесконечности
\DeclareMathOperator{\Tr}{trace} % след матрицы
\usepackage{indentfirst}


\begin{document}

\subsection{Построение Жорданова базиса}

Для характеристического многочлена справедливо разложение:

\[\frac{1}{P_n(\lambda)} = \frac{1}{(\lambda - \lambda_1)^{k_1}...(\lambda - \lambda_m)^{k_m}} = \sum\limits_{i = 1}^{m}{\sum\limits_{l=1}^{k_1}{\frac{A^i_l}{(\lambda - \lambda_i)^l}}},~ A^i_l \in R^m\]

После сложения по внутренней сумме:

\[\frac{1}{P_n(\lambda)} = \frac{1}{(\lambda - \lambda_1)^{k_1}...(\lambda - \lambda_m)^{k_m}} = \frac{f_1(\lambda)}{(\lambda-\lambda_1)^{k_1}} + ... + \frac{f_s(\lambda)}{(\lambda-\lambda_s)^{k_s}} + ... + \frac{f_m(\lambda)}{(\lambda-\lambda_m)^{k_m}}\]

где $f_s(\lambda) - $ многочлен степени не выше $k_{s-1},~ s = \overline{1,m}$. Умножим на $P_n(\lambda):$

\[1 = Q_1(\lambda) + ... + Q_m(\lambda)\]
\begin{equation}
Q_s(\lambda) = f_s(\lambda)\cdot\frac{P_n(\lambda)}{(\lambda - \lambda_s)^{k_s}} = f_s(\lambda)\cdot(\lambda - \lambda_1)^{k_1} \cdot ... \cdot (\lambda - \lambda_{s-1})^{k_{s-1}} \cdot (\lambda - \lambda_{s+1})^{k_{s+1}} \cdot ... \cdot (\lambda - \lambda_{m})^{k_{m}}
\label{20_1}
\end{equation}


Рассмотрим множество квадратных матриц одного порядка. Это множество является ассоциативным кольцом с единицей, поэтому

\[A^n \cdot A^m = A^{n+m} = A^m \cdot A^n;~ A^0 \stackrel{def}{=} E\]

Определены коммутативное и ассоциативное сложение матриц. Нулевую матрицу примем за ноль. Согласно свойствам умножения матриц на числа:

\[A^k \cdot \alpha = \alpha A^k,~ \alpha A^k + \beta A^k = (\alpha + \beta) A^k\]

Таким образом правила приведения подобных членов аналогично правилу для многочленов.

\[A^k + (-1 \cdot A^k) = A^k + (-A^k) = 0\]

В качестве символа $x$ в определении многочлена можно взять квадратную матрицу $A$ и получить множество матричных многочленов $\{P_n(A)\}$

\[P_n(A) = a_0 E + a_1 A + ... + a_n A^n\]

На множестве $\{P_n(A)\}$ сложение и умножение определяются как обычные матричные действия, поэтому $\{P_n(A)\}$ является кольцом.

\begin{enumerate}
	\item $P_n(A) + P_m(A) = P_m(A) + P_n(A)$
	\item $(P_n(A) + P_m(A)) + P_s(A) = P_n(A) + (P_m(A) + P_s(A))$
	\item $P_n(A) \cdot P_m(A) = P_m(A) \cdot P_n(A)$
	\item $(P_n(A) \cdot P_m(A)) \cdot P_s(A) = P_n(A) \cdot (P_m(A) \cdot P_s(A))$
	\item $P_n(A) \cdot (P_m(A) + P_s(A)) = P_n(A) \cdot P_m(A) + P_n(A) \cdot P_s(A)$
\end{enumerate}

За ноль в этом множестве принимается нулевая матрица.

\begin{definition}
Отображение $\varphi$ кольца $K$ на кольцо $K'$ называется гомоморфизмом, если $\forall a \in K,~ \forall b \in K:$
\[\varphi(a+b) = \varphi(a) + \varphi(b);~ \varphi(ab) = \varphi(a) \cdot \varphi(b)\] 
\end{definition}

В отличие от изоморфизма гомоморфизм не обязательно является взаимно однозначным отображением, т.е. не предполагается, что образы $K$ заполняют все кольцо $K'$, и различным элементам из $K$ соответствуют разные элементы из $K'$.

В силу определения множеств $\{P_n(A)\}$ и $\{P_n(\lambda)\}$, кольца $\{P_n(A)\}$ и $\{P_n(\lambda)\}$ гомоморфны:
\[\varphi: \varphi(P_n(\lambda)) \longrightarrow P_n(A)\]
Неоднозначность отображения $\varphi$ возникает в силу того, что существуют такие квадратные матрицы $A \neq 0: \exists n \in N: A^m = 0~ \forall m \geq n$.

\begin{theorem}[Гамильтона-Кэли]
Пусть $P_n(\lambda) - $ характерестический многочлен матрицы $A$, тогда $P_n(A) = 0$.
\end{theorem}

В силу построения гомоморфизма между $\{P_n(A)\}$ и $\{P_n(\lambda)\}$ имеет место разложение:

\[P_n(A) = A^n + a_1 \cdot A^{n-1} + ... + a_o \cdot E = (A - \lambda_1 E)^{k_1} \cdot ... \cdot (A - \lambda_m E)^{k_m}\]

где $\lambda_1, ..., \lambda_m - $ корни $P_n(A)$.

Подействуем гомоморфизмом $\varphi$ на $(\ref{20_1}):$

\[E = Q_1(A) + ... + Q_m(A)\]
\begin{equation}
Q_s(A) = f_s(A)\cdot(A - \lambda_1 E)^{k_1} \cdot ... \cdot (A - \lambda_{s-1})^{k_{s-1}} \cdot (A - \lambda_{s+1})^{k_{s+1}} \cdot ... \cdot (A - \lambda_{m})^{k_{m}}
\label{20_2}
\end{equation}
\[Q_s(A) - \text{линейные преобразования}\]

Порядок сомножетелей в $(\ref{20_2})$ не важен, т.к. матрицы $(A - \lambda_s E)$ такого вида перестоновочны между собой.

Рассмотрим $Q_i(A)$. Покажем, что $\forall i,~j = \overline{1,m} \longmapsto $
\begin{equation}
Q_i(A) \cdot Q_j(A) = 
 \begin{cases}
   0, i \neq j\\
   Q_i^2, i = j
 \end{cases}
\text{и} ~ Q_i(A) = Q_i^2(A) 
\label{20_3}
\end{equation}
\begin{proof}
$Q_i(A) \cdot Q_j(A) = f_i(A) \cdot f_j(A) \cdot (A - \lambda_1 E)^{k_1} \cdot ... \cdot (A - \lambda_{i-1} E)^{k_{i-1}} \cdot (A - \lambda_{i-1} E)^{k_{i+1}} \cdot ... \cdot (A - \lambda_m E)^{k_m} \cdot (A - \lambda_1 E)^{k_1} \cdot ... \cdot (A - \lambda_{j-1} E)^{k_{j-1}} \cdot (A - \lambda_{j+1} E)^{k_{j+1}} \cdot ... \cdot (A - \lambda_m E)^{k_m} = M(A) \cdot P_n(A) = (\text{Теорема Гамильтона-Кэли}) = 0$

В силу $(\ref{20_2})$:
\[\vec{x} = E\vec{x} = Q_1(\vec{x}) + ... + Q_i(\vec{x}) + ... + Q_m(\vec{x})\]
\[\Rightarrow Q_i(\vec{x}) = (Q_i Q_1)(\vec{x}) + ... + (Q_i^2)(\vec{x}) + ... + (Q_i Q_n)(\vec{x}) = Q_i^2(\vec{x})\]
\end{proof}

Пусть $R_i = Im Q_i(A),~ i = \overline{1,m}~ - $ образ $Q_i(A)$. Из $(\ref{20_3})$ следует, что $R_i - $ инвариантное подпространство $A$. Тогда, если $\vec{x} \in R_i \rightarrow \exists \vec{y} \in A, Q_i(\vec{y}) = \vec{x}$, то $A(\vec{x}) = A(Q_i(\vec{y})) = (A \cdot Q_i)(\vec{y}) = (Q_i A)(\vec{y}) = Q_i(A(\vec{y})) \in R_i - $ инвариантное подпространство.

При доказательстве $(\ref{20_3})$ было получено, что:
\begin{equation}
\vec{x} = E\vec{x} = Q_1(\vec{x}) + ... + Q_i(\vec{x}) + ... + Q_m(\vec{x}) = \vec{x_1} + ... + \vec{x_i} + ... + \vec{x_m}
\label{20_4}
\end{equation}
где $x_i = Q_i(\vec{x}) \in R_i,~ i = \overline{1,m}$.

$(\ref{20_3})$ означает, что $R^n$ является суммой подпространств $R_i$. Покажем, что такое разложение единственно:
\begin{proof}
Предположим, что хотя бы для одного $k = \overline{1,m}~ \exists \vec{y_k} = Q_k(z_k) \neq \vec{x_k} : \vec{x} = \sum\limits_{k=1}^m{Q_k(\vec{z_k})} = \vec{y_1} + ... + \vec{y_i} + ... + \vec{y_m}$. Тогда $Q_i(\vec{x}) = \vec{x_i} = Q_i \left(\sum\limits_{k=1}^m{Q_k(\vec{z_k})}\right) \stackrel{Th~ \text{Г.К.}}{=} Q_i^2(\vec{z_i}) = Q_i(\vec{z_i}) = \vec{y_i} \Rightarrow \vec{x_i} = \vec{y_i}$
\end{proof}

Т.к. единственное разложение эквивалентно тому, что сумма подпространств прямая, то:
\[\vec{R^n} = R_1 \oplus R_2 \oplus ... \oplus R_m\]
Тогда $A$ в таком базисе будет иметь вид:

\begin{equation}
\begin{Vmatrix}
  A_1 &     &        & 0   \\
      & A_2 &        &     \\
      &     & \ddots &     \\
  0   &     &        & A_m
\end{Vmatrix}
\end{equation}

Подпространства $R_i$ называются корневыми подпространствами $\vec{R^n}$.

\begin{theorem}
$\forall s = \overline{1,m} : R_s = ker(A - \lambda_s E)^{k_s}~ \forall \vec{x} \in R_i \longmapsto (A - \lambda_i E)^{k_i} \vec{x} = 0$
\begin{proof}
Пусть $\vec{x} \in R_s \Rightarrow \exists \vec{y} \in R_s : \vec{x} = Q_i(\vec{y})$ в силу инвариантности $R_s$. Тогда $(A - \lambda_s E)^{k_s} \vec{x} = (A - \lambda_s E)^{k_s} \cdot f_s(A) \cdot (A - \lambda_1 E)^{k_1} \cdot ... \cdot (A - \lambda_{s-1} E)^{k_{s-1}} \cdot (A - \lambda_{s+1} E)^{k_{s+1}} \cdot (A - \lambda_m E)^{k_m} \vec{y} = f_s(A) \cdot P_n(A)\vec{y} = 0$
$ \Rightarrow R_s \subseteq ker(A - \lambda_s E) ^ {k_s}$.

Пусть $\vec{x} \in ker(A - \lambda_s E)^{k_s}$. Тогда $\forall j \neq s: Q_j(\vec{x}) = 0$, поскольку множитель $(A - \lambda_s E)^{k_s}$ как множитель входит в представление $Q_j$. Поэтому из $(\ref{20_4})$ в этом случае: $\vec{x} = 0 + ... + Q_s(\vec{x}) + ... + 0 \Rightarrow \vec{x} \in R_s \Rightarrow ker(A-\lambda_s E)^{k_s} \subseteq R_s$
\end{proof}
\end{theorem}

Рассмотрим структуру корневого подпространства. Покажем, что 
\[dim(R_s = ker(A - \lambda_s E)^{k_s}) = k_s\]

\begin{lemma}
Пусть $B$ является линейным преобразованием $\vec{R^n}$ и $R = ker(B^l),~ n < l$. Тогда, если $\exists \vec{x} \in R: B^{l-1} \vec{x} \neq 0$, то $dim R \geq l$.
\begin{proof}
Рассмотрим систему векторов $\vec{x}, B\vec{x}, ..., B^{l-1} \vec{x} \in R$. Ни один из векторов этой системы не равен нулю. Покажем, что эта система линейно независима. С этой целью рассмотрим нулевую линейную комбинацию этих векторов.
\begin{equation}
a_0 \vec{x} + a_1(B \vec{x}) + ... + a_{n-1}(B^{l-1} \vec{x}) = 0
\label{20_5}
\end{equation}
Подействуем последовательно $l - 1$ раз преобразованием $B$ на $(\ref{20_5})$:

\begin{equation*}
\begin{cases}
   a_0(B \vec{x}) + a_1 (B^2 \vec{x}) + ... + a_{n-2}(B^{l-1} \vec{x}) = 0 \\
   ... \\
   a_0(B^{l-2} \vec{x}) + a_1(B^{l-1} \vec{x}) + 0 + ... + 0 = 0 \\
   a_0 (B^{l-1} \vec{x}) = 0
\end{cases}
\end{equation*}

\[(B^{l-1} \vec{x}) \neq 0 ~\text{по условию} \Rightarrow a_0 = a_1 = ... = a_{l-1} = 0 \Rightarrow \text{Вектора ЛНЗ}\]

Таким образом в $R$ лежит как минимум $l$ ЛНЗ векторов, а значит базис в $R$ не может содержать меньше, чем $l$ векторов $\Rightarrow dim R \geq l$.

Было доказано, что пространства $R_i,~ i = \overline{1,s}$ образуют прямую сумму, равную $\vec{R^n}$, поэтому размерность $\vec{R^n}$ является суммой размерностей подпространств, которые составляют эту прямую сумму. Т.к. $k_1 + k_2 + ... + k_s = n$, то $\forall i \longmapsto dim R_i = k_i$, поскольку если $\exists j: dim R_j > k_j$, то тогда должно существовать $R_i$, у которого размерность меньше, чем $k_i$, что в силу леммы невозможно.
\end{proof}
\end{lemma}

\end{document}