\documentclass[a4paper, 12pt]{article}
%%% Работа с русским языком
\usepackage{cmap}					% поиск в PDF
\usepackage{mathtext} 				% русские буквы в формулах
\usepackage[T2A]{fontenc}			% кодировка
\usepackage[utf8]{inputenc}			% кодировка исходного текста
\usepackage[russian]{babel}	% локализация и переносы

%%% Дополнительная работа с математикой
\usepackage{amsmath,amsfonts,amssymb,amsthm,mathtools} % AMS
\usepackage{icomma} % "Умная" запятая: $0,2$ --- число, $0, 2$ --- перечисление

%% Номера формул
%\mathtoolsset{showonlyrefs=true} % Показывать номера только у тех формул, на которые есть \eqref{} в тексте.
%\usepackage{leqno} % Немуреация формул слева

%% Шрифты
\usepackage{euscript}	 % Шрифт Евклид
\usepackage{mathrsfs} % Красивый матшрифт

%%% Свои команды
\DeclareMathOperator{\sgn}{\mathop{sgn}}

%% Поля
\usepackage[left=2cm,right=2cm,top=2cm,bottom=2cm,bindingoffset=0cm]{geometry}

%% Русские списки
\usepackage{enumitem}
\makeatletter
\AddEnumerateCounter{\asbuk}{\russian@alph}{щ}
\makeatother

%%% Работа с картинками
\usepackage{graphicx}  % Для вставки рисунков
\graphicspath{{images/}{images2/}}  % папки с картинками
\setlength\fboxsep{3pt} % Отступ рамки \fbox{} от рисунка
\setlength\fboxrule{1pt} % Толщина линий рамки \fbox{}
\usepackage{wrapfig} % Обтекание рисунков и таблиц текстом

%%% Работа с таблицами
\usepackage{array,tabularx,tabulary,booktabs} % Дополнительная работа с таблицами
\usepackage{longtable}  % Длинные таблицы
\usepackage{multirow} % Слияние строк в таблице

%% Красная строка
\setlength{\parindent}{2em}

%% Интервалы
\linespread{1}
\usepackage{multirow}

%% TikZ
\usepackage{tikz}
\usetikzlibrary{graphs,graphs.standard}

%% Верхний колонтитул
% \usepackage{fancyhdr}
% \pagestyle{fancy}

%% Перенос знаков в формулах (по Львовскому)
\newcommand*{\hm}[1]{#1\nobreak\discretionary{}
	{\hbox{$\mathsurround=0pt #1$}}{}}

%% дополнения
\usepackage{float} %Добавляет возможность работы с командой [H] которая улучшает расположение на странице
\usepackage{gensymb} %Красивые градусы
\usepackage{caption} % Пакет для подписей к рисункам, в частности, для работы caption*

% подключаем hyperref (для ссылок внутри  pdf)
\usepackage[unicode, pdftex]{hyperref}

%%% Теоремы
\theoremstyle{plain}                    % Это стиль по умолчанию, его можно не переопределять.
\renewcommand\qedsymbol{$\blacksquare$} % переопределение символа завершения доказательства

\newtheorem{theorem}{Теорема}[section] % Теорема (счетчик по секиям)
\newtheorem{proposition}{Утверждение}[section] % Утверждение (счетчик по секиям)
\newtheorem{definition}{Определение}[section] % Определение (счетчик по секиям)
\newtheorem{corollary}{Следствие}[theorem] % Следстиве (счетчик по теоремам)
\newtheorem{problem}{Задача}[section] % Задача (счетчик по секиям)
\newtheorem*{remark}{Примечание} % Примечание (можно переопределить, как Замечание)
\newtheorem{lemma}{Лемма}[section] % Лемма (счетчик по секиям)

\newtheorem{example}{Пример}[section] % Пример
\newtheorem{counterexample}{Контрпример}[section] % Контрпример
\newcommand{\defeq}{\stackrel{def}{=}} % по определению
\newcommand{\defarr}{\stackrel{def}{\Rightarrow}} % следует из определения

\makeatletter
\newcommand{\eqnum}{\refstepcounter{equation}\textup{\tagform@{\theequation}}}
\makeatother % создание метки и нумерация формулы одновременно

\newcommand{\deflimk}{\lim\limits_{k\rightarrow \infty}} % лимит при k -> бесконечности
\DeclareMathOperator{\Tr}{trace} % след матрицы
\usepackage{indentfirst}


\begin{document}
    \section{Билет 6. Первые интегралы автономных систем. Линейные однородные уравнения в частных производных первого порядка}
    \subsection{Общее решение линейного однородного уравнения в частных производных первого порядка}

    \begin{definition}
        Рассмотрим уравнение 
        \begin{equation}
            \sum \limits_{i = 1}^{n} f^{i} \left( \overrightarrow{x}, u \right) \frac{\partial u}{\partial x^{i}} = F \left( \overrightarrow{x}, u \right)
            \label{part_eq_1}
        \end{equation}

        Функция $u \left( \overrightarrow{x} \right)$ называется решением уравнения $\eqref{part_eq_1}$, если $u \left( \overrightarrow{x} \right) \in C^{1} \left( \mathbb{R}^n \right)$ и после подстановки в $\eqref{part_eq_1}$ получаается тождество, причём $f^{i} \left( \overrightarrow{x}, u \right) \in C^{1} \left( \mathbb{R}^n \times \mathbb{R} \right)$ -- некоторые заданные функции. Уравнение $\eqref{part_eq_1}$ называется квазилинейным уравнением в частных производных первого порядка. 
    \end{definition}
    
    \begin{definition}
        Рассмотрим систему ДУ:
        \begin{equation}
            \begin{cases}
                \dot{x}^1 = f^1 \left( \overrightarrow{x}, u \right) \\
                \dots                                                \\
                \dot{x}^n = f^n \left( \overrightarrow{x}, u \right)
            \end{cases}
            \label{part_eq_2}
        \end{equation}
    
        Система $\eqref{part_eq_2}$ называется характеристической системой уравнения $\eqref{part_eq_1}$, а $\overrightarrow{x} \left( t \right) $ -- фазовые кривые $\eqref{part_eq_2}$ -- называются характеристиками $\eqref{part_eq_1}$.
    \end{definition}
   
    Основное свойство характеристик состоит в том, что уравнение для $u \left( \overrightarrow{x} \right) $ в силу $\eqref{part_eq_2}$ имеет вид 
    \begin{equation*}
        \frac{du}{dt} = F \left( \overrightarrow{x} \left( t \right) , u \right) \; -
    \end{equation*}
    обыкновенное ДУ. Действительно, пусть $u$ -- решение $\eqref{part_eq_1}$, тогда 
    \begin{equation*}
        \frac{du}{dt} = \sum \limits_{i = 1}^{n} \frac{\partial u}{\partial x^i} \frac{\partial x^i}{\partial t} = \sum \limits_{i = 1}^{n} \frac{\partial u}{\partial x^i} f^i = F \left( \overrightarrow{x} \left( t \right) , u \right) 
    \end{equation*}

    Будем рассматривать уравнения вида
    \begin{equation}
        \sum \limits_{i = 1}^{n} f^{i} \left( \overrightarrow{x}, u \right) \frac{\partial u}{\partial x^{i}} = F \left( \overrightarrow{x}, u \right) 
        \label{part_eq_3}
    \end{equation}

    \begin{definition}
        Уравнения вида $\eqref{part_eq_3}$ называются линейными однородными уравнениями первого порядка в частных производных. Характеристической системой для $\eqref{part_eq_3}$ будем называть систему вида
        \begin{equation}
            \begin{cases}
                \dot{x}^1 = f^1 \left( \overrightarrow{x} \right) \\
                \dots                                             \\
                \dot{x}^n = f^n \left( \overrightarrow{x} \right) 
            \end{cases}
            \label{part_eq_4}
        \end{equation}
    \end{definition}

    \begin{theorem}
        Пусть $\nu_1 \left( \overrightarrow{x} \right) = C_1, \dots, \nu_k \left( \overrightarrow{x} \right) = C_k$ являются независимыми первыми интегралами системы $\eqref{part_eq_4}$. Тогда функция $u \left( \overrightarrow{x} \right) = F \left( \nu_1 \left( \overrightarrow{x} \right), \dots, \nu_k \left( \overrightarrow{x} \right) \right) $ является решением уравнения $\eqref{part_eq_3}$.
    \end{theorem}
    \begin{proof}
        Запишем уравнение $\eqref{part_eq_3}$ следующим способом:
        \begin{equation*}
            \sum \limits_{i = 1}^{n} f^{i} \left( \overrightarrow{x} \right)  \frac{\partial u}{\partial x^{i}} = \sum \limits_{i = 1}^{n} f^{i} \left( \overrightarrow{x} \right)  \sum \limits_{l = 1}^{k} \frac{\partial u}{\partial \nu_l} \frac{\partial \nu_l}{\partial x^{i}} = \sum \limits_{l = 1}^{n} \frac{\partial u}{\partial \nu_l} \sum \limits_{i = 1}^{k} f^{i} \left( \overrightarrow{x} \right)  \frac{\partial \nu_l}{\partial x^{i}} = 0
        \end{equation*}

        Получили тождество, значит $u \left( \overrightarrow{x} \right) = F \left( \nu_1 \left( \overrightarrow{x} \right), \dots, \nu_k \left( \overrightarrow{x} \right) \right)$ действительно решение уравнения $\eqref{part_eq_3}$.
    \end{proof}

    \begin{theorem}
        Пусть функция $u \left( \overrightarrow{x} \right) = F \left( \nu_1 \left( \overrightarrow{x} \right), \dots, \nu_k \left( \overrightarrow{x} \right) \right)$ является решением уравнения $\eqref{part_eq_3}$. Тогда $\nu_1 \left( \overrightarrow{x} \right) = C_1, \dots, \nu_k \left( \overrightarrow{x} \right) = C_k$ являются независимыми первыми интегралами системы $\eqref{part_eq_4}$. 
    \end{theorem}
    \begin{proof}
        Так как $u \left( \overrightarrow{x} \right)$ -- решение, то 
        \begin{equation*}
            \sum \limits_{i = 1}^{n} f^i \frac{\partial u}{\partial x^i} = 0
        \end{equation*}

        Значит $u \left( \overrightarrow{x} \right)$ -- первый интеграл системы $\eqref{part_eq_4}$ по критерию первого интеграла. Этот первый интеграл может зависеть только от независимых переменных $\nu_1 \left( \overrightarrow{x} \right), \dots, \nu_k \left( \overrightarrow{x} \right)$, причём $u \left( \nu_1 \left( \overrightarrow{x} \right), \dots, \nu_k \left( \overrightarrow{x} \right) \right) = C_0$, где $\nu_1 \left( \overrightarrow{x} \right), \dots, \nu_k \left( \overrightarrow{x} \right)$ -- первые интегралы системы $\eqref{part_eq_4}$.
    \end{proof}

    \subsection{Задача Коши для уравнения в частных производных первого порядка}

    Пусть $S: \; g \left( \overrightarrow{x} \right) = 0$ -- гладкая поверхность в $\mathbb{R}^n$ и 
    \begin{equation*}
        \triangledown g = \bigg| \bigg| \frac{\partial g}{\partial x^{1}}, \dots, \frac{\partial g}{\partial x^{n}} \bigg| \bigg| \neq \overrightarrow{0}
    \end{equation*}

    \begin{definition}
        Точка $\overrightarrow{a} \in S$ называется некритической точкой поверхности, если в системе $\eqref{part_eq_4}$ $\overrightarrow{f} \left( \overrightarrow{a} \right) \neq \overrightarrow{0}$ и $ \left( \triangledown g \left( \overrightarrow{a} \right), \overrightarrow{f} \left( \overrightarrow{a} \right) \right) \neq 0$  (фазовые траектории не лежат на $S$).
    \end{definition}

    Пусть на $S$ задана функция $U_0 \left( \overrightarrow{x} \right)$ и $U_0 \left( \overrightarrow{x} \right) \in C^1 \left( \mathbb{R}^n \right)$.

    Задача Коши: найти такое решение $u \left( \overrightarrow{x} \right)$ уравнения $\eqref{part_eq_3}$, что $u \left( \overrightarrow{x} \right) = U_0 \left( \overrightarrow{x} \right) \; \forall \overrightarrow{x} \in S$.
    
    \begin{theorem}
        Пусть на гладкой поверхности $S$ задана непрерывно дифференцируемая функция $U_0 \left( \overrightarrow{x} \right)$. Тогда если точка $\overrightarrow{a_0} \in S$ является некритической, то существует окрестность этой точки, в которой решение задачи Коши $u \left( \overrightarrow{x} \right) = U_0 \left( \overrightarrow{x} \right)$ для уравнения $\eqref{part_eq_3}$ существует и единственно.
    \end{theorem}
    \begin{proof}
        Запишем параметризацию поверхности $S$ в $\mathbb{R}^n$: $x^i = \varphi^i \left( u_1, \dots, u_{n - 1} \right), \; i = \overline{1, n}$. Поверхность $S$ может быть параметризована, поскольку требование $\triangledown g \neq \overrightarrow{0}$ означает, что 
        \begin{equation*}
            rank \bigg| \bigg| \frac{\partial g}{\partial x^{1}}, \dots, \frac{\partial g}{\partial x^{n}} \bigg| \bigg| = 1 \neq 0.
        \end{equation*}
        Значит по теореме о неявной функции параметризация поверхности $S$ задаётся следующим образом:
        \begin{equation*}
            \begin{cases}
                x^1 = \varphi \left( x^2, \dots, x^n \right) \\
                x^2 = x^2                                    \\
                \dots                                        \\
                x^n = x^n
            \end{cases}
        \end{equation*}
        Значит $u \left( \overrightarrow{x} \right) = u \left( x^1, \dots, x^n \right) = u \left( \varphi \left( x^2, \dots, x^n \right), \dots, x^n \right) = U_0 \left( x^2, \dots, x^n \right)$.

        Так как $\overrightarrow{a_0} \in S$ является некритической по условию, то существует такая окрестность этой точки $\mathcal{U}  \left( \overrightarrow{a_0} \right)$, где существуют $n - 1$ независимых первых интегралов системы $\eqref{part_eq_4}$: $\nu_1 \left( \overrightarrow{x} \right) = C_1, \dots, \nu_{n - 1} \left( \overrightarrow{x} \right) = C_{n - 1}$, а общее решение уравнения $\eqref{part_eq_3}$ $u = u \left( \nu_1 \left( \overrightarrow{x} \right), \dots, \nu_{n - 1} \left( \overrightarrow{x} \right) \right)$. 

        Рассмотрим систему уравнений относительно $x^1, \dots, x^n$:
        \begin{equation}
            \begin{cases}
                \nu_1 \left( \overrightarrow{x} \right) = C_1             \\
                \dots                                                     \\
                \nu_{n - 1} \left( \overrightarrow{x} \right) = C_{n - 1} \\
                g \left( \overrightarrow{x} \right) = 0
            \end{cases}
            \label{part_eq_5}
        \end{equation}
        Допустим, что систему удалось разрешить и была получена параметризация поверхности $S$ $g \left( \overrightarrow{x} \right) = 0$:
        \begin{equation*}
            \begin{cases}
                x^1_S = x^1_S \left( C_1, \dots, C^{n - 1} \right) \\
                \dots                                              \\
                x^n_S = x^n_S \left( C_1, \dots, C^{n - 1} \right)
            \end{cases}
        \end{equation*}

        Рассмотрим
        \begin{equation*}
            J \left( \overrightarrow{a_0} \right) =
            \begin{vmatrix}
                \frac{\partial \nu_1}{\partial x^1} & \dots & \frac{\partial \nu_1}{\partial x^n}             \\
                \dots                                                                                         \\
                \frac{\partial \nu_{n - 1}}{\partial x^1} & \dots & \frac{\partial \nu_{n - 1}}{\partial x^n} \\
                \frac{\partial g}{\partial x^1} & \dots & \frac{\partial g}{\partial x^n}                     \\
            \end{vmatrix}  \left( \overrightarrow{a_0} \right)
        \end{equation*}

        Так как $\overrightarrow{f} \left( \overrightarrow{a_0} \right) \neq 0$, то умножим $i$-ый столбец определителя $J \left( \overrightarrow{a_0} \right)$ на $r^i = f^i \left( \overrightarrow{a_0} \right)$ и прибавим к первому столбцу все те столбцы, которые умножились $r^i = f^i \left( \overrightarrow{a_0} \right) \neq 0$. Учтём, что $\forall i = \overline{1, n - 1}$:
        \begin{equation*}
            \sum \limits_{j = 1}^n \frac{\partial \nu_i}{\partial x^j} \left( \overrightarrow{a_0} \right) f^j \left( \overrightarrow{a_0} \right) = 0
        \end{equation*}
        так как $\nu_i$ -- первый интеграл. 
        Преобразованный определитель будет выглядеть следующим образом:
        \begin{equation*}
            J' \left( \overrightarrow{a_0} \right) =
            \begin{vmatrix}
                0 & \frac{\partial \nu_1}{\partial x^2} r^2 & \dots & \frac{\partial \nu_1}{\partial x^n} r^n                                           \\
                \dots                                                                                                                                   \\
                0 & \frac{\partial \nu_{n - 1}}{\partial x^2} r^2 & \dots & \frac{\partial \nu_{n - 1}}{\partial x^n} r^n                               \\
                 \left( \triangledown g, \overrightarrow{f} \right) & \frac{\partial g}{\partial x^2} r^2 & \dots & \frac{\partial g}{\partial x^n} r^n \\
            \end{vmatrix}  \left( \overrightarrow{a_0} \right) = \left( -1 \right)^{n + 1} \left( \triangledown g, \overrightarrow{f} \right)
            \begin{vmatrix}
                \frac{\partial \nu_1}{\partial x^2} r^2 & \dots & \frac{\partial \nu_1}{\partial x^n} r^n             \\
                \dots                                                                                                 \\
                \frac{\partial \nu_{n - 1}}{\partial x^2} r^2 & \dots & \frac{\partial \nu_{n - 1}}{\partial x^n} r^n \\
            \end{vmatrix} \neq 0
        \end{equation*}

        Утверждение справедливо, так как $ \left( \triangledown g, \overrightarrow{f} \right) \neq 0$ в нехарактеристической точке $\overrightarrow{a_0}$ и
        \begin{equation*}
            rank
            \begin{Vmatrix}
                \frac{\partial \nu_1}{\partial x^2} & \dots & \frac{\partial \nu_1}{\partial x^n}             \\
                \dots                                                                                         \\
                \frac{\partial \nu_{n - 1}}{\partial x^2} & \dots & \frac{\partial \nu_{n - 1}}{\partial x^n} \\
            \end{Vmatrix} = n - 1
        \end{equation*}
        так как первые интегралы функционально независимы.

        Таким образом в силу непрерывности рассматриваемых функций существует окрестность $\mathcal{U} \left( a_0 \right)$ в которой исходный определитель
        \begin{equation*}
            J \left( \overrightarrow{a_0} \right) =
            \begin{vmatrix}
                \frac{\partial \nu_1}{\partial x^1} & \dots & \frac{\partial \nu_1}{\partial x^n}             \\
                \dots                                                                                         \\
                \frac{\partial \nu_{n - 1}}{\partial x^1} & \dots & \frac{\partial \nu_{n - 1}}{\partial x^n} \\
                \frac{\partial g}{\partial x^1} & \dots & \frac{\partial g}{\partial x^n}                     \\
            \end{vmatrix} \neq 0,
        \end{equation*}
        то есть определитель матрицы Якоби исходной системы $\eqref{part_eq_5}$ не равен нулю. Тогда по теореме о системе неявных функций система однозначно разрешима и существуют единственным образом определённные функции $x^1_S = x^1_S \left( C_1, \dots, C^{n - 1} \right), \dots, x^2_S = x^2_S \left( C_1, \dots, C^{n - 1} \right)$, а значит $u = u \left( x^1_S \left( C_1, \dots, C^{n - 1} \right), \dots, x^n_S \left( C_1, \dots, C^{n - 1} \right) \right)$ является решением уравнения $\eqref{part_eq_3}$ и $u \left( \overrightarrow{x_S} \right) = U_0 \left( \overrightarrow{x} \right) \forall \overrightarrow{x} \in S$. Единственноость следует из однозначности решения.

    \end{proof}

    Рассмотрим уравнение
    \begin{equation}
        a(x, y) \frac{\partial z}{\partial x} + b(x, y) \frac{\partial z}{\partial y} + c(x, y) z = f(x, y)
        \label{part_eq_6}
    \end{equation}
    Функция $z(x, y)$ -- искомая функция, а функции $a(x, y), \; b(x, y), \; c(x, y)$ непрерывно дифференцируемы в некоторой области $D$. Имеется кривая 
    \begin{equation*}
        \gamma = 
        \begin{cases}
            x = \varphi(s) \\
            y = \psi(s)
        \end{cases}, \; s \in I = [s_1, s_2],
    \end{equation*}
    которая является непрерывно дифференцируемой в $I$ и $(\varphi'(s), \psi'(s)) \neq (0, 0) \; \forall s \in I$. На кривой $\gamma$ задано значение функции $z \big|_{\gamma} = h(s)$, то есть $z(\varphi(s), \psi(s)) = h(s)$ и $h(s)$ непрерывно дифференцируемая функция при $s \in I$. 

    Характеристическая система для уравнения $\eqref{part_eq_6}$ имеет вид
    \begin{equation}
        \begin{cases}
            \dot{x} = a(x, y) \\
            \dot{y} = b(x, y)
        \end{cases}
        \label{part_eq_7}
    \end{equation}

    \begin{theorem}
        Пусть кривая $\gamma$ в каждой своей точке не касается характеристик. Тогда задача Коши для $\eqref{part_eq_6}$ и $\eqref{part_eq_7}$ однозначно разрешима в некоторой окрестности кривой $\gamma$.
    \end{theorem}
    \begin{proof}
        Касательным вектором к фазовым траекторям $\eqref{part_eq_7}$ является вектор $\overrightarrow{\varphi} = \left( a(x, y), b(x, y) \right)$, поэтому если кривая $\gamma$ в каждой своей точке не касается фазовых характеристик, то $\overrightarrow{\varphi} \nparallel \overrightarrow{\tau} = (\varphi'(s), \psi`(s))$, а значит
        \begin{equation}
            \begin{vmatrix}
                a(\varphi(s), \psi(s)) & \varphi'(s) \\
                b(\varphi(s), \psi(s)) & \psi'(s)
            \end{vmatrix} \neq 0\; \forall s \in I
            \label{part_eq_8}
        \end{equation}
        
        Выпустим из каждой точки кривой $\gamma$ характеристику, то есть решим систему $\eqref{part_eq_7}$ с начальными условиями $x \big|_{t = 0} = \varphi(s), y \big|_{t = 0} = \psi(s)$. Пусть $x = x(t, s), \; y = y(t, s)$ -- некоторые решения системы.

        Уравнение $\eqref{part_eq_6}$ в силу системы $\eqref{part_eq_7}$ имеет вид $\frac{dz}{dt} + cz = f$. Поставим задачу Коши для этого уравнения с $z \big|_{t = 0} = h(s)$. По основной теореме и теореме о непрерывной зависимости решения от параметра (от начальных данных) существует решение поставленной задачи $z = \omega (t, s)$ -- непрерывно дифференцируемая функция в $G \subset D$. На соотношения $x = x(t, s), \; y = y(t, s)$ можно смотреть как на систему уравнений относительно $t$ и $s$, выразим их через $x$ и $y$.

        Так как
        \begin{equation*}
            I(t, s) =
            \begin{vmatrix}
                \frac{\partial x}{\partial t} & \frac{\partial x}{\partial s} \\
                \frac{\partial y}{\partial t} & \frac{\partial y}{\partial s}
            \end{vmatrix} =
            \begin{vmatrix}
                a(x(t, s), y(t, s)) & \frac{\partial x}{\partial s}(t, s) \\
                b(x(t, s), y(t, s)) & \frac{\partial y}{\partial s}(t, s)
            \end{vmatrix}
        \end{equation*}

        \begin{equation*}
            I(0, s) = 
            \begin{vmatrix}
                a(\varphi(s), \psi(s)) & \varphi'(s) \\
                b(\varphi(s), \psi(s)) & \psi'(s)
            \end{vmatrix} \neq 0 \; \forall s \in I,
        \end{equation*}
        поскольку $I(t, s)$ -- непрерывная от $t$ и $s$ функция. Тогда
        \begin{equation*}
            I(0, s) =
            \begin{vmatrix}
                \frac{\partial x}{\partial s} & \frac{\partial y}{\partial s} \\
                \frac{\partial x}{\partial t} & \frac{\partial y}{\partial t}
            \end{vmatrix} =
            \begin{vmatrix}
                \varphi'(s) & \psi'(s) \\
                a(x, y) & b(x, y)
            \end{vmatrix} \bigg|_{(x, y) \in \gamma} \neq 0.
        \end{equation*}

        Поэтому соществует окрестность кривой $\gamma$, где $I(t, s) \neq 0$.



    \end{proof}

\end{document}
