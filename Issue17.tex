\documentclass[a4paper, 12pt]{article}
%%% Работа с русским языком
\usepackage{cmap}					% поиск в PDF
\usepackage{mathtext} 				% русские буквы в формулах
\usepackage[T2A]{fontenc}			% кодировка
\usepackage[utf8]{inputenc}			% кодировка исходного текста
\usepackage[russian]{babel}	% локализация и переносы

%%% Дополнительная работа с математикой
\usepackage{amsmath,amsfonts,amssymb,amsthm,mathtools} % AMS
\usepackage{icomma} % "Умная" запятая: $0,2$ --- число, $0, 2$ --- перечисление

%% Номера формул
%\mathtoolsset{showonlyrefs=true} % Показывать номера только у тех формул, на которые есть \eqref{} в тексте.
%\usepackage{leqno} % Немуреация формул слева

%% Шрифты
\usepackage{euscript}	 % Шрифт Евклид
\usepackage{mathrsfs} % Красивый матшрифт

%%% Свои команды
\DeclareMathOperator{\sgn}{\mathop{sgn}}

%% Поля
\usepackage[left=2cm,right=2cm,top=2cm,bottom=2cm,bindingoffset=0cm]{geometry}

%% Русские списки
\usepackage{enumitem}
\makeatletter
\AddEnumerateCounter{\asbuk}{\russian@alph}{щ}
\makeatother

%%% Работа с картинками
\usepackage{graphicx}  % Для вставки рисунков
\graphicspath{{images/}{images2/}}  % папки с картинками
\setlength\fboxsep{3pt} % Отступ рамки \fbox{} от рисунка
\setlength\fboxrule{1pt} % Толщина линий рамки \fbox{}
\usepackage{wrapfig} % Обтекание рисунков и таблиц текстом

%%% Работа с таблицами
\usepackage{array,tabularx,tabulary,booktabs} % Дополнительная работа с таблицами
\usepackage{longtable}  % Длинные таблицы
\usepackage{multirow} % Слияние строк в таблице

%% Красная строка
\setlength{\parindent}{2em}

%% Интервалы
\linespread{1}
\usepackage{multirow}

%% TikZ
\usepackage{tikz}
\usetikzlibrary{graphs,graphs.standard}

%% Верхний колонтитул
% \usepackage{fancyhdr}
% \pagestyle{fancy}

%% Перенос знаков в формулах (по Львовскому)
\newcommand*{\hm}[1]{#1\nobreak\discretionary{}
	{\hbox{$\mathsurround=0pt #1$}}{}}

%% дополнения
\usepackage{float} %Добавляет возможность работы с командой [H] которая улучшает расположение на странице
\usepackage{gensymb} %Красивые градусы
\usepackage{caption} % Пакет для подписей к рисункам, в частности, для работы caption*

% подключаем hyperref (для ссылок внутри  pdf)
\usepackage[unicode, pdftex]{hyperref}

%%% Теоремы
\theoremstyle{plain}                    % Это стиль по умолчанию, его можно не переопределять.
\renewcommand\qedsymbol{$\blacksquare$} % переопределение символа завершения доказательства

\newtheorem{theorem}{Теорема}[section] % Теорема (счетчик по секиям)
\newtheorem{proposition}{Утверждение}[section] % Утверждение (счетчик по секиям)
\newtheorem{definition}{Определение}[section] % Определение (счетчик по секиям)
\newtheorem{corollary}{Следствие}[theorem] % Следстиве (счетчик по теоремам)
\newtheorem{problem}{Задача}[section] % Задача (счетчик по секиям)
\newtheorem*{remark}{Примечание} % Примечание (можно переопределить, как Замечание)
\newtheorem{lemma}{Лемма}[section] % Лемма (счетчик по секиям)

\newtheorem{example}{Пример}[section] % Пример
\newtheorem{counterexample}{Контрпример}[section] % Контрпример
\newcommand{\defeq}{\stackrel{def}{=}} % по определению
\newcommand{\defarr}{\stackrel{def}{\Rightarrow}} % следует из определения

\makeatletter
\newcommand{\eqnum}{\refstepcounter{equation}\textup{\tagform@{\theequation}}}
\makeatother % создание метки и нумерация формулы одновременно

\newcommand{\deflimk}{\lim\limits_{k\rightarrow \infty}} % лимит при k -> бесконечности
\DeclareMathOperator{\Tr}{trace} % след матрицы
\usepackage{indentfirst}


\begin{document}
    \section{Билет 6. Первые интегралы автономных систем. Линейные однородные уравнения в частных производных первого порядка}
    \subsection{Общее решение линейного однородного уравнения в частных производных первого порядка}

    \begin{definition}
        Рассмотрим уравнение 
        \begin{equation}
            \sum \limits_{i = 1}^{n} f^{i} (\overrightarrow{x}, u) \frac{\partial u}{\partial x^{i}} = F(\overrightarrow{x}, u)
            \label{part_eq_1}
        \end{equation}

        Функция $u(\overrightarrow{x})$ называется решением уравнения $\eqref{part_eq_1}$, если $u(\overrightarrow{x}) \in C^{1}(\mathbb{R}^n)$ и после подстановки в $\eqref{part_eq_1}$ получаается тождество, причём $f^{i} (\overrightarrow{x}, u) \in C^{1}(\mathbb{R}^n \times \mathbb{R})$ -- некоторые заданные функции. Уравнение $\eqref{part_eq_1}$ называется квазилинейным уравнением в частных производных первого порядка. 
    \end{definition}
    
    \begin{definition}
        Рассмотрим систему ДУ:
        \begin{equation}
            \begin{cases}
                \dot{x}^1 = f^1(\overrightarrow{x}, u) \\
                \dots                                  \\
                \dot{x}^n = f^n(\overrightarrow{x}, u)
            \end{cases}
            \label{part_eq_2}
        \end{equation}
    
        Система $\eqref{part_eq_2}$ называется характеристической системой уравнения $\eqref{part_eq_1}$, а $\overrightarrow{x}(t)$ -- фазовые кривые $\eqref{part_eq_2}$ -- называются характеристиками $\eqref{part_eq_1}$.
    \end{definition}
   
    Основное свойство характеристик состоит в том, что уравнение для $u(\overrightarrow{x})$ в силу $\eqref{part_eq_2}$ имеет вид 
    \begin{equation*}
        \frac{du}{dt} = F(\overrightarrow{x}(t), u) \; -
    \end{equation*}
    обыкновенное ДУ. Действительно, пусть $u$ -- решение $\eqref{part_eq_1}$, тогда 
    \begin{equation*}
        \frac{du}{dt} =  \sum \limits_{i = 1}^{n} \frac{\partial u}{\partial x^i} \frac{\partial x^i}{\partial t} = \sum \limits_{i = 1}^{n} \frac{\partial u}{\partial x^i} f^i = F(\overrightarrow{x}(t), u)
    \end{equation*}

    Будем рассматривать уравнения вида
    \begin{equation}
        \sum \limits_{i = 1}^{n} f^{i} (\overrightarrow{x}, u) \frac{\partial u}{\partial x^{i}} = F(\overrightarrow{x}, u)
        \label{part_eq_3}
    \end{equation}

    \begin{definition}
        Уравнения вида $\eqref{part_eq_3}$ называются линейными однородными уравнениями первого порядка в частных производных. Характеристической системой для $\eqref{part_eq_3}$ будем называть систему вида
        \begin{equation}
            \begin{cases}
                \dot{x}^1 = f^1(\overrightarrow{x}) \\
                \dots                               \\
                \dot{x}^n = f^n(\overrightarrow{x})
            \end{cases}
            \label{part_eq_4}
        \end{equation}
    \end{definition}

    \begin{theorem}
        Пусть $\nu_1(\overrightarrow{x}) = C_1, \dots, \nu_k(\overrightarrow{x}) = C_k$ являются независимыми первыми интегралами системы $\eqref{part_eq_4}$. Тогда функция $u(\overrightarrow{x}) = F(\nu_1(\overrightarrow{x}), \dots, \nu_k(\overrightarrow{x}))$ является решением уравнения $\eqref{part_eq_3}$.
    \end{theorem}
    \begin{proof}
        Запишем уравнение $\eqref{part_eq_3}$ следующим способом:
        \begin{equation*}
            \sum \limits_{i = 1}^{n} f^{i} (\overrightarrow{x}) \frac{\partial u}{\partial x^{i}} = \sum \limits_{i = 1}^{n} f^{i} (\overrightarrow{x}) \sum \limits_{l = 1}^{k} \frac{\partial u}{\partial \nu_l} \frac{\partial \nu_l}{\partial x^{i}} = \sum \limits_{l = 1}^{n} \frac{\partial u}{\partial \nu_l} \sum \limits_{i = 1}^{k} f^{i} (\overrightarrow{x}) \frac{\partial \nu_l}{\partial x^{i}} = 0
        \end{equation*}

        Получили тождество, значит $u(\overrightarrow{x}) = F(\nu_1(\overrightarrow{x}), \dots, \nu_k(\overrightarrow{x}))$ действительно решение уравнения $\eqref{part_eq_3}$.
    \end{proof}

    \begin{theorem}
        Пусть функция $u(\overrightarrow{x}) = F(\nu_1(\overrightarrow{x}), \dots, \nu_k(\overrightarrow{x}))$ является решением уравнения $\eqref{part_eq_3}$. Тогда $\nu_1(\overrightarrow{x}) = C_1, \dots, \nu_k(\overrightarrow{x}) = C_k$ являются независимыми первыми интегралами системы $\eqref{part_eq_4}$. 
    \end{theorem}
    \begin{proof}
        Так как $u(\overrightarrow{x})$ -- решение, то 
        \begin{equation*}
            \sum \limits_{i = 1}^{n} f^i \frac{\partial u}{\partial x^i} = 0
        \end{equation*}

        Значит $u(\overrightarrow{x})$ -- первый интеграл системы $\eqref{part_eq_4}$ по критерию первого интеграла. Этот первый интеграл может зависеть только от независимых переменных $\nu_1(\overrightarrow{x}), \dots, \nu_k(\overrightarrow{x})$, причём $u(\nu_1(\overrightarrow{x}), \dots, \nu_k(\overrightarrow{x})) = C_0$, где $\nu_1(\overrightarrow{x}), \dots, \nu_k(\overrightarrow{x})$ -- первые интегралы системы $\eqref{part_eq_4}$.
    \end{proof}

    \subsection{Задача Коши для уравнения в частных производных первого порядка}

    Пусть $S: \; g(\overrightarrow{x}) = 0$ -- гладкая поверхность в $\mathbb{R}^n$ и 
    \begin{equation*}
        \triangledown g = \bigg| \bigg| \frac{\partial g}{\partial x^{1}}, \dots, \frac{\partial g}{\partial x^{n}} \bigg| \bigg| \neq \overrightarrow{0}
    \end{equation*}

    \begin{definition}
        Точка $\overrightarrow{a} \in S$ называется некритической точкой поверхности, если в системе $\eqref{part_eq_4}$ $\overrightarrow{f}(\overrightarrow{a}) \neq \overrightarrow{0}$ и $(\triangledown g(\overrightarrow{a}), \overrightarrow{f}(\overrightarrow{a})) \neq 0$ (фазовые траектории не лежат на $S$).
    \end{definition}

    Пусть на $S$ задана функция $U_0(\overrightarrow{x})$ и $U_0(\overrightarrow{x}) \in C^1(\mathbb{R}^n)$.

    Задача Коши: найти такое решение $u(\overrightarrow{x})$ уравнения $\eqref{part_eq_3}$, что $u(\overrightarrow{x}) = U_0(\overrightarrow{x}) \; \forall \overrightarrow{x} \in S$.
    
    \begin{theorem}
        Пусть на гладкой поверхности $S$ задана непрерывно дифференцируемая функция $U_0(\overrightarrow{x})$. Тогда если точка $\overrightarrow{a_0} \in S$ является некритической, то существует окрестность этой точки, в которой решение задачи Коши $u(\overrightarrow{x}) = U_0(\overrightarrow{x})$ для уравнения $\eqref{part_eq_3}$ существует и единственно.
    \end{theorem}
    \begin{proof}
        Запишем параметризацию поверхности $S$ в $\mathbb{R}^n$: $x^i = \varphi^i(u_1, \dots, u_{n - 1}), \; i = \overline{1, n}$. Поверхность $S$ может быть параметризована, поскольку требование $\triangledown g \neq \overrightarrow{0}$ означает, что 
        \begin{equation*}
            rank \bigg| \bigg| \frac{\partial g}{\partial x^{1}}, \dots, \frac{\partial g}{\partial x^{n}} \bigg| \bigg| = 1 \neq 0.
        \end{equation*}
        Значит по теореме о неявной функции параметризация поверхности $S$ задаётся следующим образом:
        \begin{equation*}
            \begin{cases}
                x^1 = \varphi(x^2, \dots, x^n) \\
                x^2 = x^2                      \\
                \dots                          \\
                x^n = x^n
            \end{cases}
        \end{equation*}
        Значит $u(\overrightarrow{x}) = u(x^1, \dots, x^n) = u(\varphi(x^2, \dots, x^n), \dots, x^n) = U_0(x^2, \dots, x^n)$.

        Так как $\overrightarrow{a_0} \in S$ является некритической по условию, то существует такая окрестность этой точки $\mathcal{U} (\overrightarrow{a_0})$, где существуют $n - 1$ независимых первых интегралов системы $\eqref{part_eq_4}$: $\nu_1(\overrightarrow{x}) = C_1, \dots, \nu_{n - 1}(\overrightarrow{x}) = C_{n - 1}$, а общее решение уравнения $\eqref{part_eq_3}$ $u = u(\nu_1(\overrightarrow{x}), \dots, \nu_{n - 1}(\overrightarrow{x}))$. 

        Рассмотрим систему уравнений относительно $x^1, \dots, x^n$:
        \begin{equation}
            \begin{cases}
                \nu_1(\overrightarrow{x}) = C_1             \\
                \dots                                       \\
                \nu_{n - 1}(\overrightarrow{x}) = C_{n - 1} \\
                g(\overrightarrow{x}) = 0
            \end{cases}
            \label{part_eq_5}
        \end{equation}
        Допустим, что систему удалось разрешить и была получена параметризация поверхности $S$ $g(\overrightarrow{x}) = 0$:
        \begin{equation*}
            \begin{cases}
                x^1_S = x^1_S(C_1, \dots, C^{n - 1}) \\
                \dots                                \\
                x^n_S = x^n_S(C_1, \dots, C^{n - 1})
                G()
            \end{cases}
        \end{equation*}

        Рассмотрим
        \begin{equation*}
            J(\overrightarrow{a_0}) =
            \begin{vmatrix}
                \frac{\partial \nu_1}{\partial x^1} & \dots & \frac{\partial \nu_1}{\partial x^n}             \\
                \dots                                                                                         \\
                \frac{\partial \nu_{n - 1}}{\partial x^1} & \dots & \frac{\partial \nu_{n - 1}}{\partial x^n} \\
                \frac{\partial g}{\partial x^1} & \dots & \frac{\partial g}{\partial x^n}                     \\
            \end{vmatrix} (\overrightarrow{a_0})
        \end{equation*}

        Так как $\overrightarrow{f}(\overrightarrow{a_0}) \neq 0$, то умножим $i$-ый столбец определителя $J(\overrightarrow{a_0})$ на $r^i = f^i(\overrightarrow{a_0})$ и прибавим к первому столбцу все те столбцы, которые умножились $r^i = f^i(\overrightarrow{a_0}) \neq 0$. Учтём, что $\forall i = \overline{1, n - 1}$:
        \begin{equation*}
            \sum \limits_{j = 1}^n \frac{\partial \nu_i}{\partial x^j}(\overrightarrow{a_0}) f^j(\overrightarrow{a_0}) = 0
        \end{equation*}
        так как $\nu_i$ -- первый интеграл. 
        Преобразованный определитель будет выглядеть следующим образом:
        \begin{equation*}
            J'(\overrightarrow{a_0}) =
            \begin{vmatrix}
                0 & \frac{\partial \nu_1}{\partial x^2} r^2 & \dots & \frac{\partial \nu_1}{\partial x^n} r^n                             \\
                \dots                                                                                                                     \\
                0 & \frac{\partial \nu_{n - 1}}{\partial x^2} r^2 & \dots & \frac{\partial \nu_{n - 1}}{\partial x^n} r^n                 \\
                (\triangledown g, \overrightarrow{f}) & \frac{\partial g}{\partial x^2} r^2 & \dots & \frac{\partial g}{\partial x^n} r^n \\
            \end{vmatrix} (\overrightarrow{a_0}) = (-1)^{n + 1} (\triangledown g, \overrightarrow{f})
            \begin{vmatrix}
                \frac{\partial \nu_1}{\partial x^2} r^2 & \dots & \frac{\partial \nu_1}{\partial x^n} r^n             \\
                \dots                                                                                                 \\
                \frac{\partial \nu_{n - 1}}{\partial x^2} r^2 & \dots & \frac{\partial \nu_{n - 1}}{\partial x^n} r^n \\
            \end{vmatrix} \neq 0
        \end{equation*}

        Утверждение справедливо, так как $ (\triangledown g, \overrightarrow{f}) \neq 0$ в нехарактеристической точке $\overrightarrow{a_0}$ и
        \begin{equation*}
            rank
            \begin{Vmatrix}
                \frac{\partial \nu_1}{\partial x^2} & \dots & \frac{\partial \nu_1}{\partial x^n}               \\
                \dots                                                                                                              \\
                \frac{\partial \nu_{n - 1}}{\partial x^2}  & \dots & \frac{\partial \nu_{n - 1}}{\partial x^n}\\
            \end{Vmatrix} = n - 1
        \end{equation*}
        так как первые интегралы функционально независимы.

        Таким образом в силу непрерывности рассматриваемых функций существует окрестность $\mathcal{U}(a_0)$ в которой исходный определитель
        \begin{equation*}
            J(\overrightarrow{a_0}) =
            \begin{vmatrix}
                \frac{\partial \nu_1}{\partial x^1} & \dots & \frac{\partial \nu_1}{\partial x^n}             \\
                \dots                                                                                         \\
                \frac{\partial \nu_{n - 1}}{\partial x^1} & \dots & \frac{\partial \nu_{n - 1}}{\partial x^n} \\
                \frac{\partial g}{\partial x^1} & \dots & \frac{\partial g}{\partial x^n}                     \\
            \end{vmatrix} \neq 0,
        \end{equation*}
        то есть определитель матрицы Якоби исходной системы $\eqref{part_eq_5}$ не равен нулю. Тогда по теореме о системе неявных функций система однозначно разрешима и существуют единственным образом определённные функции $x^1_S = x^1_S(C_1, \dots, C^{n - 1}), \dots, x^2_S = x^2_S(C_1, \dots, C^{n - 1})$, а значит $u = u(x^1_S(C_1, \dots, C^{n - 1}), \dots, x^n_S(C_1, \dots, C^{n - 1}))$ является решением уравнения $\eqref{part_eq_3}$ и $u(\overrightarrow{x_S}) = U_0(\overrightarrow{x}) \forall \overrightarrow{x} \in S$.

    \end{proof}

\end{document}
