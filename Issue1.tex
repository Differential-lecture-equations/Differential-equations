%% яотключил верхние колонтинулы
%%% Работа с русским языком
\usepackage{cmap}					% поиск в PDF
\usepackage{mathtext} 				% русские буквы в формулах
\usepackage[T2A]{fontenc}			% кодировка
\usepackage[utf8]{inputenc}			% кодировка исходного текста
\usepackage[russian]{babel}	% локализация и переносы

%%% Дополнительная работа с математикой
\usepackage{amsmath,amsfonts,amssymb,amsthm,mathtools} % AMS
\usepackage{icomma} % "Умная" запятая: $0,2$ --- число, $0, 2$ --- перечисление

%% Номера формул
%\mathtoolsset{showonlyrefs=true} % Показывать номера только у тех формул, на которые есть \eqref{} в тексте.
%\usepackage{leqno} % Немуреация формул слева

%% Шрифты
\usepackage{euscript}	 % Шрифт Евклид
\usepackage{mathrsfs} % Красивый матшрифт

%%% Свои команды
\DeclareMathOperator{\sgn}{\mathop{sgn}}

%% Поля
\usepackage[left=2cm,right=2cm,top=2cm,bottom=2cm,bindingoffset=0cm]{geometry}

%% Русские списки
\usepackage{enumitem}
\makeatletter
\AddEnumerateCounter{\asbuk}{\russian@alph}{щ}
\makeatother

%%% Работа с картинками
\usepackage{graphicx}  % Для вставки рисунков
\graphicspath{{images/}{images2/}}  % папки с картинками
\setlength\fboxsep{3pt} % Отступ рамки \fbox{} от рисунка
\setlength\fboxrule{1pt} % Толщина линий рамки \fbox{}
\usepackage{wrapfig} % Обтекание рисунков и таблиц текстом

%%% Работа с таблицами
\usepackage{array,tabularx,tabulary,booktabs} % Дополнительная работа с таблицами
\usepackage{longtable}  % Длинные таблицы
\usepackage{multirow} % Слияние строк в таблице

%% Красная строка
\setlength{\parindent}{2em}

%% Интервалы
\linespread{1}
\usepackage{multirow}

%% TikZ
\usepackage{tikz}
\usetikzlibrary{graphs,graphs.standard}

%% Верхний колонтитул
% \usepackage{fancyhdr}
% \pagestyle{fancy}

%% Перенос знаков в формулах (по Львовскому)
\newcommand*{\hm}[1]{#1\nobreak\discretionary{}
	{\hbox{$\mathsurround=0pt #1$}}{}}

%% дополнения
\usepackage{float} %Добавляет возможность работы с командой [H] которая улучшает расположение на странице
\usepackage{gensymb} %Красивые градусы
\usepackage{caption} % Пакет для подписей к рисункам, в частности, для работы caption*

% подключаем hyperref (для ссылок внутри  pdf)
\usepackage[unicode, pdftex]{hyperref}

%%% Теоремы
\theoremstyle{plain}                    % Это стиль по умолчанию, его можно не переопределять.
\renewcommand\qedsymbol{$\blacksquare$} % переопределение символа завершения доказательства

\newtheorem{theorem}{Теорема}[section] % Теорема (счетчик по секиям)
\newtheorem{proposition}{Утверждение}[section] % Утверждение (счетчик по секиям)
\newtheorem{definition}{Определение}[section] % Определение (счетчик по секиям)
\newtheorem{corollary}{Следствие}[theorem] % Следстиве (счетчик по теоремам)
\newtheorem{problem}{Задача}[section] % Задача (счетчик по секиям)
\newtheorem*{remark}{Примечание} % Примечание (можно переопределить, как Замечание)
\newtheorem{lemma}{Лемма}[section] % Лемма (счетчик по секиям)

\newtheorem{example}{Пример}[section] % Пример
\newtheorem{counterexample}{Контрпример}[section] % Контрпример

\begin{document}

    \section{Основные понятия, простейшие типы дифференциальных уравнений}
    \subsection{Основные понятия}

    \begin{definition} 
        Уравнение вида \[F(x, y'(x), y''(x), \dots, y^{(n)}(x)) = 0\] называется обыкновенным дифференциальным уравнением,
        где $x$ -- аргумент, $y(x)$ -- неизвестная функция, $F$ -- известная функция.
    \end{definition}

    \begin{definition}
        Если это уравнение удается разрешить относительно старшей производной, такое дифференциальное
        уравнение называется разрешенным относительно старшей производной и записывается в виде 
        \[y^{(n)}(x) = f(x, y'(x), y''(x), \dots, y^{(n-1)}(x))\]
    \end{definition}

    Порядок уравнения определяется порядком старшей производной от $y$.

    \begin{definition}
        Функция $y = \varphi(x)$ называется решением ДУ, если она $n$ раз дифференцируема и 
        \[F(x, \varphi(x), \varphi'(x), \dots, \varphi^{(n)}(x)) \equiv 0 ~~ \forall x\]
    \end{definition}

    \begin{definition}
        Система $n$ уравнений
        \begin{equation*}
            \begin{cases}
                \dot x^1 = f_1(t, x^1(t), \dots, x^n(t)) \\
                \dots \\
                \dot x^n = f_n(t, x^1(t), \dots, x^n(t)) \\
            \end{cases}
        \end{equation*}
        где $x^1(t), \dots, x^n(t)$ -- искомые функции, называется нормальной системой ДУ n-го порядка.
    \end{definition}

    Рассмотрим ДУ $y^{(n)}(x) = f(x, y'(x), y''(x), \dots, y^{(n-1)}(x))$. Введем обозначения: $y = v_1(x)$, $y' = v_2(x)$,
    $y'' = v_3(x)$, $\dots$, $y^{(n - 1)} = v_n(x)$. Тогда $f(x, v_1, v_2, \dots, v_n) = \dot v_n$.
    
    Покажем, что
    \begin{equation*}
        f(x, v_1, v_2, \dots, v_n) = \dot v_n \Leftrightarrow y^{(n)}(x) = f(x, y'(x), y''(x), \dots, y^{(n-1)}(x))
    \end{equation*}

    В силу (1.4) видно, что любое решение (1.2) есть решение (1.5).
    Пусть вектор-функция
    \begin{equation*}
        \vec{v}(x) = 
        \begin{pmatrix}
            v^1 \\
            \dots \\
            v^n
        \end{pmatrix}
    \end{equation*}
    является решением (1.5). Тогда, в силу (1.4) имеем 
    
\end{document}