%%% Работа с русским языком
\usepackage{cmap}					% поиск в PDF
\usepackage{mathtext} 				% русские буквы в формулах
\usepackage[T2A]{fontenc}			% кодировка
\usepackage[utf8]{inputenc}			% кодировка исходного текста
\usepackage[russian]{babel}	% локализация и переносы

%%% Дополнительная работа с математикой
\usepackage{amsmath,amsfonts,amssymb,amsthm,mathtools} % AMS
\usepackage{icomma} % "Умная" запятая: $0,2$ --- число, $0, 2$ --- перечисление

%% Номера формул
%\mathtoolsset{showonlyrefs=true} % Показывать номера только у тех формул, на которые есть \eqref{} в тексте.
%\usepackage{leqno} % Немуреация формул слева

%% Шрифты
\usepackage{euscript}	 % Шрифт Евклид
\usepackage{mathrsfs} % Красивый матшрифт

%%% Свои команды
\DeclareMathOperator{\sgn}{\mathop{sgn}}

%% Поля
\usepackage[left=2cm,right=2cm,top=2cm,bottom=2cm,bindingoffset=0cm]{geometry}

%% Русские списки
\usepackage{enumitem}
\makeatletter
\AddEnumerateCounter{\asbuk}{\russian@alph}{щ}
\makeatother

%%% Работа с картинками
\usepackage{graphicx}  % Для вставки рисунков
\graphicspath{{images/}{images2/}}  % папки с картинками
\setlength\fboxsep{3pt} % Отступ рамки \fbox{} от рисунка
\setlength\fboxrule{1pt} % Толщина линий рамки \fbox{}
\usepackage{wrapfig} % Обтекание рисунков и таблиц текстом

%%% Работа с таблицами
\usepackage{array,tabularx,tabulary,booktabs} % Дополнительная работа с таблицами
\usepackage{longtable}  % Длинные таблицы
\usepackage{multirow} % Слияние строк в таблице

%% Красная строка
\setlength{\parindent}{2em}

%% Интервалы
\linespread{1}
\usepackage{multirow}

%% TikZ
\usepackage{tikz}
\usetikzlibrary{graphs,graphs.standard}

%% Верхний колонтитул
% \usepackage{fancyhdr}
% \pagestyle{fancy}

%% Перенос знаков в формулах (по Львовскому)
\newcommand*{\hm}[1]{#1\nobreak\discretionary{}
	{\hbox{$\mathsurround=0pt #1$}}{}}

%% дополнения
\usepackage{float} %Добавляет возможность работы с командой [H] которая улучшает расположение на странице
\usepackage{gensymb} %Красивые градусы
\usepackage{caption} % Пакет для подписей к рисункам, в частности, для работы caption*

% подключаем hyperref (для ссылок внутри  pdf)
\usepackage[unicode, pdftex]{hyperref}

%%% Теоремы
\theoremstyle{plain}                    % Это стиль по умолчанию, его можно не переопределять.
\renewcommand\qedsymbol{$\blacksquare$} % переопределение символа завершения доказательства

\newtheorem{theorem}{Теорема}[section] % Теорема (счетчик по секиям)
\newtheorem{proposition}{Утверждение}[section] % Утверждение (счетчик по секиям)
\newtheorem{definition}{Определение}[section] % Определение (счетчик по секиям)
\newtheorem{corollary}{Следствие}[theorem] % Следстиве (счетчик по теоремам)
\newtheorem{problem}{Задача}[section] % Задача (счетчик по секиям)
\newtheorem*{remark}{Примечание} % Примечание (можно переопределить, как Замечание)
\newtheorem{lemma}{Лемма}[section] % Лемма (счетчик по секиям)

\newtheorem{example}{Пример}[section] % Пример
\newtheorem{counterexample}{Контрпример}[section] % Контрпример

\begin{document}
    \section*{Билет 4}
    \subsection*{Фундаментальная система и фундаментальная матрица решений линейной однородной
    системы}

    Будем рассматривать однородную систему ДУ вида:

    \begin{equation*}
        \frac{d \vec x}{dt} = A \vec x; ~~ \dot x^i = \sum^n_{k = 1} a^i_k x^k; ~~ i, k = \overline{1, n} 
    \end{equation*}

    \begin{proposition}
        Для однородных систем линейных уравнений верен принцип суперпозиций, т.е если
        система функций $\varphi_1, \dots, \varphi_n$ -- решение системы уравнений, то любая их линейная комбинация тоже
        является решением.
    \end{proposition}
    %% доказательство пока приводить не стал

    \begin{definition}
        Пусть имеется система вектор-функций $\vec \varphi_1(t), \dots, \vec \varphi_n(t)$
        \begin{equation*}
            \vec \varphi_i(t) =
            \begin{pmatrix}
                \varphi_i^1(t) \\
                \dots \\
                \varphi_i^n(t) \\
            \end{pmatrix}           
        \end{equation*}
        непрерывна на $I(x)$, тогда такая система называется
        линейно-зависимой на $I$, если \[\exists ~ C_1, \dots, C_n : \sum^n_{i = 1} |C_i| \neq 0 ~ \& ~ \sum^n_{i = 1} C_i \vec \varphi_i(t) = 0 ~ \forall t \in I\]
        В противном случае, система вектор-функций называется линейно-независимой, то есть условие
        \[\sum^n_{i = 1} C_i \vec \varphi_i(t) = 0 ~ \forall t \in I\] выполняется только при $C_1 = C_2 = \dots = C_n = 0$.
    \end{definition}

    \begin{definition}
        Пусть система вектор-функций $\vec \varphi_1(t), \dots, \vec \varphi_n(t)$ линейно-независима на $I$ и каждая вектор-функция
        $\vec \varphi_i(t)$ является решением системы ДУ $\frac{d \vec x}{dt} = A \vec x$.Тогда такая система вектор-функций
        называется фундаментальной системой решений (ФСР) данной системы ДУ.
    \end{definition}

    \begin{theorem}
        Рассмотрим систему ДУ $\frac{d \vec x}{dt} = A \vec x$. Если матрица $A$ является непрерывной на отрезке $[a, b]$, то система
        имеет ФСР на этом отрезке.
    \end{theorem}
    %% доказательство пока не привожу

    \begin{theorem}
        Пусть система вектор-функций $\vec \varphi_1(t), \dots, \vec \varphi_n(t)$ является ФСР системы ДУ, тогда
        любое решение этой системы ДУ можно представить, как линейную комбинацию компонентов ФСР: 
        $\vec x(t) = C_1 \vec \varphi_1(t) + \dots + C_n \vec \varphi_n(t)$, где $C_1, dots, C_n$ -- произвольные постоянные.
    \end{theorem}
    %% доказательство пока не привожу

    \begin{definition}
        Решение системы ДУ вида $\vec x(t) = C_1 \vec \varphi_1(t) + \dots + C_n \vec \varphi_n(t)$, где $C_1, dots, C_n$
        называется общим решением сисстемы ДУ.
    \end{definition}

    \subsection*{Структура общего решения линейной однородной и неоднородной систем}

    Введем оператор $L$ такой, что $L = \frac{d}{dt} - A$. Тогда однородная система ДУ $\frac{d \vec x}{dt} = A \vec x$ запишется в виде
    $L(\vec x) = 0$, неоднородная система ДУ $\frac{d \vec x}{dt} - A \vec x = q(t)$ запишется в виде $L(\vec x) = q(t)$.

    \begin{proposition}
        Общее решение неоднородной системф ДУ $\frac{d \vec x}{dt} - A \vec x = q(t)$ представляет собой следующее выражение:
        \begin{equation*}
            \vec x = \vec x^s + \vec x^{\text{об}}_0
        \end{equation*}
        где $\vec x^s$ -- частное решение линейного неоднородного уравнение, т. е. $L(\vec x^s) = q(t)$, а
        $\vec x^{\text{об}}_0$ -- общее решение системы линейный \textbf{однородных} уравнений $L(\vec x^{\text{об}}_0) = 0$.
        Таким образом, получаем:
        \[L(\vec x) = L(\vec x^s + \vec x^{\text{об}}_0) = L(\vec x^s) + L(\vec x^{\text{об}}_0) = q(t) + 0\]
    \end{proposition}

    \subsection*{Определитель Вронского}

    \begin{definition}
        Пусть на $I$ определена система вектор-функций $\vec \varphi_1(t), \dots, \vec \varphi_n(t)$, тогда определитель
        \begin{equation*}
            W(t) = 
            \begin{vmatrix}
                \varphi^1_1(t) \dots \varphi^1_n(t) \\
                \dots ~~~~~~~~ \dots \\
                \varphi^n_1(t) \dots \varphi^n_n(t) \\
            \end{vmatrix}
        \end{equation*}
        называется определителем Вронского.
    \end{definition}

    \begin{theorem}
        Если $\exists ~ t_0 \in I : ~ W(t_0) \neq 0$, то система является линейно независимой на $I$. Обратное неверно,
        пример:
        \begin{equation*}
            \varphi_1 = 
            \begin{pmatrix}
                t \\
                0
            \end{pmatrix}, ~
            \varphi_2 = 
            \begin{pmatrix}
                1 \\
                0
            \end{pmatrix} ~ \text{ЛНЗ, но} ~~ W(t) = 0
        \end{equation*}
    \end{theorem}

    \begin{proof}
        Будем доказывать от противного: пусть система является линейно-зависимой, тогда $\exists ~ C_1, \dots, C_n:$
        $C_1 \vec \varphi_1(t) + \dots + C_n \vec \varphi_n(t) = 0 ~ \forall t \in I$. Тогда в определителе Вронского $W(t)$
        есть хотя бы два линейно-зависымих столбца, так как $\vec \varphi_i(t)$ являются столбцами определителя, но тогда получам, что
        $W(t) = 0 ~ \forall t \in I$ (хотя предпологалось, что $\exists ~ t_0 \in I : ~ W(t_0) \neq 0$). Таким образом, мы получили противоречие,
        откуда следует, что система является линейно независимой на $I$.
    \end{proof}

    \subsubsection*{Свойства Вронскиана}

    \begin{enumerate}
        \item Если $\exists ~ t_0 \in I : ~ W(t_0) \neq 0$, то система является линейно независимой на $I$ (см. доказательство теоремы).
        \item Пусть вектор-функции $\vec \varphi_1(t), \dots, \vec \varphi_n(t)$ являются решениями системы ДУ, и существует точка
        $t_0 \in I: ~ W(t_0) = 0$, тогда система $\vec \varphi_1(t), \dots, \vec \varphi_n(t)$ является линейно-зависимой.
        %% доказательство пока не привожу
    \end{enumerate}

\end{document}