%%% Работа с русским языком
\usepackage{cmap}					% поиск в PDF
\usepackage{mathtext} 				% русские буквы в формулах
\usepackage[T2A]{fontenc}			% кодировка
\usepackage[utf8]{inputenc}			% кодировка исходного текста
\usepackage[russian]{babel}	% локализация и переносы

%%% Дополнительная работа с математикой
\usepackage{amsmath,amsfonts,amssymb,amsthm,mathtools} % AMS
\usepackage{icomma} % "Умная" запятая: $0,2$ --- число, $0, 2$ --- перечисление

%% Номера формул
%\mathtoolsset{showonlyrefs=true} % Показывать номера только у тех формул, на которые есть \eqref{} в тексте.
%\usepackage{leqno} % Немуреация формул слева

%% Шрифты
\usepackage{euscript}	 % Шрифт Евклид
\usepackage{mathrsfs} % Красивый матшрифт

%%% Свои команды
\DeclareMathOperator{\sgn}{\mathop{sgn}}

%% Поля
\usepackage[left=2cm,right=2cm,top=2cm,bottom=2cm,bindingoffset=0cm]{geometry}

%% Русские списки
\usepackage{enumitem}
\makeatletter
\AddEnumerateCounter{\asbuk}{\russian@alph}{щ}
\makeatother

%%% Работа с картинками
\usepackage{graphicx}  % Для вставки рисунков
\graphicspath{{images/}{images2/}}  % папки с картинками
\setlength\fboxsep{3pt} % Отступ рамки \fbox{} от рисунка
\setlength\fboxrule{1pt} % Толщина линий рамки \fbox{}
\usepackage{wrapfig} % Обтекание рисунков и таблиц текстом

%%% Работа с таблицами
\usepackage{array,tabularx,tabulary,booktabs} % Дополнительная работа с таблицами
\usepackage{longtable}  % Длинные таблицы
\usepackage{multirow} % Слияние строк в таблице

%% Красная строка
\setlength{\parindent}{2em}

%% Интервалы
\linespread{1}
\usepackage{multirow}

%% TikZ
\usepackage{tikz}
\usetikzlibrary{graphs,graphs.standard}

%% Верхний колонтитул
% \usepackage{fancyhdr}
% \pagestyle{fancy}

%% Перенос знаков в формулах (по Львовскому)
\newcommand*{\hm}[1]{#1\nobreak\discretionary{}
	{\hbox{$\mathsurround=0pt #1$}}{}}

%% дополнения
\usepackage{float} %Добавляет возможность работы с командой [H] которая улучшает расположение на странице
\usepackage{gensymb} %Красивые градусы
\usepackage{caption} % Пакет для подписей к рисункам, в частности, для работы caption*

% подключаем hyperref (для ссылок внутри  pdf)
\usepackage[unicode, pdftex]{hyperref}

%%% Теоремы
\theoremstyle{plain}                    % Это стиль по умолчанию, его можно не переопределять.
\renewcommand\qedsymbol{$\blacksquare$} % переопределение символа завершения доказательства

\newtheorem{theorem}{Теорема}[section] % Теорема (счетчик по секиям)
\newtheorem{proposition}{Утверждение}[section] % Утверждение (счетчик по секиям)
\newtheorem{definition}{Определение}[section] % Определение (счетчик по секиям)
\newtheorem{corollary}{Следствие}[theorem] % Следстиве (счетчик по теоремам)
\newtheorem{problem}{Задача}[section] % Задача (счетчик по секиям)
\newtheorem*{remark}{Примечание} % Примечание (можно переопределить, как Замечание)
\newtheorem{lemma}{Лемма}[section] % Лемма (счетчик по секиям)

\newtheorem{example}{Пример}[section] % Пример
\newtheorem{counterexample}{Контрпример}[section] % Контрпример
\newcommand{\defeq}{\stackrel{def}{=}} % по определению
\newcommand{\defarr}{\stackrel{def}{\Rightarrow}} % следует из определения
\DeclareMathOperator{\Tr}{trace}

\begin{document}
	\section{Билет 6. Первые интегралы автономных систем}
	
		\subsection{Основные определения}
		
		\begin{definition}
			Рассмотрим неавтономную систему дифференциальных уравнений $\dot{\vec{x}} = \vec{f}(\vec{x}, t)$. Пусть в некоторой области $G \subset \mathbb{R}^{n+1}_{t, \vec{x}}$ выполнены условия основной теоремы. Пусть функция $u(t, \vec{x})$ непрерывно дифференцируема в $G$, а $\vec{x} = \vec{x}(t)$ -- решение системы. Тогда величину
			\begin{equation*}
				\frac{du}{dt} = \frac{\partial u}{\partial t} + \sum\limits_{i = 1}^n \frac{\partial u}{\partial x_i} \frac{\partial x_i}{\partial t} = \frac{\partial u}{\partial t} + \sum\limits_{i = 1}^n \frac{\partial u}{\partial x_i} \frac{dx_i}{dt} = \frac{\partial u}{\partial t} + \sum\limits_{i = 1}^n \frac{\partial u}{\partial x_i} f_i(t, \vec{x}) = \frac{\partial u}{\partial t} + (\nabla u, \vec{f})
			\end{equation*} 
			будем называть производной функции $u$ в силу системы, или производной Ли.
			
			Для автономной системы $\dfrac{du}{dt} = (\nabla u, \vec{f})$.
		\end{definition}
	
		\begin{definition}
			Первым интегралом автономной системы $\dot{\vec{x}} = \vec{f}(\vec{x})$ в области $\mathscr{D}$ ее фазового пространства называется функция $u = u(\vec{x})$, сохраняющая постоянное значение вдоль каждой траектории из $\mathscr{D}$, то есть $u = C = \text{const}$ для каждой траектории в области $\mathscr{D}$.
		\end{definition}
	
		\subsection{Критерий первого интеграла}
		\begin{theorem}
			Для того, чтобы некоторая функция $u(\vec{x})$ была первым интегралом системы $\dot{\vec{x}} = \vec{f}(\vec{x})$, необходимо и достаточно, чтобы она удовлетворяла соотношению $(\nabla u, \vec{f}) = 0$.
		\end{theorem}
		
		\begin{proof}
			\hfill\\
			\textit{Необходимость}
			
			Пусть $u = u(\vec{x})$ -- первый интеграл системы. Тогда:
			\begin{equation*}
				0 = \frac{du}{dt} = \sum\limits_{i = 1}^n \frac{\partial u}{\partial x_i}\dot{x_i} = \sum\limits_{i = 1}^n \frac{\partial u}{\partial x_i} f_i(\vec{x}) = (\nabla u, \vec{f})
			\end{equation*}
		
			\hfill\\
			\textit{Достаточность}
			
			Пусть условие выполнено. Тогда:
			\begin{equation*}
				0 = (\nabla u, \vec{f}) = \sum\limits_{i = 1}^n \frac{\partial u}{\partial x_i} f_i(\vec{x}) = \sum\limits_{i = 1}^n \frac{\partial u}{\partial x_i}\dot{x_i} = \frac{du}{dt},
			\end{equation*}
			откуда и следует, что $u$ -- первый интеграл системы.
		\end{proof}
	
		\subsection{Теорема о числе независимых первых интегралов}
		
		\begin{definition}
			Система первых интегралов $u_1(\vec{x}), u_2(\vec{x}), \ldots, u_k(\vec{x})$, где $k < n$ называется \textbf{функционально} независимой в области $\mathscr{D}$, если:
			
			\begin{equation*}
				rank\left(\frac{\partial u_i}{\partial x_k}\right) = rank \begin{pmatrix}
					\dfrac{\partial u_1}{\partial x_1} & \dfrac{\partial u_1}{\partial x_2} & 	\cdots & \dfrac{\partial u_1}{\partial x_n} \\
					\dfrac{\partial u_2}{\partial x_1} & \dfrac{\partial u_2}{\partial x_2} & 	\cdots & \dfrac{\partial u_2}{\partial x_n} \\
					\vdots & \vdots & \ddots & \vdots \\
					\dfrac{\partial u_k}{\partial x_1} & \dfrac{\partial u_k}{\partial x_2} & 	\cdots & \dfrac{\partial u_k}{\partial x_n}
				\end{pmatrix} = k					
			\end{equation*}
		
		
		Другими словами, если их градиенты $\nabla u_i(\vec{x})$ \textbf{линейно} независимы.
		\end{definition}
	
		\begin{remark}
			Из линейной зависимости первых интегралов следует их функциональная зависимость. Обратное утверждение неверно.
		\end{remark}
		
		\begin{theorem}
			Пусть точка $M(\vec{x}_0) \in \mathscr{D}$ \textbf{не} является положением равновесия системы $\dot{\vec{x}} = \vec{f}(\vec{x})$. Тогда в окрестности $U(\vec{x}_0)$ этой точки существуют $n - 1$ функционально независимых первых интегралов системы. Теорема имеет локальный характер.
		\end{theorem}
	
		\begin{proof}
			\hfill\\
			Пусть $\vec{x}(t)$ является решением: $\vec{x}(0) = \vec{x}_0$.
			
			Так как $M \in \mathscr{D}$ не является положением равновесия, то через нее проходит единственная фазовая траектория, и хотя бы одна из компонент $\vec{f}(\vec{x}_0)$ не равна нулю. Пускай без ограничения общности это будет $f_n(\vec{x}_0)$. 
			
			В силу непрерывности $f_n(\vec{x})$ существует окрестность $U(\vec{x}_0)$, в которой $f_n(\vec{x}) \neq 0$.
			
			Поделим каждое уравнение нашей системы на последнее. Получим следующее:
			\begin{equation*}
				\begin{cases*}
					\dfrac{dx_1}{dx_n} = \dfrac{f_1}{f_n} = \widetilde{f_1} \\
					\dfrac{dx_2}{dx_n} = \dfrac{f_2}{f_n} = \widetilde{f_2} \\
					\hspace{3mm}\vdots \hspace{10mm} \vdots \\
					\dfrac{dx_{n-1}}{dx_n} = \dfrac{f_{n-1}}{f_n} = \widetilde{f}_{n-1}
				\end{cases*}
			\end{equation*}
		
		Все $\widetilde{f}_i$ непрерывно дифференцируемы, поэтому существует окрестность $U(\vec{x}_0)$, где выполнены условия основной теоремы. Значит $\forall \vec{\xi} \in U(\vec{x}_0)\;\exists !$ решение системы выше такое, что при $x_n = \xi_n$ мы имеем $x_1(\xi_n) = \xi_1, x_2(\xi_n) = \xi_2, \ldots, x_{n-1}(\xi_n) = \xi_{n-1}$.
	
		Давайте запишем это решение. Оно имеет вид:
		\begin{equation}\label{x_from_phi_system}
			\begin{cases*}
				x_1 = \varphi_1(x_n, \xi_1, \xi_2, \ldots, \xi_{n-1}) \\
				x_2 = \varphi_2(x_n, \xi_1, \xi_2, \ldots, \xi_{n-1}) \\
				\hspace{3mm}\vdots \hspace{10mm} \vdots \\
				x_{n-1} = \varphi_{n-1}(x_n, \xi_1, \xi_2, \ldots, \xi_{n-1})
			\end{cases*}
		\end{equation}
	
		На все это дело можно смотреть как на систему уравнений относительно $\xi_1, \xi_2, \ldots, \xi_{n-1}$. Якобиан этой системы имеет вид:
		\begin{equation*}
			J(x_n) = \begin{vmatrix}
				\dfrac{\partial \varphi_1}{\partial \xi_1} & \cdots & \dfrac{\partial \varphi_1}{\partial \xi_{n-1}} \\
				\vdots & \ddots & \vdots \\
				\dfrac{\partial \varphi_{n-1}}{\partial \xi_1} & \cdots & \dfrac{\partial \varphi_{n-1}}{\partial \xi_{n-1}}
			\end{vmatrix}
		\end{equation*}
	
		В силу того, что $J(x_n^0) = \begin{vmatrix}
			1 & \cdots & 0 \\
			\vdots & \ddots & \vdots \\
			0 & \cdots & 1
		\end{vmatrix} = |E| = 1 \neq 0$, и все производные $\dfrac{\partial \varphi_i}{\partial \xi_k}$ непрерывны, то существует окрестность точки $\vec{\xi}$, в которой $J(x_n) \neq 0$. Тогда по теореме о неявно заданной функции можно разрешить систему относительно $\xi_k$:
		\begin{equation}\label{xi_from_psi_system}
			\begin{cases*}
				\xi_1 = \psi_1(x_1, x_2, \ldots, x_n) \\
				\hspace{3mm}\vdots \hspace{10mm} \vdots \\
				\xi_{n-1} = \psi_{n-1}(x_1, x_2, \ldots, x_n) \\
			\end{cases*}
		\end{equation}
	
		Проинтегрируем формально последнее уравнение системы $\dot{\vec{x}} = \vec{f}(\vec{x})$ с условием, что при $t = \tau$: $x_n(\tau) = \xi_n$:
		\begin{equation*}
			x_n = \xi_n + \int\limits_\tau^t f_n(\vec{x}(\tau))d\tau = x_n(t).
		\end{equation*}
	
		Подставим это и (\ref{x_from_phi_system}) в (\ref{xi_from_psi_system}). Тогда:
		\begin{flalign*}
			& \forall k = \overline{1, n}: const = \xi_k =\psi_k(x_n, x_1, x_2, \ldots, x_{n-1}) =  \\ 
			& = \psi_k(x_n, \varphi_1(x_n, \xi_1, \ldots, \xi_{n-1}), \varphi_2(x_n, \xi_1, \ldots, \xi_{n-1}), \ldots, \varphi_{n-1}(x_n, \xi_1, \ldots, \xi_{n-1})) = \\
			& = \psi_k(\widetilde{\varphi}_1(t + \tau, \xi_1, \ldots, \xi_{n-1}), \widetilde{\varphi}_2(t + \tau, \xi_1, \ldots, \xi_{n-1}), \ldots, \widetilde{\varphi}_{n-1}(t + \tau, \xi_1, \ldots, \xi_{n-1}))
		\end{flalign*}
	
		Так как $\vec{\xi}$ -- произвольная точка из окрестности $U$, где выполняется основная теорема, то функции $\widetilde{\varphi}_1(t + \tau, \xi_1, \ldots, \xi_{n-1}), \widetilde{\varphi}_2(t + \tau, \xi_1, \ldots, \xi_{n-1}), \ldots, \widetilde{\varphi}_{n-1}(t + \tau, \xi_1, \ldots, \xi_{n-1})$ являются решениями исходной системы. Тогда система (\ref{xi_from_psi_system}) является системой первых интегралов.
		
		Таких интегралов $n - 1$ штук. Причем:
		\begin{equation*}
			\begin{vmatrix}
				\dfrac{\partial \psi_1}{\partial x_1} & \cdots & \dfrac{\partial \psi_1}{\partial x_{n-1}} \\
				\vdots & \ddots & \vdots \\
				\dfrac{\partial \psi_{n-1}}{\partial x_1} & \cdots & \dfrac{\partial \psi_{n-1}}{\partial x_{n-1}}
			\end{vmatrix} = \frac{1}{J(x_n)} \neq 0.
		\end{equation*}
		Откуда следует, что данная система первых интегралов функционально независима.
		
		\end{proof}
	
		
		\subsection{Применение первых интегралов для понижения порядка системы}
		
		\begin{theorem}
			Пусть $u_1(\vec{x}), u_2(\vec{x}), \ldots, u_k(\vec{x})$, где $k < n$ -- система первых интегралов системы $\dot{\vec{x}} = \vec{f}(\vec{x})$. Тогда порядок системы может быть понижен на $k$.
		\end{theorem}
	
		\begin{proof}
			\hfill\\
			Если $u_1, u_2, \ldots, u_k$ -- первые интегралы, то они постоянны на любом решении системы. На систему первых интегралов
			\begin{equation*}
				\begin{cases*}
					u_1(\vec{x}) = C_1 \\ 
					u_2(\vec{x}) = C_2 \\
					\hspace{3mm}\vdots \hspace{10mm} \vdots \\
					u_k(\vec{x}) = C_k
				\end{cases*}
			\end{equation*}
			можно смотреть как на систему уравнений относительно неизвестных $x_1, x_2, \ldots, x_n$, где $C_1, C_2, \ldots, C_k$ -- известные константы.
		
			Система первых интегралов функционально независима, поэтому ранг матрицы Якоби равен $k$. Пусть базисный минор матрицы Якоби расположен в первых $k$ столбцах (иначе просто меняем порядок переменных). Тогда по теореме о неявно заданной функции получаем:
			\begin{equation*}
				\begin{cases*}
					x_1 = \varphi_1(x_{k+1}, \ldots, x_n, C_1, \ldots, C_k) \\
					\hspace{3mm} \vdots \hspace{10mm} \vdots \\
					x_k = \varphi_k(x_{k+1}, \ldots, x_n, C_1, \ldots, C_k) \\
				\end{cases*} \Longrightarrow
				\begin{cases*}
					\dot{x}_{k+1} = f_{k+1}(\varphi_1, \ldots, \varphi_k, x_{k+1}, \ldots, x_n) \\
					\hspace{3mm} \vdots \hspace{10mm} \vdots \\
					\dot{x}_{n} = f_{n}(\varphi_1, \ldots, \varphi_k, x_{k+1}, \ldots, x_n)
				\end{cases*}
			\end{equation*}
		
			Решив последнюю систему относительно $x_{k+1}, \ldots, x_n$, то есть понизив порядок системы на $k$, найдем остальные $x_1, x_2, \ldots, x_k$.
		\end{proof}
	
	
\end{document}