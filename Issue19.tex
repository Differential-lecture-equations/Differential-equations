%%% Работа с русским языком
\usepackage{cmap}					% поиск в PDF
\usepackage{mathtext} 				% русские буквы в формулах
\usepackage[T2A]{fontenc}			% кодировка
\usepackage[utf8]{inputenc}			% кодировка исходного текста
\usepackage[russian]{babel}	% локализация и переносы

%%% Дополнительная работа с математикой
\usepackage{amsmath,amsfonts,amssymb,amsthm,mathtools} % AMS
\usepackage{icomma} % "Умная" запятая: $0,2$ --- число, $0, 2$ --- перечисление

%% Номера формул
%\mathtoolsset{showonlyrefs=true} % Показывать номера только у тех формул, на которые есть \eqref{} в тексте.
%\usepackage{leqno} % Немуреация формул слева

%% Шрифты
\usepackage{euscript}	 % Шрифт Евклид
\usepackage{mathrsfs} % Красивый матшрифт

%%% Свои команды
\DeclareMathOperator{\sgn}{\mathop{sgn}}

%% Поля
\usepackage[left=2cm,right=2cm,top=2cm,bottom=2cm,bindingoffset=0cm]{geometry}

%% Русские списки
\usepackage{enumitem}
\makeatletter
\AddEnumerateCounter{\asbuk}{\russian@alph}{щ}
\makeatother

%%% Работа с картинками
\usepackage{graphicx}  % Для вставки рисунков
\graphicspath{{images/}{images2/}}  % папки с картинками
\setlength\fboxsep{3pt} % Отступ рамки \fbox{} от рисунка
\setlength\fboxrule{1pt} % Толщина линий рамки \fbox{}
\usepackage{wrapfig} % Обтекание рисунков и таблиц текстом

%%% Работа с таблицами
\usepackage{array,tabularx,tabulary,booktabs} % Дополнительная работа с таблицами
\usepackage{longtable}  % Длинные таблицы
\usepackage{multirow} % Слияние строк в таблице

%% Красная строка
\setlength{\parindent}{2em}

%% Интервалы
\linespread{1}
\usepackage{multirow}

%% TikZ
\usepackage{tikz}
\usetikzlibrary{graphs,graphs.standard}

%% Верхний колонтитул
% \usepackage{fancyhdr}
% \pagestyle{fancy}

%% Перенос знаков в формулах (по Львовскому)
\newcommand*{\hm}[1]{#1\nobreak\discretionary{}
	{\hbox{$\mathsurround=0pt #1$}}{}}

%% дополнения
\usepackage{float} %Добавляет возможность работы с командой [H] которая улучшает расположение на странице
\usepackage{gensymb} %Красивые градусы
\usepackage{caption} % Пакет для подписей к рисункам, в частности, для работы caption*

% подключаем hyperref (для ссылок внутри  pdf)
\usepackage[unicode, pdftex]{hyperref}

%%% Теоремы
\theoremstyle{plain}                    % Это стиль по умолчанию, его можно не переопределять.
\renewcommand\qedsymbol{$\blacksquare$} % переопределение символа завершения доказательства

\newtheorem{theorem}{Теорема}[section] % Теорема (счетчик по секиям)
\newtheorem{proposition}{Утверждение}[section] % Утверждение (счетчик по секиям)
\newtheorem{definition}{Определение}[section] % Определение (счетчик по секиям)
\newtheorem{corollary}{Следствие}[theorem] % Следстиве (счетчик по теоремам)
\newtheorem{problem}{Задача}[section] % Задача (счетчик по секиям)
\newtheorem*{remark}{Примечание} % Примечание (можно переопределить, как Замечание)
\newtheorem{lemma}{Лемма}[section] % Лемма (счетчик по секиям)

\newtheorem{example}{Пример}[section] % Пример
\newtheorem{counterexample}{Контрпример}[section] % Контрпример
\newcommand{\defeq}{\stackrel{def}{=}} % по определению
\newcommand{\defarr}{\stackrel{def}{\Rightarrow}} % следует из определения
\DeclareMathOperator{\Tr}{trace}

\begin{document}
	\section{Билет 5. Автономные системы дифференциальных уравнений}
	\subsection{Основные определения}
	Система ДУ вида: 
	\begin{equation}\label{avt_sys}
		\frac{dx^i}{dt} = f^i(x^1, ..., x^n); \quad \frac{d\vec{x}}{dt} = \vec{f}(\vec{x}); \quad \dot{x}^i = f^i(\vec{x}) \quad \quad i = \overline{1, n}
	\end{equation}
	Называется автономной системой ДУ, если $ \vec{f} = \{f_i(x^1, ..., x^n)\} $,  $\ i = \overline{1, n}$ не зависит явно от аргумента $ t $; $ x^j = x^j(t), \ j = \overline{1, n} $ являются интегральными кривыми \eqref{avt_sys}. \\ $\vec{x}(t) = \{ x^j(t) \} \in \mathbb{R}^{n+1} = t \times \mathbb{R}^n$
	\begin{definition}
		Пусть $ \vec{x}(t) $ является решением \eqref{avt_sys}. Кривая $ \gamma $ в $ \mathbb{R}^n $ называется фазовой траекторией \eqref{avt_sys}. Само $ \mathbb{R}^n $ называется фазовым пространством \eqref{avt_sys}.
		\begin{equation}\label{gamma_sys}
			\gamma = \left\{
				\begin{aligned}
					x^1 &= x^1(t) \\
					x^2 &=x^2(t) \\
					.... \\
					x^n &= x^n(t) \\
				\end{aligned}
			\right.
		\end{equation}
		Будем предполагать, что $ \vec{f} = \{ f^i(x^1, ..., x^n) \} \in D \subset \mathbb{R}^n, i = \overline{1, n} $ непрерывно дифференцируемые функции по всей совокупности переменных.
	\end{definition}

	\begin{theorem}
		Если $ \varphi(t) $ решение \eqref{avt_sys}, то $ \varphi(t + \tau) \ \forall \tau = const \in \mathbb{R}$ тоже решение \eqref{avt_sys}
	\end{theorem}

	\begin{proof}
		\ \\
		Пусть $ u = t + \tau: \dfrac{d(\varphi(t + \tau))}{dt} = | t + \tau = u | = \dfrac{d \varphi(u)}{du} \dfrac{du}{dt} = \dfrac{d \varphi(u)}{dt} = f(\varphi(u)) = f(\varphi(t + \tau))-$ \\ $-$ т.е. $\varphi(t + \tau)$ - решение
	\end{proof}

	\begin{corollary}
		Пусть $ \vec{\varphi}(t_0, \vec{x}_0)$ - решение \eqref{avt_sys}, такое что $ \vec{\varphi}(t, t_0, \vec{x}_0) = \vec{x}_0 $. В силу доказанной теоремы $ \vec{\varphi}(t + \tau, t_0 + \tau, \vec{x}_0) $ тоже решение \eqref{avt_sys}. (Формально заменяем $ t + \tau $ на $ u $, $ t_0 + \tau $ на $ u_0$),  причём $ \vec{\varphi}(t_0 + \tau, t_0 + \tau, \vec{x}_0) = \vec{x}_0 $. Тогда, если $ \vec{f}(x^1,..., x^n) $ является непрерывной функцией n переменных вместе с $ \dfrac{\partial \vec{f}}{\partial x_i} $, то показанные решения совпадают по основной теореме. \\
		$ \vec{\varphi}(t + \tau, t_0 + \tau, \vec{x}_0) \equiv \vec{\varphi}(t, t_0, \vec{x}_0)$. Положим, в силу произвольности $ \tau $, $ \tau = - t_0 \Rightarrow$ \\ $\Rightarrow \vec{\varphi}(t, t_0, \vec{x}_0) = \vec{\varphi}(t - t_0, 0, \vec{x}_0) = \vec{\varphi}(t - t_0, \vec{x}_0) $ \\
		Т.о. положение движущейся по фазовой траектории точки определяется начальным положением $ \vec{x}_0 $ в момент времени $ t_0 $ и длительностью $ t - t_0 $, отсчитываемого от начального момента времени $ t_0 $, но не самим этим моментом. (Т.е. начальный момент не существенен и можно положить его равным нулю).
	\end{corollary}

	\begin{theorem}
		Фазовые траектории либо не имеют общих точек, либо совпадают
	\end{theorem}

	\begin{proof}
		\ \\
		Пусть $ \varphi (t) и \psi(t) $ - решения \eqref{avt_sys}, причём $ x_0 = \varphi(t_1) = \psi(t_2) $ Рассмотрим $ \chi (t) = \psi (t + (t_2 - t_1)) $, согласно предыдущей теореме $ \chi(t) $ тоже явл. реш. \eqref{avt_sys}, причём $ \chi (t_1) \overset{\text{по постр.}}{=\joinrel=\joinrel=} x_0 = \psi (t_2) = \\ =\varphi (t_1) \Rightarrow$ По основной теореме $ \varphi (t) \equiv \chi(t) \overset{def}{=} \psi(t + (t_2 - t_1)) \Rightarrow $ траектории $ \varphi (t) $ и $ \psi(t) $ совпали.
	\end{proof}
	\noindent Согласно доказаному можно считать, что фазовое пространство \eqref{avt_sys} "склеено" из фазовых траекторий.
	
	\subsection{Типы фазовых траекторий}
	\begin{definition}
		Точка $\vec{a} \in \vec{\mathbb{R}}^n$ называется положением равновесия \eqref{avt_sys}, если \\ $ \vec{f}(\vec{a}) = 0 \ \ (f^i(a^1, ..., a^n) = 0, \ i = \overline{1, n} )$
	\end{definition}

	\begin{proposition}
		Если $ \vec{a} \in \vec{\mathbb{R}}^n $ - положение равновесия \eqref{avt_sys}, то $ \vec{x}(t) = \vec{a}, -\infty < t < + \infty$ является решением \eqref{avt_sys}
	\end{proposition}

	\begin{proof}
		\ \\
		$\vec{x}(t) \equiv \vec{a} \overset{\eqref{avt_sys}}{\Rightarrow} 0 = \dfrac{d\vec{x}}{dt} = \dfrac{d \vec{a}}{t} = f(\vec{a}) = 0 \Rightarrow$ удовлетворяет \eqref{avt_sys}
	\end{proof}
	\noindent Т.о. точка равновесия $ \vec{a} \in \vec{\mathbb{R}}^n$ является фазовой траекторией \eqref{avt_sys}
	
	\begin{corollary}
		Решение \eqref{avt_sys} не может прийти в положение равновесия за конечное время.
	\end{corollary}

	\begin{proof}
		\ \\
		Пусть это не так и фазовая траектория пришла в положение равновесия за конечное время. Т.о., т.к. положение равновесия тоже является фазовой траекторией, то они пересекаются, что невозможно $\Rightarrow$ противоречие 
	\end{proof}

	\begin{theorem}
		Фазовые траектории принадлежат одному из трёх типов:
		\begin{enumerate}
			\item Точка (равновесия)
			\item Фазовая траектория, отличная от точки, есть гладкая кривая
			\item Замкнутая кривая(цикл) - периодическая
		\end{enumerate}
	\end{theorem}

	\subsection{Групповые свойства автономных систем}
		\begin{enumerate}
			\item $	\vec{\varphi}(t_1 + t_2, \vec{x}_0) = \varphi (t_2, \vec{\varphi}(t_1, \vec{x}_0)) = \vec{\varphi}(t_1, \vec{\varphi}(t_2, \vec{x}_0))$
			\begin{proof}
				\ \\
				Рассмотрим $ \vec{\varphi}(t, \vec{\varphi}(t_1, \vec{x}_0)) $ - решение \eqref{avt_sys}; $\vec{\varphi}(t + t_1; \vec{x}_0)$ - тоже решение \eqref{avt_sys}
				\[
					\left. 
						\begin{aligned}
							&\vec{\varphi}(0, \vec{\varphi}(t_1, \vec{x}_0)) = \vec{\varphi}(t_1, \vec{x}_0) \\
							&\vec{\varphi}(0 + t_1, \vec{x}_0) = \vec{\varphi}(t_1, \vec{x}_0) \\
						\end{aligned}
					\right\rbrace \stackrel{\text{основная теорема}}{=\joinrel=\joinrel=\joinrel=\joinrel=\joinrel=\joinrel\Rightarrow} \vec{\varphi}(t + t_1; \vec{x}_0) \equiv \vec{\varphi} (t, \vec{\varphi}(t_1, \vec{x}_0))
				\]
				Аналогично, $ \vec{\varphi}(t + t_2, \vec{x}_0) \equiv \vec{\varphi}(t, \vec{\varphi}(t_2, \vec{x}_0))$
			\end{proof}
			\item $ \vec{\varphi} (-t; \vec{\varphi}(t, \vec{x}_0)) = \vec{x}_0 $
			\begin{proof}
				\ \\
				Из 1):  $\ \vec{\varphi} (t + \tau, \vec{x}_0) = \vec{\varphi}(\tau, \vec{\varphi}(t, \vec{x}_0)) $. В силу произвольности $ \tau $ при $ \tau = -t $: $\\ \vec{\varphi}(-t, \vec{\varphi}(t, \vec{x}_0)) \stackrel{1)}{=} \vec{\varphi}(0, \vec{x}_0) = \vec{x}_0 $
			\end{proof}
		\end{enumerate}

	\subsection{Понятия фазового потока и фазового объема}
	\begin{definition}
		Рассматриваем давно привычную нам систему $\dfrac{d\vec{x}}{dt} = \vec{f}(\vec{x})$.
		
		Пусть $\mathscr{D} \subset \mathbb{R}^n$ -- это область в ее фазовом пространстве. Возьмем произвольную точку $\vec{x}_0 \in \mathscr{D}$ и выпустим из нее фазовую траекторию. Таким образом, с течением времени $t$ мы будем двигаться по этой траектории. Обозначим точку на данной траектории в момент времени $t$ как $g^t \vec{x}_0$. 
		
		Теперь можно определить преобразование области $\mathscr{D}$: $\forall \vec{x}_0 \in \mathscr{D}$ сделаем отображение $\vec{x}_0 \rightarrow g^t \vec{x}_0$. Получаем $\mathscr{D} \rightarrow g^t\mathscr{D}$. Другими словами, каждую точку $\mathscr{D}$ сносим по фазовой траектории на время $t$.
		
		
		
		\tikzset{every picture/.style={line width=0.75pt}}        
		\begin{tikzpicture}[x=0.75pt,y=0.75pt,yscale=-1,xscale=1]
			\draw    (247.53,281.36) .. controls (246.65,147.81) and (519.67,340.56) .. (539.08,208.1) ;

			\draw  [line width=2.25]  (147.34,109.16) .. controls (256.61,189.1) and (333.31,22.38) .. (383.41,102.31) .. controls (433.5,182.25) and (557.18,294.15) .. (464.81,371.8) .. controls (372.45,449.45) and (178.65,314.7) .. (147.34,246.19) .. controls (116.03,177.68) and (38.07,29.23) .. (147.34,109.16) -- cycle ;

			\draw  [fill={rgb, 255:red, 0; green, 2; blue, 0 }  ,fill opacity=1 ] (245.37,281.36) .. controls (245.37,279.81) and (246.34,278.55) .. (247.53,278.55) .. controls (248.72,278.55) and (249.68,279.81) .. (249.68,281.36) .. controls (249.68,282.91) and (248.72,284.17) .. (247.53,284.17) .. controls (246.34,284.17) and (245.37,282.91) .. (245.37,281.36) -- cycle ;

			\draw  [fill={rgb, 255:red, 20; green, 4; blue, 0 }  ,fill opacity=1 ] (371.82,242.13) .. controls (371.82,240.49) and (372.74,239.16) .. (373.86,239.16) .. controls (374.98,239.16) and (375.89,240.49) .. (375.89,242.13) .. controls (375.89,243.77) and (374.98,245.09) .. (373.86,245.09) .. controls (372.74,245.09) and (371.82,243.77) .. (371.82,242.13) -- cycle ;

			\draw [color={rgb, 255:red, 74; green, 144; blue, 226 }  ,draw opacity=1 ]   (239.76,272.04) .. controls (221.69,230.23) and (304.05,180.28) .. (364.71,231.08) ;
			\draw [shift={(365.63,231.85)}, rotate = 220.6] [color={rgb, 255:red, 74; green, 144; blue, 226 }  ,draw opacity=1 ][line width=0.75]    (10.93,-3.29) .. controls (6.95,-1.4) and (3.31,-0.3) .. (0,0) .. controls (3.31,0.3) and (6.95,1.4) .. (10.93,3.29)   ;
			

			\draw (156.76,157.66) node [anchor=north west][inner sep=0.75pt]    {$\mathscr{D} \subset \mathbb{R}^{n}$};

			\draw (258.56,283.08) node [anchor=north west][inner sep=0.75pt]    {$x_{0}$};

			\draw (363.97,267.92) node [anchor=north west][inner sep=0.75pt]    {$g^{t} x_{0}$};

			\draw (492.88,273.91) node [anchor=north west][inner sep=0.75pt]  [rotate=-317.27] [align=left] {{\small траектория}};

			\draw (486.03,248.66) node [anchor=north west][inner sep=0.75pt]  [rotate=-316.21] [align=left] {{\small Фазовая}};

			\draw (264.11,188.54) node [anchor=north west][inner sep=0.75pt]    {$g^{t}$};			
		\end{tikzpicture}
		
	
		Так вот преобразование $g^t$ и называется фазовым потоком.
	\end{definition}


	Перечислим несколько полезных свойств введенного нами фазового потока:
	\begin{itemize}
		\item $g^{t_1 + t_2} = g^{t_1} \cdot g^{t_2} = g^{t_2} \cdot g^{t_1}$;
		
		\item $g^{t} \cdot g^{-t} = g^{-t} \cdot g^t = \text{Id}$ -- тождественное преобразование;
		
		\item Фазовый поток является группой;
		
		\item И еще сильнее, фазовый поток -- однопараметрическая группа, то есть каждому числу $t \in \mathbb{R}$ соответствует единственное преобразование $g^t: \; \mathscr{D} \rightarrow g^t \mathscr{D}$.
	\end{itemize}
	
	
	\begin{definition}
		Пусть у нас опять есть область $\mathscr{D}$ фазового пространства $\mathbb{R}^n$. Подействуем на $\mathscr{D}$ фазовым потоком $g^t$. Тогда $\mathscr{D}(t) = g^t \mathscr{D}$ и $\vec{x} = g^t \vec{x}_0$. Определим следующую величину как фазовый объем:
		\begin{equation*}
			V_\mathscr{D}(t) = \int\limits_{\mathscr{D}(t)} d\vec{x} = \int\limits_{g^t \mathscr{D}} d(g^t \vec{x}_0).
		\end{equation*}
	\end{definition}

	\subsection{Теорема Лиувилля}
	\begin{theorem}
		В автономной системе дифференциальных уравнений $\dfrac{d\vec{x}}{dt} = \vec{f}(\vec{x})$ производная фазового объема $V_\mathscr{D}(t)$ области $\mathscr{D} \subset \mathbb{R}^n$ фазового пространства может быть вычислена по формуле:
		\begin{equation*}
			\frac{dV_\mathscr{D}(t)}{dt} = \int\limits_{\mathscr{D}} \diverg\vec{f}\cdot d\vec{y},
		\end{equation*}
		где $\displaystyle \diverg \vec{f} = \sum\limits_{i = 1}^n \frac{\partial f^i}{\partial x^i}$ -- дивергенция $\vec{f}$, а $\vec{y} = \vec{x}(0)$.
	\end{theorem}

	\begin{proof}
		\ \\
		Докажем, что производная равна этому при $t = 0$, а в силу автономности системы это будет верно в каждой точке.
		
		Пишем производную по определению: $\displaystyle \frac{dV_\mathscr{D}}{dt}(0) = \lim\limits_{t \rightarrow 0} \frac{V_\mathscr{D}(t) - V_\mathscr{D}(0)}{t}$.
		
		Из системы имеем $\displaystyle \vec{x} = \vec{y} + \int\limits_{0}^t \vec{f}(\tau)d\tau$.
		
		При малых значениях $t$ получаем следующее: $\displaystyle x^i = y^i + f^i(\vec{y})t + o(t), t\rightarrow 0$.
		
		На все это дело можно смотреть как на замену координат $x^i \longrightarrow y^i$. Тогда получаем следующее выражение для фазового объема:
		\begin{equation*}
			V_\mathscr{D}(t) = \int\limits_{\mathscr{D}(t)} d\vec{x} \stackrel{\mathscr{D}(0) = \mathscr{D}}{=\joinrel=\joinrel=\joinrel=\joinrel=} \int\limits_{\mathscr{D}} |J| d\vec{y},
		\end{equation*}
		где $J = \dfrac{\partial (x^1, x^2, \dots, x^n)}{\partial (y^1, y^2, \dots, y^n)}$ -- якобиан преобразования.
		
		Посчитаем этот якобиан:
		\begin{equation*}
			J = \begin{vmatrix}
				1 + \dfrac{\partial f^1}{\partial y^1}t & \dfrac{\partial f^1}{\partial y^2}t & \cdots & \dfrac{\partial f^1}{\partial y^n}t \\
				\dfrac{\partial f^2}{\partial y^1}t & 1 + \dfrac{\partial f^2}{\partial y^2}t & \cdots & \dfrac{\partial f^2}{\partial y^n}t \\
				\vdots & \vdots & \ddots & \vdots \\
				\dfrac{\partial f^n}{\partial y^1}t & \dfrac{\partial f^n}{\partial y^2}t & \cdots & 1 + \dfrac{\partial f^n}{\partial y^n}t \\
			\end{vmatrix} = \left(1 + \dfrac{\partial f^1}{\partial y^1}t\right)\left(1 + \dfrac{\partial f^2}{\partial y^2}t\right)\dots\left(1 + \dfrac{\partial f^n}{\partial y^n}t\right) + o(t).
		\end{equation*}
		Здесь мы все, что имеет множители $t^2, t^3, \dots, t^n$, завернули в $o(t)$. Однако если раскрыть скобки, то такие слагаемые все еще остаются. Раскроем эти скобки и опять впихнем все ненужное в $o(t)$:
		\begin{equation*}
			J = 1 + \left(\frac{\partial f^1}{\partial y^1} + \frac{\partial f^2}{\partial y^2} + \dots + \frac{\partial f^n}{\partial y^n}\right)t + o(t) = 1 + t\diverg\vec{f} + o(t).
		\end{equation*}
		
		Ну, а теперь считаем эту производную:
		\begin{equation*}
			\frac{dV_\mathscr{D}}{dt} = \lim\limits_{t \rightarrow 0} \frac{V_\mathscr{D}(t) - 	V_\mathscr{D}(0)}{t} = \lim\limits_{t \rightarrow 0} \dfrac{\displaystyle \int_{\mathscr{D}} \left(1 + t\diverg\vec{f} + o(t)\right) d\vec{y} - \int_{\mathscr{D}} d\vec{y}}{t} = \int\limits_{\mathscr{D}} \diverg\vec{f}\cdot d\vec{y}.
		\end{equation*}
	\end{proof}

	\subsection{Теорема Пуанкаре}
	\begin{theorem}	
		Пускай $g^t$ - непрерывное взаимнооднозначное отображение, сохраняющее фазовый объем и переводящее ограниченную область $\mathscr{D}$ саму в себя, то есть $g^t\mathscr{D} = \mathscr{D}$. Тогда:
		\begin{equation*}
			\forall x_0 \in \mathscr{D} \longmapsto \forall U(x_0) \;\; \exists \overline{x} \in U(x_0):\; g^n \overline{x} \in U(x_0) \;\;\; (n = t_0),
		\end{equation*}
		где $U(x_0)$ -- некоторая окрестность точки $x_0$.
		
		Другими словами, для любой окрестности $U$ любой точки $x_0$ области $\mathscr{D}$ найдется точка $\overline{x}$, возвращающаяся обратно в эту же окрестность.
	\end{theorem}
\end{document}