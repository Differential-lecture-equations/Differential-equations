%%% Работа с русским языком
\usepackage{cmap}					% поиск в PDF
\usepackage{mathtext} 				% русские буквы в формулах
\usepackage[T2A]{fontenc}			% кодировка
\usepackage[utf8]{inputenc}			% кодировка исходного текста
\usepackage[russian]{babel}	% локализация и переносы

%%% Дополнительная работа с математикой
\usepackage{amsmath,amsfonts,amssymb,amsthm,mathtools} % AMS
\usepackage{icomma} % "Умная" запятая: $0,2$ --- число, $0, 2$ --- перечисление

%% Номера формул
%\mathtoolsset{showonlyrefs=true} % Показывать номера только у тех формул, на которые есть \eqref{} в тексте.
%\usepackage{leqno} % Немуреация формул слева

%% Шрифты
\usepackage{euscript}	 % Шрифт Евклид
\usepackage{mathrsfs} % Красивый матшрифт

%%% Свои команды
\DeclareMathOperator{\sgn}{\mathop{sgn}}

%% Поля
\usepackage[left=2cm,right=2cm,top=2cm,bottom=2cm,bindingoffset=0cm]{geometry}

%% Русские списки
\usepackage{enumitem}
\makeatletter
\AddEnumerateCounter{\asbuk}{\russian@alph}{щ}
\makeatother

%%% Работа с картинками
\usepackage{graphicx}  % Для вставки рисунков
\graphicspath{{images/}{images2/}}  % папки с картинками
\setlength\fboxsep{3pt} % Отступ рамки \fbox{} от рисунка
\setlength\fboxrule{1pt} % Толщина линий рамки \fbox{}
\usepackage{wrapfig} % Обтекание рисунков и таблиц текстом

%%% Работа с таблицами
\usepackage{array,tabularx,tabulary,booktabs} % Дополнительная работа с таблицами
\usepackage{longtable}  % Длинные таблицы
\usepackage{multirow} % Слияние строк в таблице

%% Красная строка
\setlength{\parindent}{2em}

%% Интервалы
\linespread{1}
\usepackage{multirow}

%% TikZ
\usepackage{tikz}
\usetikzlibrary{graphs,graphs.standard}

%% Верхний колонтитул
% \usepackage{fancyhdr}
% \pagestyle{fancy}

%% Перенос знаков в формулах (по Львовскому)
\newcommand*{\hm}[1]{#1\nobreak\discretionary{}
	{\hbox{$\mathsurround=0pt #1$}}{}}

%% дополнения
\usepackage{float} %Добавляет возможность работы с командой [H] которая улучшает расположение на странице
\usepackage{gensymb} %Красивые градусы
\usepackage{caption} % Пакет для подписей к рисункам, в частности, для работы caption*

% подключаем hyperref (для ссылок внутри  pdf)
\usepackage[unicode, pdftex]{hyperref}

%%% Теоремы
\theoremstyle{plain}                    % Это стиль по умолчанию, его можно не переопределять.
\renewcommand\qedsymbol{$\blacksquare$} % переопределение символа завершения доказательства

\newtheorem{theorem}{Теорема}[section] % Теорема (счетчик по секиям)
\newtheorem{proposition}{Утверждение}[section] % Утверждение (счетчик по секиям)
\newtheorem{definition}{Определение}[section] % Определение (счетчик по секиям)
\newtheorem{corollary}{Следствие}[theorem] % Следстиве (счетчик по теоремам)
\newtheorem{problem}{Задача}[section] % Задача (счетчик по секиям)
\newtheorem*{remark}{Примечание} % Примечание (можно переопределить, как Замечание)
\newtheorem{lemma}{Лемма}[section] % Лемма (счетчик по секиям)

\newtheorem{example}{Пример}[section] % Пример
\newtheorem{counterexample}{Контрпример}[section] % Контрпример

\begin{document}

\section{Элементы группового анализа ДУ}

Уравнение первого порядка в общем виде:

\begin{equation}
	P(x,y)dx + Q(x,y)dy = 0
	\label{Issue13_1}
\end{equation}

Если выполняется $\dfrac{\partial P}{\partial y} = \dfrac{\partial Q}{\partial x} \Rightarrow F(x,y) = const$ -- решение уравнения в полных дифференциалах.

Если же $\dfrac{\partial P}{\partial y} \neq \dfrac{\partial Q}{\partial x}$, то ищется интегрирующий множитель $\mu(x,y)$ :

$\mu Pdx + \mu Qdy = 0$ -- уже уравнение в полных дифференциалах.

\begin{equation}
	P\frac{\partial \mu}{\partial y} - Q \frac{\partial \mu}{\partial x} = \mu \dfrac{\partial Q}{\partial x} - \mu \dfrac{\partial P}{\partial y}
	\label{Issue13_2}
\end{equation}

\begin{equation}
	P(y)dx + \varphi(x)dy = 0,
	\label{Issue13_3}
\end{equation}

уравнение с разделяющимися переменными.

Если ДУ может быть приведено к виду ($\ref{Issue13_3}$), то оно интегрируемо. Рассмотрит, к каким переменным нужно перейти, чтобы уравнение $y' = \dfrac{dy}{dx} = f(x,y)$ свелось бы к уравнению с разделяющимися переменными.

\subsection{Однопараметрические группы}

Пусть имеется множество взаимно однозначных преобразований $\mathbb{R}^n: \tau(\mathbb{R}^n)$.
Это множество образуем группу (относительно композиции). Каждому $a \in \mathbb{R}$ соответствием $\varphi$ сопоставим преобразованние $g_a = \varphi(a) \in \tau(\mathbb{R}^n)$.

Следует ответить, что ассоциативность следует из ассоциативности матричного умножения.

Причем $\varphi(a+b) = \varphi(a)\cdot \varphi(b)$ и $\varphi(0) = E$, т.е. $\varphi$ осуществляет изоморфизм коммутативной группы $\mathbb{R}$ на группу $\tau(\mathbb{R}^n)$.Образ $\varphi(R) \in \tau(\mathbb{R}^n)$ называется однопараметрической группой преобразований.

Было доказано, что однопараметрической группой будет фазовый поток автоновной системы ДУ. Эта группа $g_a = g_a(M(\vec{x})) = M(\overrightarrow{\overline{x}})$ задается в виде:

$\overline{x^i} = \varphi^i(x^1,\ \dots\ ,x^n) = \varphi^i(\vec{x}, a)$, $i = \overline{1, n}$ или 
\begin{equation}
	\overrightarrow{\overline{x}} = \vec{\varphi}(\vec{x}, a)
	\label{Issue13_4}
\end{equation}

Т.к. группа коммутативна, то $\vec{\varphi}(\vec{x}, a + b) = \vec{\varphi}(\vec{\varphi}(\vec{x}, a), b) = \vec{\varphi}(\vec{\varphi}(\vec{x}, b), a)$, а $\vec{\varphi}(\vec{x}, 0) = \vec{x}$.

Будем предпологать, что вектор-функция $\vec{\varphi}(\vec{x}, a)$ непрерывно дифференцируема по всем своим аргументами.

Рассмотрим однопараметрическую группу преобразований плоскости $(x,y)(\mathbb{R}^2)$ --

\begin{equation}
	\begin{gathered}
		g_a = g_a(M(x,y)) \Rightarrow \vec{x} = \varphi(x,y,a)\ ,\ \vec{y} = \psi(x,y,a),\\
		\varphi(x,y,0) = x, \psi(x,y,0) = y
	\end{gathered}	
	\label{Issue13_5}
\end{equation}


% add definition
\begin{definition}

\textbf{траекторией (или орбита группы)} -- параметрическое предстачление кривой $\gamma$, проходящей через $(x;y)$, при фиксированных $x,y\ (\ref{Issue13_5})$.

\end{definition}


Кривая $\gamma$ при сделанных предположениях является $\textbf{гладкой кривой}$, поэтому с ней можно связать векторное поле, т.е. в каждой точке $M(x,y)$ поставим в соответствие вектор $\vec{h}(\xi(x,y), \zeta(x,y))$, касательный к $\gamma$, проходящей через эту точку.

Компоненты вектора $\vec{h}$, касательного к кривой $\gamma$ в точке $(x,y)$ равны

\[ \xi(x,y) = \dfrac{\partial \varphi}{\partial a}|_{a \rightarrow 0},\ \zeta(x,y) = \dfrac{\partial \psi }{\partial a}|_{a \rightarrow 0}, \]

а само векторное поле определено как отображение:

\begin{equation}
	(x,y) \rightarrow \partial g_a(M(x,y)) = \vec{h}(\xi(x,y), \zeta(x,y)) = \dfrac{dg_a}{da}|_{a \rightarrow 0}
	\label{Issue13_6}
\end{equation}

Это векторное поле называется $\textbf{касательным векторным полем}$ группы.

Рассмотрим 

\begin{equation}
	\begin{gathered}
		\dfrac{dg_{a+b}(M)}{db}|_{b\rightarrow 0} = \dfrac{d(g_a\cdot g_b)}{db}|_{b\rightarrow0} = \dfrac{d(g_b\cdot g_a)}{db}|_{b\rightarrow 0} = \\
		= (\dfrac{dg_a}{db}|_{b\rightarrow 0})g_a = \partial g_a (g_a(M(x,y))) = \partial g_a (x,y,a) = \vec{h}(\xi(x,y), \zeta(x,y)).
	\end{gathered}
\end{equation}

Т.к. $\vec{h}(\xi(x,y), \zeta(x,y))$ является косательным к $\gamma$ при фиксированным $a$, то кривая $\gamma$ является $\textbf{фазовой траекторией}$ автономной системы.

\begin{equation}
	\begin{cases}
		\dfrac{d\vec{x}}{da} = \xi(\vec{x}, \vec{y}) = \varphi_a'(\vec{x}, \vec{y}), \\
		\dfrac{d\vec{y}}{da} = \zeta(\vec{x}, \vec{y}) = \psi_a'(\vec{x}, \vec{y}),		
	\end{cases}
	\label{Issue13_7}
\end{equation}

Система $(\ref{Issue13_7})$ (она можеть записываться в виде $\partial_a g(x,y,a) = \vec{h}(g_a(x,y,a))$) называется $\textbf{уравнением Ли}$.

Ранее было получено, что любая автономная система определяет однопараметрическую группу преобразований (фазовый поток).


\begin{equation}
\text{Оператор } X = \xi \dfrac{\partial}{\partial x} + \zeta \dfrac{\partial}{\partial y} - \text{ генератор группы.}
\label{Issue13_8}
\end{equation}

Т.к. $X(u) = \xi \dfrac{\partial u}{\partial x} + \zeta \dfrac{\partial u}{\partial y}$, то становится ясно, что генератор группы является оператором дифференцирования в силу системы Ли (группы Ли) или оператором дифференцирования по направлению векторного поля группы.

\begin{definition}
Функция $F(x,y)$ называется $\textbf{инвариантом группы}$ $(\ref{Issue13_5})$, если \
$F(\vec{x}, \vec{y}) = F(x,y)\  \forall a$, т.е. $F$ постоянна на любой траектории ($\ref{Issue13_5}$).
\end{definition}


Т.о., если функция $F(x,y)$ является инвариантом группы, то $X(F(x,y)) = \xi \dfrac{\partial F}{\partial x} + \zeta \dfrac{\partial F}{\partial y} = \xi \cdot 0 + \zeta \cdot 0 = 0$, и т.о. инвариант группы ($\ref{Issue13_5}$) является просто первым интегралом ($\ref{Issue13_7}$).

Расммотрим группы $\vec{x} = x + a,\ \vec{y} = y$ -- группа смещений $\Rightarrow$ генератор группы $X = 1 \dfrac{\partial}{\partial x} + 0 \dfrac{\partial}{\partial y} = \dfrac{\partial}{\partial x}$, а инвариантом этой группы является любой $F(x,y) = f(y)$.

\begin{theorem}
Любая однопараметрическая группа с генератором $\ref{Issue13_8}$ может быть с помощью подходящей замены 

\begin{equation}
	t = t(x,y),\ u = u(x,y)
	\label{Issue13_9}
\end{equation}

приведена к группе смещений 

\begin{equation}
	\vec{t} = t + a,\ \vec{u} = u.
	\label{Issue13_10}
\end{equation}

\textbf{Замечание:} в новых переменных генератор имеет вид $X = \dfrac{\partial}{\partial t}$, и инвариант группы остается инвариантом и в новых переменных (см. инвариантность ПИ относительно гладкой замены).

\begin{proof}

Имеется \[\xi \dfrac{\partial}{\partial x} + \zeta \dfrac{\partial}{\partial y} = \xi(\dfrac{\partial}{\partial t}t_x' + \dfrac{\partial}{\partial u}u_x') + \zeta(\dfrac{\partial}{\partial t}t_y' + \dfrac{\partial}{\partial u}u_y') = X(t)\dfrac{\partial}{\partial t} + X(u)\dfrac{\partial}{\partial u}.\]

Отсюда получаем, что функции $(\ref{Issue13_9})$, которые приводят группу к группе смещений, должны удовлетворять условиям:

\begin{equation}
	X(t) = 1 \Rightarrow \xi \dfrac{\partial t}{\partial x} + \zeta \dfrac{\partial t}{\partial y} = 1;\ X(u) = 0 \Rightarrow \xi \dfrac{\partial u}{\partial x} + \zeta \dfrac{\partial u}{\partial y} = 0.
	\label{Issue13_11}
\end{equation}

Так, определенные переменные $t$ и $u$ называются $\textbf{каноническими переменными}$.

Заметим, что переменные и являются инвариантом исходной группы, поскольку \\
 $X(u) = 0$

\end{proof}
\end{theorem} 


\begin{theorem}

Орбиты группы либо совпадают, либо не пересекаются.

\begin{proof}

Пусть произошло пересечение: $g_a(M,a) = g_b(M_1,b)$, причем $M_1$ не принадлежит орбите точки $M$. Пусть $b < a$, подействуем $g_{-b}$ на последнее равенство:

\[ g_{-b}(g_a(M,1)) = g_{-b}(g_b(M_1, b)) \Rightarrow g_{-b+a}(M,a) = E(M_1) \Rightarrow\]

т. $M_1$ принадлежит орбите т.$M$ -- противоречие.

\end{proof}

\end{theorem}


Рассмотрим ДУ: 

\begin{equation}
 	y' = \dfrac{dy}{dx} = f(x,y)
 	\label{Issue13_12}
\end{equation}

Будем говорить, что группа $g_a$ является $\textbf{группой симметрии}$ ДУ ($\ref{Issue13_12}$) (или ($\ref{Issue13_12}$) допускает группу $g_a$), если форма ДУ ($\ref{Issue13_12}$) остается неизменной после замены переменных при замене 

\begin{equation}
	\begin{cases}
	\overline{x} = \varphi(x,y,a), \\
	\overline{y} = \psi(x,y,a)
	\end{cases}
\end{equation}

т.е. $\dfrac{d\overline{y}}{d\overline{x}} = f(\overline{x}, \overline{y})$, где $f$ то же самое, что и в ($\ref{Issue13_12}$).

Если ДУ ($\ref{Issue13_12}$) допускает группу, то тогда $f(\overline{x}, \overline{y}) = f(x,y)\ \forall a$, и правая часть ($\ref{Issue13_12}$) является инвариантом группу. Тогда, перейдя к каноническим переменным, получим, что в таких переменных $t$ и $u$ уравнение примет вид:

\begin{equation}
	\dfrac{du}{dt} = g(u),
	\label{Issue13_13}
\end{equation}

т.е. получили уравнение с разделяющимися переменными.

%\textbf{Пример:} рассмотрим группу растяжений $\overline{x} = e^ax,\ \overline{y} = e^{\alpha a}y$ и найдем канонические переменные:

%$\xi = x,\ \zeta = \alpha y$.

%\[x\dfrac{\partial t}{\partial x} + \alpha y \dfrac{\partial t}{\partialt y} = 1.\]

%Ищем решение в виде $t = t(x):\ x \dfrac{\partial t}{\partial x} = 1 \Rightarrow t = ln|x|$.



















\end{document}
 
