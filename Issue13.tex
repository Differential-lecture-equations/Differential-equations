%%% Работа с русским языком
\usepackage{cmap}					% поиск в PDF
\usepackage{mathtext} 				% русские буквы в формулах
\usepackage[T2A]{fontenc}			% кодировка
\usepackage[utf8]{inputenc}			% кодировка исходного текста
\usepackage[russian]{babel}	% локализация и переносы

%%% Дополнительная работа с математикой
\usepackage{amsmath,amsfonts,amssymb,amsthm,mathtools} % AMS
\usepackage{icomma} % "Умная" запятая: $0,2$ --- число, $0, 2$ --- перечисление

%% Номера формул
%\mathtoolsset{showonlyrefs=true} % Показывать номера только у тех формул, на которые есть \eqref{} в тексте.
%\usepackage{leqno} % Немуреация формул слева

%% Шрифты
\usepackage{euscript}	 % Шрифт Евклид
\usepackage{mathrsfs} % Красивый матшрифт

%%% Свои команды
\DeclareMathOperator{\sgn}{\mathop{sgn}}

%% Поля
\usepackage[left=2cm,right=2cm,top=2cm,bottom=2cm,bindingoffset=0cm]{geometry}

%% Русские списки
\usepackage{enumitem}
\makeatletter
\AddEnumerateCounter{\asbuk}{\russian@alph}{щ}
\makeatother

%%% Работа с картинками
\usepackage{graphicx}  % Для вставки рисунков
\graphicspath{{images/}{images2/}}  % папки с картинками
\setlength\fboxsep{3pt} % Отступ рамки \fbox{} от рисунка
\setlength\fboxrule{1pt} % Толщина линий рамки \fbox{}
\usepackage{wrapfig} % Обтекание рисунков и таблиц текстом

%%% Работа с таблицами
\usepackage{array,tabularx,tabulary,booktabs} % Дополнительная работа с таблицами
\usepackage{longtable}  % Длинные таблицы
\usepackage{multirow} % Слияние строк в таблице

%% Красная строка
\setlength{\parindent}{2em}

%% Интервалы
\linespread{1}
\usepackage{multirow}

%% TikZ
\usepackage{tikz}
\usetikzlibrary{graphs,graphs.standard}

%% Верхний колонтитул
% \usepackage{fancyhdr}
% \pagestyle{fancy}

%% Перенос знаков в формулах (по Львовскому)
\newcommand*{\hm}[1]{#1\nobreak\discretionary{}
	{\hbox{$\mathsurround=0pt #1$}}{}}

%% дополнения
\usepackage{float} %Добавляет возможность работы с командой [H] которая улучшает расположение на странице
\usepackage{gensymb} %Красивые градусы
\usepackage{caption} % Пакет для подписей к рисункам, в частности, для работы caption*

% подключаем hyperref (для ссылок внутри  pdf)
\usepackage[unicode, pdftex]{hyperref}

%%% Теоремы
\theoremstyle{plain}                    % Это стиль по умолчанию, его можно не переопределять.
\renewcommand\qedsymbol{$\blacksquare$} % переопределение символа завершения доказательства

\newtheorem{theorem}{Теорема}[section] % Теорема (счетчик по секиям)
\newtheorem{proposition}{Утверждение}[section] % Утверждение (счетчик по секиям)
\newtheorem{definition}{Определение}[section] % Определение (счетчик по секиям)
\newtheorem{corollary}{Следствие}[theorem] % Следстиве (счетчик по теоремам)
\newtheorem{problem}{Задача}[section] % Задача (счетчик по секиям)
\newtheorem*{remark}{Примечание} % Примечание (можно переопределить, как Замечание)
\newtheorem{lemma}{Лемма}[section] % Лемма (счетчик по секиям)

\newtheorem{example}{Пример}[section] % Пример
\newtheorem{counterexample}{Контрпример}[section] % Контрпример

\begin{document}

\section{Матричная экспонента, ее свойства и применение к решению нормальных линейных систем}

\subsection{Матричная экспонента}
Необходимо решить ОЛДУ вида:

\begin{equation}
  	\frac{d\vec{x}}{dt} = A\vec{x},\ \vec{x}(t_0) = \vec{x_0},
  	\label{Issue5_1}
\end{equation}

Если $A(t) = ||a_j^i||,\ a_j^i \in \mathbf{R},\ i,j = 1,\dots,\ n$, тогда:

\begin{equation*}
\begin{gathered}
         \vec{x_{0}} = E\vec{x_0},\ \vec{x_1} = E\vec{x_0} + \frac{t-t_0}{1!}A\vec{x_0} = \left(E + \frac{t-t_0}{1!}A\right) \vec{x_0}, \\  
         \vec{x_n} = \left(E + \frac{t-t_0}{1!}A +\ \dots\ + \frac{(t-t_0)^n}{n!}A^n\right)\vec{x_0},	 
\end{gathered}
\end{equation*}

Этот процесс будет сходиться к задаче Коши с решением:

\begin{equation*}
   	\vec{x} = \left(E + \frac{t-t_0}{1!}A +\ \dots\ + \frac{(t-t_0)^n}{n!}A^n+\ \dots \right)\vec{x_0} = \left(\sum\limits_{n = 0}^{\infty} \frac{(t-t_0)^n}{n!}A^n\right)\vec{x_0},	 
\end{equation*}

при условии, что $A^0 = E.$

\begin{definition}
	Матричной экспонентой называют следующий степенной ряд:
	\begin{equation*}
		e^{(t-t_0)A} = E + \frac{t-t_0}{1!}A +\ \dots\ + \frac{(t-t_0)^n}{n!}A^n+\ \dots\ = \sum\limits_{n = 0}^{\infty} \frac{(t-t_0)^n}{n!}A^n.
	\end{equation*}
\end{definition}

\subsection{Свойства матричной экспоненты}

Это квадратная матрица, по размерам аналогична матрице $A$, и каждый элемент этой матрицы представляет из себя степенной ряд с радиусом сходимости $+\infty$.

\begin{enumerate}
		\item Решение задачи Коши для $(\ref{Issue5_1})$, если $A = const$:

		\begin{equation*} 
			\vec{x}(t) = e^{(t-t_0)A}\vec{x_0},\ (\vec{x}(t_0) = \vec{x_0}) .
		\end{equation*}
		
		\item $ e^{0 A} = E.$
		
		\item $e^{(t_1+t_2)A} = e^{t_1A}e^{t_2A} \Rightarrow e^{t_1A}e^{t_2A} = e^{t_2A}e^{t_1A}$ (коммутативность).
		
		\item $\left(e^{tA}\right)^{-1} = e^{-tA}.$
		
		\item $(e^{tA})^{'} = A e^{tA} = e^{tA}A.$

\end{enumerate}

\begin{proof}

Так как квадратные матрицы составляют определенное кольцо, то \\
$A^{n+m} = A^nA^m=A^mA^n.$

\begin{enumerate}

\item

\item $ e^{tA} = E + \frac{t-t_0}{1!}A +\ \dots\ + \frac{(t-t_0)^n}{n!}A^n+\ \dots $, если $t = 0$ :
\[ e^{0A} = E + 0 + \dots = E\]

\item рассматриваем $(\ref{Issue5_1})$, если $\vec{x}(t)$ - решение этого ДУ, то $\vec{x}(t+t_0)$ тоже решение этого ДУ $\forall t_0 \in \mathbf{R}$. ($u = t + t_0$) :

\[ \frac{d\vec{x}(t+t_0)}{dt} = \frac{d\vec{x}}{du}\frac{du}{dt} = \frac{d\vec{x}}{du} = A\vec{x}(u) = A\vec{x}(t+t_0).\]

Тогда $(\ref{Issue5_1})$, с задачей Коши $\vec{x}(0) = \vec{x_0}$ имеет решение:

\[ \vec{x}(t) =  e^{tA}\vec{x_0},\]
\[ \vec{x}(t+t_0) = e^{(t+t_0)}\vec{x_0}\text{ - решение }\frac{d\vec{x}}{dt} = A\vec{x}. \]

Рассмотрим тогда тоже самое уравнение для функции $z(t)$:

\[ \frac{d\vec{z}}{dt} = A\vec{z}, \text{ с задачей Коши } \vec{z}(0) = e^{t_0A}\vec{x_0} \Rightarrow \vec{z}(t) = e^{tA} (e^{t_0A}\vec{x_0}) = (e^{tA}e^{t_0A})\vec{x_0}.\]

Рассмотрим это решение в нуле:

\[ \vec{x}(0 + t_0) = e^{t_0A}\vec{x_0},\]

из основной теоремы следует, что $\vec{x}(t+t_0) = \vec{z}(t)\ \forall t$.

Тогда и получается основная формула:

\[ \vec{x}(t+t_0) = e^{(t+t_0)A}\vec{x_0} = (e^{tA}e^{t_0A})\vec{x_0}\]

\item $E = e^{0A} = e^{(t-t)A} = e^{tA}e^{-tA} = E \Rightarrow \left(e^{tA}\right)^{-1} = e^{-tA}.$

\item Берем представление матричной экспоненты в виде степенного ряда, который можно дифференцировать, тогда получаем:

\[ (e^{tA})^{'} = A + tA^2 +\ \dots\ +\frac{t^{n-1}}{(n-1)!} A^n +\ \dots = A\left(E + tA +\ \dots\ + \frac{t^{n-1}}{(n-1)!}A^{n-1}\right),\]

\[ (e^{tA})^{'} = Ae^{tA} = e^{tA}A.\]

\end{enumerate}

\end{proof}

\textbf{Примечание.} Формула $e^{t(A+B)} = e^{tA}e^{tB}$ не имеет места, кроме случая, если $AB = BA$ (т.е. матрицы коммутативны).

\subsection{Применение к решению нормальных линейных систем}

\begin{theorem}

Пусть $S$ - матрица перехода от исходного базиса к новому базису. Тогда в новой базисе $\overline{A} = S^{-1}AS$, или $A = S\overline{A}S^{-1}$. И главное:

\[e^{tA} = S^{-1}e^{t\overline{A}}S.\]

\end{theorem}

\begin{proof}
	\begin{equation*}
		\begin{gathered}
 e^{tA} = \left(E + tA +\ \dots\ + \frac{t^{n}}{n!}A^{n}\right) = \left(E + tS^{-1}e^{t\overline{A}}S +\ \dots\ + \frac{t^{n}}{n!}(S^{-1}e^{t\overline{A}}S)^{n}\right), \\
 (S\overline{A}S^{-1})^n = S\overline{A}^nS^{-1},\ SES^{-1} = SS^{-1} = E \\
 e^{tA} = S^{-1}e^{t\overline{A}}S.
		\end{gathered}
	\end{equation*}
\end{proof}

Для решения нормальных линейных систем методом матричной экспоненты мы будем находить собственные вектора.

Матрица $A$ в базисе из собственных векторов (если они соответствуют действительным собственным значениям) будет иметь диагональный вид. Произведение диагональной матрицы на диагональную -- диагональная. Тогда для случая без кратных корней:

\[ e^{tA} = E + t\cdot diag(\lambda_1,\ ...\ ,\lambda_n) + \frac{t^n}{n!}\cdot diag(\lambda_1^n,\ ...\ ,\lambda_n^n).\]

\[ e^{tA} = diag(e^{t\lambda_1},\ ...\ , e^{t\lambda_n}).\]

Если $\lambda$ -- корень кратности $l$, то матрица $A$ приводится к Жордановой клетке (диагональная матрица с единицами над главной диагональю).

\[ A = \lambda E + B \Rightarrow B = A - \lambda E. \]

\[ e^{tA} = e^{t(\lambda E + B)} = e^{t\lambda E}e^{tB},\ e^{t\lambda E} = diag(e^{t\lambda},\ ...\ , e^{t\lambda}), e^{tB} = E + tB +\ ...\ + \frac{t^{l-1}}{(l-1)!}B^{l-1} + 0 \]

\begin{equation*}
	\text{тогда } e^{tA} = e^{\lambda t}
 	\begin{pmatrix}
            1\ \ t\ \ \ \ \  \dots\ \ \ \ \ \frac{t^{l-1}}{(l-1)!} \\
            0\ \ 1\ \ t\ \ \ \dots\ \ \ \ \  \frac{t^{l-2}}{(l-2)!} \\
            \dots \\
            0\ \ \ \ \ \ \ \ \ \dots\ \ \ \ \ \ \ \ \  1
    \end{pmatrix}
\end{equation*}

\textbf{Метод решения линейных неоднородных уравнений с постоянными коэффициентами} (матричный метод вариации постоянной)

\[ \frac{d\vec{x}}{dt} = A\vec{x} + \vec{f}(t),\ \ \text{ решение будем искать в виде } \ \vec{x}(t) = e^{tA}\vec{C}(t), \]

\[ \text{ тогда } Ae^{tA}\vec{C}(t) + e^{tA}\dot{\vec{C}}(t) = Ae^{tA}\vec{C} + \vec{f}(t),\]

\[ e^{tA}\dot{\vec{C}}(t) = \vec{f}(t)\ \Rightarrow \dot{\vec{C}}(t) = (e^{tA})^{-1}\vec{f}(t) = e^{-tA}\vec{f}(t). \]

\newpage

\section{Теоремы существования и единственности решения задачи Коши для нормальной линейной системы уравнений и
для линейного уравнения $n$-го порядка в нормальном виде}

Рассматривается система вида 

\begin{equation}
	\frac{d\vec{x}}{dt} = A\vec{x} + \vec{q}(t),
	\label{Issue5_2}
\end{equation} 

где $A = ||a_j^i(t)||$, $i,j = \vec{1,n}$ -- матрица, $\vec{q}(t)$ -- заданная вектор-фенкция. Наряду с векторной записью также будем использовать координатную запись $\dot{x}^i = \sum\limits_{j = 1}^{n} a_j^i x^j + q^i(t),\ i = \vec{1,n}$.

$\textbf{Необходимым условием линейности}$ является факт того, что все $A_j^i$ и $q^i$ зависят только от $t$ и не зависят от $\vec{x}$.

Для $(\ref{Issue5_2})$ ставится задача Коши:

\[ \vec{x}(t_0) = \vec{x_0}.\]

\begin{theorem}

$\textbf{Основная теорема для линейных систем.}$ Пусть $a_j^i(t),\ i,j = \vec{1,n}$ и $\vec{q}(t)$ в $(\ref{Issue5_2})$ непрерывны на отрезке $[a;b]$. Тогда рпешение задачи Коши существует и единственно на всем отрезке $[a;b].$

\end{theorem}

$\textbf{Предварительные замечания:}$

Пусть вектор-функция $\vec{f}(x) \in B$ и $A$ -- линейный оператор, действубщий из $B$ в $B$, т.е. $A(\vec{f} + \vec{g}) = A\vec{f} + A\vec{g}$.

$\textbf{Определим норму оператора:}$

\[ ||A|| =  \sup\limits_{\vec{\varphi} \in B,\ \vec{\varphi} \neq \vec{0}} \frac{||A(\vec{\varphi})||}{||\vec{\varphi}||}. \]

Тогда получаем неравенство: $||A|| \\leqslant ||A||||\vec{\varphi}||$.

Нормой для вектор-функции выберем $||\vec{x}(t)|| = \max\limits_{1 \leqslant i \leqslant n} (\max\limits_{t \in [a;b]} x^i(t)),$ а нормой для оператора $||A|| = = \max\limits_{1 \leqslant i \leqslant n} (\max\limits_{t \in [a;b]} \sum\limits_{j = 1}^{n} |a_j^i(t)|)$

\begin{proof}

Определим $\vec{g}(t) = \vec{x_0} + \int\limits_{t_0}^{t} \vec{q}(S)dS$ и построим итерационную процедуру.

Т.к $q^i(t) \in C_{[a;b]}\ \forall i = \vec{1,n} \Rightarrow \exists ||\vec{q}||_c = M_1.$ Тогда $||\vec{g}||_c = ||\vec{x_0} + \int\limits_{t_0}^{t}\vec{q}(S)dS|| \leqslant ||\vec{x_0}|| + ||\int\limits_{t_0}^{t}\vec{q}(S)dS|| \leqslant ||\vec{x_0}|| + M_1(b-a) = C.$

Рассмотрим интегральное уравнение $\vec{x} = \vec{g} + \int\limits_{t_0}^{t}A(s)\vec{x}(s)ds$.

Аналогично основной лемме доказывается, что последнее интегральное уравнение эквивалентно задаче $(\ref{Issue5_2})$.

Итерационная процедура: $\vec{x_0} = \vec{g}$; $\vec{x_k} = \vec{g} + \int\limits_{t_0}^{t} A(s)\vec{x_{k-1}}(s)ds $, $k = 0,1,\ \dots$

Оценим норму:

$$ ||\vec{x_1} - \vec{x_0}|| = || \int\limits_{t_0}^{t} A(s)\vec{g}(s)ds || \leqslant | \int\limits_{t_0}^{t} ||A(s)\vec{g}(s)||ds | \leqslant  | \int\limits_{t_0}^{t} ||A(s)||\cdot ||\vec{g}(s)||ds | \leqslant C_1 C |t - t_0|;$$ 

Таким образом $||\vec{x_1} -\vec{x_0}|| \leqslant C_1C|t-t_0|.$

Теперь докажем по индукции неравенство: $||\vec{x_k} - \vec{x_{k-1}}|| \leqslant \frac{CC_1^k}{k!}|t-t_0|^k.$

Базой индукции выступает полученное выше неравенство. Предположим, что верно для $n = k$, т.е.: $||\vec{x_k} - \vec{x_{k-1}}|| \leqslant \frac{CC_1^k}{k!}|t-t_0|^k.$

Докажем для 

\[n = k + 1:\  ||\vec{x_{k+1}} - \vec{x_{k}}|| = || \int\limits_{t_0}^{t} A(s)(\vec{x_k}(s) - \vec{x_{k-1}}(s))ds || \leqslant | \int\limits_{t_0}^{t} ||A(s)(\vec{x_k}(s) - \vec{x_{k-1}}(s))|| ds | \leqslant \]

\[ \leqslant | \int\limits_{t_0}^{t} ||A(s)||\cdot ||(\vec{x_k}(s) - \vec{x_{k-1}}(s))|| ds | \leqslant C |\int\limits_{t_0}^{t} \frac{C_1C^k|s-t_0|^k}{k!}ds | = \frac{C^{k+1}C_1|t-t_0|^{k+1}}{(k+1)!} \]

Т.к. $|t-t_0| \leqslant (b-a)$, то предыдущее неравенство можно усилить $||\vec{x_k} - \vec{x_{k-1}}|| \leqslant \frac{C_1C^k}{k!}(b-a)^k.$

Функциональная последовательно $\vec{x_k}$ сходиться равномерно, т.к. сходится равномерно ряд $\vec{x_0} + (\vec{x_1} - \vec{x_0}) +\ \dots\ +(\vec{x_k} - \vec{x_{k-1}}) +\ \dots$, который межорируется сходящимся рядом $||\vec{x_0}|| + ||(\vec{x_1} - \vec{x_0})|| +\ \dots\ + ||(\vec{x_k} - \vec{x_{k-1}})|| +\ \dots \leqslant ||\vec{x_0}|| + C_1\sum\limits_{k = 0}^{\infty}\frac{C^k|b-a|^k}{k!} = ||\vec{x_0}|| + C_1 e^{C(b-a)} < \infty \Rightarrow$ Существует (в сиду банаховости пр-ва) непрерывно дифф. $\vec{\varphi(t)}:$ $\exists \lim\limits_{n \rightarrow \infty} \vec{x_n} = \varphi(t).$

Рассмотрим $|| \int\limits_{t_0}^{t} A\vec{x_n}dS - \int\limits_{t_0}^{t} A\vec{\varphi} dS || = || \int\limits_{t_0}^{t} A(\vec{x_n} - \vec{\varphi})dS || \leqslant ||A||\cdot | \int\limits_{t_0}^{t} ||\vec{x_n} - \vec{\varphi}|| dS | $, где $||\vec{x_n} - \vec{\varphi} || \rightarrow_{n\rightarrow \infty} 0.$

Т.о. итерационная процедура сходится в силу существования пределов слева и справа.

Полученное решение эквивалентно решению задачи $(\ref{Issue5_2})$. В отличии от основной теоремы для нормальных систем ДУ: $\dot{\vec{x}} = \vec{f}(t, \vec{x})$, где существование было получено только на отрезке Пеано, для СЛДУ существование решения доказано для всего отрезка $[a;b]$ -- промежутка, где $a_j^i(t)$ и $\vec{q}(t)$ непрерывны. В нашем случае $\vec{f}$ соответствует $\vec{f} = A\vec{x} + \vec{q}$. Она непрерывна, т.к. полученное решение $\vec{x}(t)$ непрерывно. Условие непрерывности $\frac{\partial f}{\partial x_i}$ также выполнены, т.к. в нашем случае $\frac{\partial f}{\partial x_i} = a_{ij}(t)$ -- непр. на $[a;b]$. Отсюда следует единственность, т.к. два решения задачи $(\ref{Issue5_2})$, согласно основной теореме для нормальныэ систем, совпадает на промежутке, где они оба определены. В нашем случае это $[a;b]$.

Т.о. теорема не носит локальных характер.

\end{proof}



















\end{document}
 
